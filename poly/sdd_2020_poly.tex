%-*- coding: iso-latin-1 -*-
\documentclass[french,11pt]{book}
\usepackage{babel}

% Font
\usepackage[latin1]{inputenc}
\usepackage[T1]{fontenc}
\usepackage{gentium}
% \usepackage{structure}

\usepackage[tmargin=2cm,bmargin=2cm,lmargin=3cm,footnotesep=1cm]{geometry}

\usepackage{multirow}

\usepackage[sectionbib]{natbib}
%\newcommand{\mycite}[1]{\cite{#1}\footnote{\printbibliography}}

\let\cleardoublepage\clearpage


% % Math symbols
\usepackage{amsmath}
\usepackage{amssymb}
\usepackage{bm}

% Side-by-side figures
\usepackage{subcaption}

% Table becomes Tableau
\usepackage{caption}
\captionsetup{labelfont=sc}
\def\frenchtablename{Tableau}

\usepackage{graphicx}

\usepackage[colorlinks=true,allcolors=black]{hyperref}

% Math symbols
\newcommand{\Acal}{\mathcal{A}}
\newcommand{\Bcal}{\mathcal{B}}
\newcommand{\Ccal}{\mathcal{C}}
\newcommand{\Ecal}{\mathcal{E}}
\newcommand{\Gcal}{\mathcal{G}}
\newcommand{\Kcal}{\mathcal{K}}
\newcommand{\Ncal}{\mathcal{N}}
\newcommand{\Pcal}{\mathcal{P}}
\newcommand{\Qcal}{\mathcal{Q}}
\newcommand{\Rcal}{\mathcal{R}}
\newcommand{\Scal}{\mathcal{S}}
\newcommand{\Tcal}{\mathcal{T}}

\newcommand{\DD}{\mathcal{D}}
\newcommand{\FF}{\mathcal{F}}
\newcommand{\HH}{\mathcal{H}}
\newcommand{\LL}{\mathcal{L}}
\newcommand{\MM}{\mathcal{M}}
\newcommand{\OO}{\mathcal{O}}
\newcommand{\TT}{\mathcal{T}}
\newcommand{\UU}{\mathcal{U}}
\newcommand{\VV}{\mathcal{V}}
\newcommand{\XX}{\mathcal{X}}
\newcommand{\YY}{\mathcal{Y}}
\newcommand{\ZZ}{\mathcal{Z}}

\newcommand{\CC}{\mathbb{C}}
\newcommand{\EE}{\mathbb{E}}
\newcommand{\NN}{\mathbb{N}}
\newcommand{\PP}{\mathbb{P}}
\newcommand{\RR}{\mathbb{R}}


% Vectors
\makeatletter
\newcommand{\avec}{\vec{a}\@ifnextchar{^}{\,}{}}
\newcommand{\bb}{\vec{b}\@ifnextchar{^}{\,}{}}
\newcommand{\cvec}{\vec{c}\@ifnextchar{^}{\,}{}}
\newcommand{\gvec}{\vec{g}\@ifnextchar{^}{\,}{}}
\newcommand{\hh}{\vec{h}\@ifnextchar{^}{\,}{}}
\newcommand{\mm}{\vec{m}\@ifnextchar{^}{\,}{}}
\newcommand{\oo}{\vec{o}\@ifnextchar{^}{\,}{}}
\newcommand{\pp}{\vec{p}\@ifnextchar{^}{\,}{}}
\newcommand{\rr}{\vec{r}\@ifnextchar{^}{\,}{}}
\newcommand{\tvec}{\vec{t}\@ifnextchar{^}{\,}{}}
\newcommand{\uu}{\vec{u}\@ifnextchar{^}{\,}{}}
\newcommand{\uprime}{\vec{u'}\@ifnextchar{^}{\,}{}}
\newcommand{\vv}{\vec{v}\@ifnextchar{^}{\,}{}}
\newcommand{\vprime}{\vec{v'}\@ifnextchar{^}{\,}{}}
\newcommand{\ww}{\vec{w}\@ifnextchar{^}{\,}{}}
\newcommand{\xx}{\vec{x}\@ifnextchar{^}{\,}{}}
\newcommand{\yy}{\vec{y}\@ifnextchar{^}{\,}{}}
\newcommand{\zz}{\vec{z}\@ifnextchar{^}{\,}{}}
\newcommand{\aalpha}{\vec{\alpha}\@ifnextchar{^}{\,}{}}
\newcommand{\bbeta}{\vec{\beta}\@ifnextchar{^}{\,}{}}
\newcommand{\ttheta}{\vec{\theta}\@ifnextchar{^}{\,}{}}
\newcommand{\mmu}{\vec{\mu}\@ifnextchar{^}{\,}{}}
\newcommand{\xxi}{\vec{\xi}\@ifnextchar{^}{\,}{}}
\makeatother

\newcommand{\thetamle}{{\hat \theta_{\text{MLE}}}}
\newcommand{\thetabayes}{{\hat \theta_{\text{Bayes}}}}
\newcommand{\thetahat}{{\hat \theta}}

\DeclareMathOperator*{\argmax}{arg\,max}
\DeclareMathOperator*{\argmin}{arg\,min}

\newcommand{\lzeronorm}[1]{\left|\left|#1\right|\right|_0}
\newcommand{\lonenorm}[1]{\left|\left|#1\right|\right|_1}
\newcommand{\ltwonorm}[1]{\left|\left|#1\right|\right|_2}


% TODO
\usepackage[dvipsnames]{xcolor}
\newcommand{\todo}[1]{{\color{BrickRed}{TODO~#1}}}
\newcommand{\rewrite}[1]{{\color{RoyalBlue}{#1}}}

% Emacs: to save in encoding iso-latin-1:
% C-x C-m f
% iso-latin-1

% aspell --lang=fr --encoding='iso-8859-1' -t check selection-modele.tex


\begin{document}

\renewcommand{\bibname}{}


\begin{center}
  
  \hfill

  \vfill
  
  \LARGE
  ECUE2.1 Science des donn�es \\ 
  
  \Large
  Chlo�-Agathe Azencott (CBIO) \\

  Printemps 2020 -- Mines ParisTech

  \vfill

  \large
  \textbf{Comp�tences}
  \begin{tabular}[h]{|p{0.2\textwidth}|p{0.8\textwidth}|}
    \hline
    C1 & Ma�triser des m�thodes statistiques usuelles permettant de traiter convenablement des cas simples d'analyse de donn�es \\ \hline
    C2 & Ma�triser des m�thodes usuelles d'exploration des donn�es \\ \hline
    C3 &  Conna�tre les limites d'applications des m�thodes vues en cours \\ \hline
    C4 & Pouvoir se r�f�rer � un cas d'application avec des donn�es r�elles en lien avec une discipline autre que celle de l'analyse des donn�es \\ \hline
    C5 & Savoir �valuer la complexit� num�rique de quelques algorithmes \\ \hline
    C6 & Conna�tre des m�thodes d'apprentissage statistique (machine learning) supervis� et des m�thodes d'apprentissage statistique non supervis� \\ \hline
    C7 & Savoir valider et s�lectionner un mod�le statistique \\ \hline
  \end{tabular}

  \vfill

\end{center}



\chapter{Intro}
%-*- coding: iso-latin-1 -*-
\label{chap:intro}

\section{Continuit� p�dagogique}
\todo{Mise au point sur les modalit�s d'enseignement � distance.}


\section{Qu'est-ce que la science des donn�es ?}
\todo{D�finir donn�es, science des donn�es, donner des exemples d'application
  diverses et vari�es.}

Le premier but de ce cours est de d�mystifier la science des donn�es, le Big
Data, l'intelligence artificielle telle qu'on en parle de nos jours et de vous
donner les cl�s n�cessaires � recevoir les informations sur le sujet d'un
\oe{}il critique.

Le deuxi�me but de ce cours est de poser les bases math�matiques et
algorithmiques de l'exploitation de donn�es. Les domaines de la statistique
inf�rentielle et de l'apprentissage automatique sont vastes, et vous aurez, si
vous le souhaitez, amplement l'occasion de les explorer en deuxi�me et troisi�me
ann�e.

\section{Qu'est-ce que la statistique ?}

Le terme de � statistique � est d�riv� du latin � \textit{status} � (signifiant
� �tat �). Historiquement, \textbf{les statistiques} concernent l'�tude
m�thodique, par des proc�d�s num�riques (inventaires, recensements, etc.) des
faits sociaux qui d�finissent un �tat. Elles sont d�sormais utilis�es dans tous
les secteurs o� l'on dispose de donn�es : sciences sociales mais aussi sant�,
environnement, industrie, �conomie, recherche scientifique, etc.

Par contraste, \textbf{la statistique} est un ensemble de m�thodes des
math�matiques appliqu�es permettant de d�crire et d'analyser des ph�nom�nes
dont la nature rend une �tude exhaustive de tous leurs facteurs impossible. Ces
m�thodes permettent d'�tudier des donn�es, ou observations, consistant en la
mesure d'une ou plusieurs caract�ristiques d'un ensemble de personnes ou objets
�quivalents.

\section{Vocabulaire}

L'ensemble de personnes ou d'objets �quivalents �tudi�s est appel� \textbf{la
  population.} Il peut s'agir d'une population au sens � courant � du terme
(par exemple, l'ensemble de la population fran�aise, ou l'ensemble des
individus d'une esp�ce animale sur un territoire) mais aussi plus largement
d'un ensemble plus g�n�rique d'objets que l'on cherche � �tudier (par exemple,
l'ensemble des pi�ces produites par une cha�ne de montage, un ensemble de
particules en physique, etc.)

Chacun des �l�ments de la population est appel� \textbf{individu.} 

Les caract�risiques que l'on mesure pour chacun de ces individus sont appel�es
les \textbf{variables} ; les individus pour lesquels ces caract�ristiques ont
�t� mesur�es sont appel�es des \textbf{observations}. Un ensemble de $n$
observations $(x_1, x_2, \dots, x_n)$ d'une variable est appel�e \textbf{s�rie
  statistique.}

Par exemple, si j'�tudie les donn�es climatiques pour la station m�t�o de
Paris-Montsouris en 2019 (cf. table~\ref{tab:meteo_data}), il s'agit d'une
population de 365 individus. Cette population peut contenir 8 variables :
temp�ratures minimale, maximale et moyenne ; vitesse maximale du vent ;
ensoleillement ; pr�cipitations totales ; pressions atmosph�riques minimale et
maximale.


Lorsque la population � �tudier est trop grande pour qu'il soit possible
d'observer chacun de ses individus, on �tudie alors une partie seulement de la
population. Cette partie est appel�e \textbf{�chantillon}. On parle alors de
\textbf{sondage}, par opposition � un \textbf{recensement}, qui consiste �
�tudier tous les individus d'une population. Nous parlerons plus en d�tails de
la construction d'un �chantillon dans le chapitre~\ref{chap:estimation}.

Par exemple, la population des �l�ves de premi�re ann�e des Mines est
compos�e de 125 individus. Si je recueille l'�ge, le d�partement de naissance
et le nombre de fr�res et s\oe{}urs de 20 de ces �l�ves, j'aurai mesur� 3 variables
sur un �chantillon de 20 observations. 

On distinguera plusieurs types de variables :
\begin{itemize}
\item les \textbf{variables quantitatives} : des caract�ristiques num�riques
  qui s'expriment naturellement � l'aide de nombres r�els. Ces variables
  peuvent �tre \textbf{discr�tes} si le nombre de valeurs qu'elles peuvent
  prendre est fini ou d�nombrable (ex : �ge, nombre de fr�res et s\oe{}urs) ou
  \textbf{continues} (ex : temp�ratures, taille, pression atmosph�rique)
\item les \textbf{variables qualitatives} : des caract�ristiques qui, bien
  qu'elles puissent �tre encod�es num�riquement (ex : d�partement de
  naissance), rel�vent plut�t de cat�gories et sur lesquelles les op�rations
  arithm�tiques de base (somme, moyenne) n'ont aucun sens. On parle de
  variables \textbf{nominales} s'il n'y a pas d'ordre total sur l'ensemble de
  ces cat�gories (ex : d�partement de naissance) ou \textbf{ordinales} s'il y
  en a (ex : enti�rement d'accord, assez d'accord, pas vraiment d'accord, pas
  du tout d'accord).
\end{itemize}

Remarque : seuiller des variables quantitatives permet de les transformer en
variables qualitatives ordinales. Par exemple, une variable d'�ge peut �tre
transform�e en cat�gories (< 18, 18 -- 20, 20 -- 35, etc.)

Le tableau~\ref{tab:remboursement_data} montre un exemple d'un �chantillon de
20 individus d'une population de donn�es de remboursements d'un acte biologique
bien pr�cis : le dosage de l'antig�ne tumoral 125. Ces donn�es sont issues de
la base de d�penses de biologie m�dicale en France mise � disposition par
l'Assurance
Maladie\footnote{\url{http://open-data-assurance-maladie.ameli.fr/biologie/index.php}}. La
population compl�te, de 604 individus, est disponible dans le fichier\\
\texttt{data/OPEN\_BIO\_2018\_7325.csv}.

Chaque individu de cette population (i.e. ligne du tableau) correspond � un
ensemble de dosages et est d�crit par 5 variables : la tranche d'�ge des
patients et patientes ; leur r�gion ; le nombre de dosages ; et enfin les
montants rembours�s et remboursables. Dans ce tableau, l'�ge est une variable
qualitative ordinale ; la r�gion une variable qualitative ; et les nombres et
montants des variables quantitatives. On pourra choisir de traiter le nombre de
remboursements comme une variable discr�te ou continue.




\section{Types de questions statistiques}

\subsection{Statistique descriptive} 
La \textbf{statistique descriptive,} aussi appel�e \textbf{statistique
  exploratoire,} consiste � caract�riser une population par la d�termination
d'un certain nombre de grandeurs qui la d�crivent. Son objectif est de
synth�tiser l'information contenue dans un ensemble d'observations et de mettre
en �vidence des propri�t�s de cet ensemble. Elle permet aussi de sugg�rer des
hypoth�ses relatives � la population dont sont issues les observations. Il
s'agit principalement de calculer des indicateurs (par exemple des moyennes) et
de visualiser les donn�es par des graphiques. La visualisation peut �tre
enrichie par des techniques de r�duction de dimension (voir
chapitre~\ref{chap:dimred}), qui permettent de cr�er un petit nombre de
variables qui r�sument les mesures prises sur les observations, et des
techniques de partitionnement, ou clustering, qui permettent de r�duire la
taille d'un �chantillon en regroupant les individus pr�sentant des
caract�ristiques homog�nes (ce sujet sera abord� plus en d�tail dans \todo{nom
  du cours ML avanc�}). Nous d�taillons la statistique descriptive dans le
chapitre~\ref{chap:stat_descr}.

\subsection{Statistique inf�rentielle} 
Aussi appel�e \textbf{statistique d�cisionnaire,} ou encore \textbf{inf�rence
  statistique,} la \textbf{statistique inf�rentielle} consiste � tirer des
conclusions sur une population � partir de l'�tude d'un �chantillon de
celle-ci. Les donn�es observ�es sont consid�r�es comme un �chantillon d'une
population. Il s'agit alors d'�tendre des propri�t�s constat�es sur
l'�chantillon � la population. L'inf�rence statistique repose beaucoup sur les
probabilit�s : on consid�rera les observations comme les r�alisations de
variables al�atoires, ce qui permettra d'approcher les caract�ristiques
probabilistes de ces variables al�atoires � l'aide d'indicateurs calcul�s sur
l'�chantillon. Les chapitres~\ref{chap:estimation} et~\ref{chap:tests} tra�tent
de statistique inf�rentielle.


\subsection{L'apprentissage statistique} 
\todo{Introduire le ML + notions d'apprentissage non-supervis� / exploratoire /
  analyse exploratoire de donn�es ; vs apprentissage supervis�.}


\section{D�roul� du cours et �valuation}
Le document que vous �tes en train de lire vit sur GitHub :
\href{https://github.com/chagaz/sdd\_2020}{\texttt{https://github.com/chagaz/sdd\_2020}}.

Dans ce m�me r�pertoire, vous trouverez aussi les dossiers
\begin{itemize}
\item \texttt{notebooks/}, contenant les notebooks jupyter utilis�s pour
  g�n�rer les tables et figures de ce document ;
\item \texttt{data/}, contenant les jeux de donn�es utilis�s dans le cours ou
  pour les TP ;
\item \texttt{pc/}, contenant les sujets des petites classes et du mini-projet
  num�rique.
\end{itemize}

Vous pouvez bien s�r cloner ce r�pertoire et proposer des corrections ou
ajouts via des pull-requests.

\section{Remerciements}
Les contenus de ce poly s'appuient en partie sur des documents mis �
disposition par Laure Reboul, Joseph Salmon, \todo{compl�ter} ainsi que les
ouvrages \textit{Probabilit�s, analyse des donn�es et Statistique} (Technip) de
Gilbert Saporta et \textit{Introduction au Machine Learning} (Dunod InfoSup) de
Chlo�-Agathe Azencott.


%\section{Tableaux de donn�es}
\begin{table}
  \centering
  \begin{tabular}[h]{|c|c|c|c|c|c|c|c|} \hline
    T min & T max & T moy & Vent & Ensoleillement & Pr�cipitations & P min & P max \\
    \textdegree C & \textdegree C & \textdegree C & km/h & min & mm & hPa & hPa \\ \hline
    7.6 & 9.6 & 8.4 & 22.2 & 0 & 0 & 1034 & 1036.6 \\ \hline
    5.6 & 7.2 & 6.3 & 24.1 & 0 & 0 & 1037.3 & 1041.3 \\ \hline
    4.1 & 6.6 & 5.4 & 16.7 & 0 & 0 & 1040.2 & 1041.8 \\ \hline
    3.1 & 6 & 4.7 & 20.4 & 0 & 0 & 1039.5 & 1041.7 \\ \hline
    4.2 & 5.9 & 5 & 20.4 & 0 & 0 & 1037.5 & 1039.6 \\ \hline
    4.3 & 6.8 & 5.6 & 16.7 & 0 & 0 & 1036.5 & 1038 \\ \hline
    6.8 & 8.6 & 7.4 & 20.4 & 0 & 0.6 & 1030.5 & 1037.2\\ \hline
    7.4 & 9.7 & 8.5 & 24.1 & 120 & 0 & 1025.9 & 1029.7 \\ \hline
    4 & 7.7 & 5.2 & 29.6 & 42 & 0.8 & 1024.1 & 1026.4 \\ \hline 
    2.1 & 5.5 & 4 & 18.5 & 30 & 0 & 1026.6 & 1029.5 \\ \hline
    4.2 & 8.3 & 6.2 & 14.8 & 0 & 1.2 & 1028.1 & 1030.5 \\ \hline
    6.7 & 9.1 & 8 & 22.2 & 0 & 0.8 & 1021.7 & 1030.6 \\ \hline
    8.8 & 11.9 & 10.5 & 31.5 & 30 & 0.8 & 1014 & 1021.1 \\ \hline
    8.5 & 10.9 & 8.8 & 29.6 & 0 & 0 & 1014 & 1024.8 \\ \hline
    6.9 & 8.6 & 7.6 & 16.7 & 0 & 0 & 1020.3 & 1025.3 \\ \hline
    1.9 & 7.8 & 5.2 & 27.8 & 276 & 3 & 1007.9 & 1019.6 \\ \hline
    4 & 8.5 & 5.4 & 27.8 & 0 & 0 & 1007.5 & 1019.7 \\ \hline
    0.9 & 6.1 & 2.4 & 18.5 & 342 & 0 & 1017.2 & 1021 \\ \hline
    -1.7 & 2.8 & 0.3 & 14.8 & 78 & 4 & 1009.7 & 1016.8 \\ \hline
    1.9 & 3 & 2.5 & 20.4 & 0 & 0.8 & 1010.1 & 1021.8 \\ \hline
    -2.2 & 3.6 & 0.1 & 13 & 480 & 0 & 1019.2 & 1024.9 \\ \hline
    -2.4 & 1.7 & -0.1 & 20.4 & 0 & 7.4 & 998.4 & 1018.3 \\ \hline
    0.6 & 2.1 & 1.3 & 24.1 & 6 & 1 & 995.6 & 1007.4 \\ \hline
    -0.4 & 2.5 & 1.2 & 20.4 & 0 & 1 & 1008.4 & 1015.5 \\ \hline
    1.7 & 5.5 & 3.9 & 13 & 0 & 1.2 & 1015.2 & 1018.4 \\ \hline
    5.5 & 9.6 & 8.4 & 29.6 & 24 & 1.6 & 998.1 & 1016.9 \\ \hline
    4.4 & 8.2 & 6.4 & 37 & 6 & 4.4 & 989.8 & 1001.4 \\ \hline
    2.7 & 6.9 & 4.4 & 25.9 & 216 & 0 & 1001.7 & 1011.7 \\ \hline
    -0.8 & 5.2 & 1.7 & 24.1 & 18 & 19.6 & 989.5 & 1011.7 \\ \hline
    0.5 & 5.2 & 2.3 & 33.3 & 252 & 0.4 & 990 & 998.9 \\ \hline 
    -1 & 2.5 & 1.3 & 25.9 & 24 & 1.2 & 983.5 & 998 \\ \hline
  \end{tabular}
  \caption{Exemple de 8 variables pour 31 observations (celles du mois de janvier 2019) 
    de la population des donn�es climatiques pour la station m�t�o de Paris-Montsouris. Ces donn�es sont disponibles dans le fichier \texttt{data/meteo\_data.csv}.}
  \label{tab:meteo_data}
\end{table}

\begin{table}
  \centering
  \begin{tabular}{|r|r|r|r|r|r|} \hline
    \multirow{2}{*}{�ge} & \multirow{2}{*}{R�gion} & Nombre & Montants & Montants  \\ 
    & & d'actes & rembours�s & remboursables \\ 
    ans & & & (\texteuro) & (\texteuro) \\\hline 
    $>$ 60 & 76 & 26 & 377,96 & 402,80\\  \hline
    $>$ 60 & 75 & 1\,401 & 14\,054,37 & 21\,332,15\\  \hline
    $>$ 60 & 44 & 5\,299 & 65\,928,93 & 80\,447,00\\  \hline
    $>$ 60 & 32 & 1\,706 & 25\,137,65 & 26\,032,65\\  \hline
    $>$ 60 & 32 & 2\,596 & 37\,877,02 & 39\,336,15\\  \hline
    $>$ 60 & 27 & 14 & 159,85 & 211,35\\  \hline
    $>$ 60 & 24 & 3\,565 & 50\,770,46 & 54\,076,15\\  \hline
    $>$ 60 & 11 & 396 & 5\,226,55 & 6\,060,05\\  \hline
    $>$ 60 & 5 & 260 & 4\,496,91 & 4\,676,40\\  \hline
    $>$ 60 & 93 & 162 & 2\,303,56 & 2\,466,10\\  \hline
    $>$ 60 & 76 & 578 & 8\,499,53 & 8\,793,10\\  \hline
    40-59 & 76 & 13 & 172,26 & 199,80\\  \hline
    40-59 & 44 & 102 & 1\,204,93 & 1\,557,20\\  \hline
    40-59 & 11 & 48 & 555,39 & 733,05\\  \hline
    40-59 & 84 & 14 & 190,21 & 217,85\\  \hline
    40-59 & 32 & 126 & 1\,350,06 & 1\,912,15\\  \hline
    20--39 & 32 & 749 & 7\,941,69 & 11\,362,40\\  \hline
    20--39 & 32 & 24 & 289,35 & 365,25\\  \hline
    20--39 & 5 & 918 & 9\,704,10 & 16\,550,40\\  \hline
    20--39 & 11 & 106 & 1\,073,32 & 1\,618,35\\  \hline
  \end{tabular}
  \caption{Population de remboursements du dosage de l'antig�ne 125 dans le sang en 2018, 
    compos�e de 20 individus d�crits par 5 variables et extraite du fichier \texttt{data/OPEN\_BIO\_2018\_7325.csv}. \\
    R�gion : 5 = R�gions et D�partements d'outre-mer. 
    11 = Ile-de-France. 
    24 = Centre-Val de Loire.
    27 = Bourgogne-Franche-Comt�.
    32 = Hauts-de-France.
    44 = Grand-Est.
    75 = Nouvelle-Aquitaine.
    76 = Occitanie.
    84 = Auvergne-Rh�ne-Alpes.
    93 = Provence-Alpes-C�te d'Azur et Corse.}
  \label{tab:remboursement_data}
\end{table}



%%% Local Variables:
%%% mode: latex
%%% TeX-master: "sdd_2020_poly"
%%% End:

\part{Estimation statistique}
\chapter{Notions de statistique}
% \documentclass[a4paper,12pt]{article}

% \usepackage[T1]{fontenc}
% \usepackage{gentium}

% \usepackage[colorlinks=true,urlcolor=Aubergine,citecolor=MyDarkGrey]{hyperref}

% \usepackage[margin=2cm]{geometry}

% \usepackage{amsmath}
% \usepackage{amsfonts}
% \usepackage{amssymb}
% \usepackage{amsthm}
% \usepackage{bm}
% \usepackage{graphicx}
% \usepackage{multirow}
% \usepackage{parskip} % adjust skip between paragraphs
% \usepackage{textcomp}
% \usepackage{xcolor}

% \usepackage[frenchb]{babel} % pour les caract�res fran�ais
% \usepackage[utf8]{inputenc} % pour les caract�res fran�ais
% \usepackage[T1]{fontenc} % meilleur rendu de certains caract�res sp�ciaux 

% \usepackage{enumitem}
% \setlist[itemize]{label=\textendash}

% \begin{document}


%-*- coding: iso-latin-1 -*-
\label{chap:stat}
\paragraph{Notions :} population, �chantillon, estimation, biais, convergence
d'un estimateur, variance, covariance, compromis biais-variance.

\paragraph{Objectifs p�dagogiques :}
\begin{itemize}
\item D�finir, �valuer et caract�riser un estimateur statistique.
\end{itemize}

\section{Qu'est-ce que la statistique ?}

Le terme de � statistique � est d�riv� du latin � \textit{status} � (signifiant
� �tat �). Historiquement, \textbf{les statistiques} concernent l'�tude
m�thodique, par des proc�d�s num�riques (inventaires, recensements, etc.) des
faits sociaux qui d�finissent un �tat. Elles sont d�sormais utilis�es dans tous
les secteurs o� l'on dispose de donn�es : sciences sociales mais aussi sant�,
environnement, industrie, �conomie, recherche scientifique, etc.

Par contraste, \textbf{la statistique} est un ensemble de m�thodes des
math�matiques appliqu�es permettant de d�crire et d'analyser des ph�nom�nes
dont la nature rend une �tude exhaustive de tous leurs facteurs impossible. Ces
m�thodes permettent d'�tudier des donn�es, ou observations, consistant en la
mesure d'une ou plusieurs caract�ristiques d'un ensemble de personnes ou objets
�quivalents.

\subsection{Vocabulaire}

L'ensemble de personnes ou d'objets �quivalents �tudi�s est appel� \textbf{la
  population.} Il peut s'agir d'une population au sens � courant � du terme
(par exemple, l'ensemble de la population fran�aise, ou l'ensemble des
individus d'une esp�ce animale sur un territoire) mais aussi plus largement
d'un ensemble plus g�n�rique d'objets que l'on cherche � �tudier (par exemple,
l'ensemble des pi�ces produites par une cha�ne de montage, un ensemble de
particules en physique, etc.)

Chacun des �l�ments de la population est appel� \textbf{individu.} 

Les caract�risiques que l'on mesure pour chacun de ces individus sont appel�es
les \textbf{variables} ; les individus pour lesquels ces caract�ristiques ont
�t� mesur�es sont apel�es des \textbf{observations}.

Par exemple, si j'�tudie les donn�es climatiques pour la station m�t�o de
Paris-Montsouris en 2019 (cf. table~\ref{tab:meteo_data}), il s'agit d'une
population de 365 individus. Cette population peut contenir 8 variables :
temp�ratures minimale, maximale et moyenne ; vitesse maximale du vent ;
ensoleillement ; pr�cipitations totales ; pressions atmosph�riques minimale et
maximale.

\begin{table}[h]
  \centering
  \begin{tabular}[h]{|c|c|c|c|c|c|c|c|} \hline
    T min & T max & T moy & Vent & Ensoleillement & Pr�cipitations & P min & P max \\
    \textdegree C & \textdegree C & \textdegree C & km/h & min & mm & hPa & hPa \\ \hline
    7.6 & 9.6 & 8.4 & 22.2 & 0 & 0 & 1034 & 1036.6 \\ \hline
    5.6 & 7.2 & 6.3 & 24.1 & 0 & 0 & 1037.3 & 1041.3 \\ \hline
    4.1 & 6.6 & 5.4 & 16.7 & 0 & 0 & 1040.2 & 1041.8 \\ \hline
    3.1 & 6 & 4.7 & 20.4 & 0 & 0 & 1039.5 & 1041.7 \\ \hline
    4.2 & 5.9 & 5 & 20.4 & 0 & 0 & 1037.5 & 1039.6 \\ \hline
    4.3 & 6.8 & 5.6 & 16.7 & 0 & 0 & 1036.5 & 1038 \\ \hline
    6.8 & 8.6 & 7.4 & 20.4 & 0 & 0.6 & 1030.5 & 1037.2\\ \hline
    7.4 & 9.7 & 8.5 & 24.1 & 120 & 0 & 1025.9 & 1029.7 \\ \hline
    4 & 7.7 & 5.2 & 29.6 & 42 & 0.8 & 1024.1 & 1026.4 \\ \hline 
    2.1 & 5.5 & 4 & 18.5 & 30 & 0 & 1026.6 & 1029.5 \\ \hline
    4.2 & 8.3 & 6.2 & 14.8 & 0 & 1.2 & 1028.1 & 1030.5 \\ \hline
    6.7 & 9.1 & 8 & 22.2 & 0 & 0.8 & 1021.7 & 1030.6 \\ \hline
    8.8 & 11.9 & 10.5 & 31.5 & 30 & 0.8 & 1014 & 1021.1 \\ \hline
    8.5 & 10.9 & 8.8 & 29.6 & 0 & 0 & 1014 & 1024.8 \\ \hline
    6.9 & 8.6 & 7.6 & 16.7 & 0 & 0 & 1020.3 & 1025.3 \\ \hline
    1.9 & 7.8 & 5.2 & 27.8 & 276 & 3 & 1007.9 & 1019.6 \\ \hline
    4 & 8.5 & 5.4 & 27.8 & 0 & 0 & 1007.5 & 1019.7 \\ \hline
    0.9 & 6.1 & 2.4 & 18.5 & 342 & 0 & 1017.2 & 1021 \\ \hline
    -1.7 & 2.8 & 0.3 & 14.8 & 78 & 4 & 1009.7 & 1016.8 \\ \hline
    1.9 & 3 & 2.5 & 20.4 & 0 & 0.8 & 1010.1 & 1021.8 \\ \hline
    -2.2 & 3.6 & 0.1 & 13 & 480 & 0 & 1019.2 & 1024.9 \\ \hline
    -2.4 & 1.7 & -0.1 & 20.4 & 0 & 7.4 & 998.4 & 1018.3 \\ \hline
    0.6 & 2.1 & 1.3 & 24.1 & 6 & 1 & 995.6 & 1007.4 \\ \hline
    -0.4 & 2.5 & 1.2 & 20.4 & 0 & 1 & 1008.4 & 1015.5 \\ \hline
    1.7 & 5.5 & 3.9 & 13 & 0 & 1.2 & 1015.2 & 1018.4 \\ \hline
    5.5 & 9.6 & 8.4 & 29.6 & 24 & 1.6 & 998.1 & 1016.9 \\ \hline
    4.4 & 8.2 & 6.4 & 37 & 6 & 4.4 & 989.8 & 1001.4 \\ \hline
    2.7 & 6.9 & 4.4 & 25.9 & 216 & 0 & 1001.7 & 1011.7 \\ \hline
    -0.8 & 5.2 & 1.7 & 24.1 & 18 & 19.6 & 989.5 & 1011.7 \\ \hline
    0.5 & 5.2 & 2.3 & 33.3 & 252 & 0.4 & 990 & 998.9 \\ \hline 
    -1 & 2.5 & 1.3 & 25.9 & 24 & 1.2 & 983.5 & 998 \\ \hline
  \end{tabular}
  \caption{Exemple de 8 variables pour 31 observations (celles du mois de janvier) de la population des donn�es climatiques pour la station m�t� de Paris-Montsouris en 2019.}
  \label{tab:meteo_data}
\end{table}

Lorsque la population � �tudier est trop grande pour qu'il soit possible
d'observer chacun de ses individus, on �tudie alors une partie seulement de la
population. Cette partie est appel�e \textbf{�chantillon}. On parle alors de
\textbf{sondage}, par opposition � un \textbf{recensement}, qui consiste �
�tudier tous les individus d'une population. Nous parlerons plus en d�tails de
la construction d'un �chantillon dans la section~\ref{ref:echantilonnage}.

Par exemple, la population des �l�ves de premi�re ann�e des Mines est
compos�e de 125 individus. Si je recueille l'�ge, le d�partement de naissance
et le nombre de fr�res et s\oe{}urs de 20 de ces �l�ves, j'aurai mesur� 3 variables
sur un �chantillon de 20 observations. 

On distinguera plusieurs types de variables :
\begin{itemize}
\item les \textbf{variables quantitatives} : des caract�ristiques num�riques
  qui s'expriment naturellement � l'aide de nombres r�els. Ces variables
  peuvent �tre \textbf{discr�tes} si le nombre de valeurs qu'elles peuvent
  prendre est fini ou d�nombrable (ex : �ge, nombre de fr�res et s\oe{}urs) ou
  \textbf{continues} (ex : temp�ratures, taille, pression atmosph�rique)
\item les \textbf{variables qualitatives} : des caract�ristiques qui, bien
  qu'elles puissent �tre encod�es num�riquement (ex : d�partement de
  naissance), rel�vent plut�t de cat�gories et sur lesquelles les op�rations
  arithm�tiques de base (somme, moyenne) n'ont aucun sens. On parle de
  variables \textbf{nominales} s'il n'y a pas d'ordre total sur l'ensemble de
  ces cat�gories (ex : d�partement de naissance) ou \textbf{ordinales} s'il y
  en a (ex : enti�rement d'accord, assez d'accord, pas vraiment d'accord, pas
  du tout d'accord).
\end{itemize}

Remarque : seuiller des variables quantitatives permet de les transformer en
variables qualitatives ordinales. Par exemple, une variable d'�ge peut �tre
transform�e en cat�gories (< 18, 18 -- 20, 20 -- 35, etc.)

\subsection{Types de questions statistiques}

\paragraph{La statistique descriptive:} Aussi appel�e \textbf{statistique
  exploratoire,} elle consiste � caract�riser une population par la
d�termination d'un certain nombre de grandeurs qui la d�crivent. Son objectif
est de synth�tiser l'information contenue dans un ensemble d'observations et de
mettre en �vidence des propri�t�s de cet ensemble. Elle permet aussi de
sugg�rer des hypoth�ses relatives � la population dont sont issues les
observations. Il s'agit principalement de calculer des indicateurs (par exemple
des moyennes) et de visualiser les donn�es par des graphiques. La visualisation
peut �tre enrichie par des techniques de r�duction de dimension (voir chapitre
TODO REF), qui permettent de cr�er un petit nombre de variables qui r�sument
les mesures prises sur les observations, et des techniques de partitionnement,
ou clustering, qui permettent de r�duire la taille d'un �chantillon en
regroupant les individus pr�sentant des caract�ristiques homog�nes (ce sujet
sera abord� plus en d�tail dans TODO nom cours ML avanc�. Nous d�taillons la
statistique descriptive dans la section~\ref{sec:stat_descr}.

\paragraph{La statistique inf�rentielle:} Aussi appell�e \textbf{statistique
  d�cisionnaire,} ou encore \textbf{inf�rence statistique,} elle consiste �
tirer des conclusions sur une population � partir de l'�tude d'un �chantillon
de celle-ci. Les donn�es observ�es sont consid�r�es comme un �chantillon d'une
population. Il s'agit alors d'�tendre des propri�t�s constat�es sur
l'�chantillon � la population. L'inf�rence statistique repose beaucoup sur les
probabilit�s : on consid�rera les observations comme les r�alisations de
variables al�atoires, ce qui permettra d'approcher les caract�ristiques
probabilistes de ces variables al�atoires � l'aide d'indicateurs calcul�es sur
l'�chantillon.  Nous d�taillons la statistique inf�rentielle dans la
section~\ref{sec:inference} ainsi qu'aux chapitres TODO et TODO.

\paragraph{L'apprentissage statistique} TODO

\section{Statistique descriptive}
\label{sec:stat_descr}
Le r�le de la statistique descriptive est de caract�riser une population par la
d�termination d'un certain nombre de grandeurs qui la d�crivent.

\begin{itemize}
\item cf cours Laure Reboul.
\item Utiliser http://iml.univ-mrs.fr/~reboul/enseignement.html pour proposer des exercices ?
\end{itemize}

\section{Inf�rence statistique}
L'inf�rence statistique a pour but de tirer des conclusions sur une population
� partir de l'�tude d'un �chantillon de celle-ci. Pour cela, il est n�cessaire
de s'int�resser :
\begin{itemize}
\item aux techniques d'\textbf{�chantillonnage} (cf section~\ref{sec:echantilonnage})
  permettant de construire des �chantillons d'une population ;
\item � la \textbf{mod�lisation} permettant de supposer un mod�le probabiliste
  sur la population ;
\item aux techniques d'\textbf{estimation} (cf section~\ref{sec:estimation})
  permettant de d�terminer (approximativement) un param�tre d'une population �
  partir d'un �chantillon de celle-ci ;
\item aux \textbf{tests d'hypoth�se} (cf chapitre TODO) permettant de valider
  ou d'infirmer des hypoth�ses sur la population.
\end{itemize}

\subsection{�chantillonnage}
\label{ref:echantilonnage}

\subsection{Estimation ponctuelle}
\begin{itemize}
\item cf cours Joseph
\item cf ProbaAgreg p5-13 onwards
\end{itemize}

\subsection{Estimation par intervalle}
\begin{itemize}
\item intervalle de confiance
\item cf ProbaAgreg p13 onwards
\end{itemize}
% \end{document}

\chapter{Tests d'hypoth�se}
%%-*- coding: iso-latin-1 -*-
\label{chap:tests}

\paragraph{Notions :} hypoth�se nulle, hypoth�se alternative, statistique de
test, p-valeur, tests multiples.
\paragraph{Objectifs p�dagogiques :}
\begin{itemize}      
  \setlength{\itemsep}{3pt}
\item Interpr�ter une p-valeur.
\item Reconna�tre une situation dans laquelle un test statistique est
  appropri�.
\end{itemize}


\section{Principe d'un test statistique}
\label{sec:principe_test}
Le but d'un test statistique est de d�terminer la fiabilit� d'une observation
faite sur une �chantillon.

\begin{exemple}
  Si je lance une pi�ce 5 fois et obtiens 5 fois pile, puis-je en d�duire que
  la pi�ce est d�s�quilibr�e ? Ou ce r�sultat est-il d� au hasard de
  l'�chantillonnage ? Qu'en est-il si j'obtiens le m�me r�sultat apr�s 50
  lancers ?
\end{exemple}

\textbf{Un test statistique permet de d�terminer si l'�chantillon observ�
  permet d'invalider une hypoth�se qu'il �tait raisonnable de formuler avant
  d'observer les donn�es.}

\begin{exemple}
  Reprenons l'exemple du lancer de pi�ce. Sous l'hypoth�se que la pi�ce est
  �quilibr�e, la probabilit� $\pi$ d'obtenir � pile � pour un lancer est $0.5$
  et celle d'obtenir pile pour 5 lancers est $0.5^5 = 3\%.$ Cette probabilit�
  est faible, mais non n�gligeable : on a 3\% de chance d'obtenir un r�sultat
  aussi extr�me que celui observ� sur un �chantillon.

  Pour 50 lancers, cette probabilit� tombe � $0.5^{50} = 9.10^{-16}.$ Cette
  probabilit� est extr�mement faible, et l'�chantillon ne soutient pas
  l'hypoth�se selon laquelle la pi�ce est �quilibr�e : nous pouvons la rejeter.
\end{exemple}

\section{Formalisme}
\label{sec:formalisme_test}
Soit $(X_1, X_2, \dots, X_n)$ un �chantillon al�atoire de taille $n \in \NN^*$
d'une variable al�atoire r�elle $X$ de loi $\PP_X.$ Rappelons que les
composantes $X_i$ de ce vecteur al�atoire sont ind�pendantes et identiquement
distribu�es, de m�me loi que $X$. Les notions pr�sent�es dans ce chapitre
s'appliquent aussi � des variables al�atoires de nature plus complexe (par
exemple, des valeurs al�atoires multi-dimensionnelles) mais nous nous limitons
aux variables al�atoires r�elles par souci de simplicit�.
Nous supposons aussi disposer d'un �chantillon $(x_1, x_2, \dots, x_n)$ qui est
une r�alisation de $(X_1, X_2, \dots, X_n)$.

Un test statistique repose sur les �l�ments suivants :
\begin{itemize}
\item Une \textbf{hypoth�se nulle,} not�e $\HH_0$. L'hypoth�se nulle est
  celle que l'on chercher � rejeter.
\item Une \textbf{hypoth�se alternative,} not�e $\HH_1$ ou $\HH_a$. C'est en
  g�n�ral la n�gation de $\HH_0$.
\item Une \textbf{statistique de test,} $T$, qui sert � mesurer � quel point un
  �chantillon � d�vie � de l'hypoth�se nulle.
\item Un \textbf{niveau de signification,} $0 < \alpha < 1$, qui est la
  probabilit� de rejeter l'hypoth�se nulle alors qu'elle est correcte. 
% qui sert �
  % d�terminer si la probabilit� d'observer, sous $\HH_0$, une statistique de
  % test au moins aussi extr�me que celle observ�e sur l'�chantillon
  % $(x_1, x_2, \dots, x_n)$ est suffisamment faible pour rejeter $\HH_0$.
\end{itemize}

Le but de cette section est de d�velopper ces notions.

\subsection{Hypoth�ses de test}
Conduire un test d'hypoth�se n�cessite de formuler deux hypoth�ses :
\begin{itemize}
\item Une \textbf{hypoth�se nulle,} not�e $\HH_0$. Cette hypoth�se doit �tre
  pr�cise et permettre de faire des calculs. Le but du test est de d�terminer
  s'il est raisonnable de rejeter cette hypoth�se.
\item Une \textbf{hypoth�se alternative,} not�e $\HH_1$ ou $\HH_a$. Cette
  hypoth�se est une forme de n�gation de $\HH_0$, et c'est l'hypoth�se que l'on
  adoptera si l'hypoth�se nulle est rejet�e.
\end{itemize}

L'hypoth�se nulle est souvent une hypoth�se formul�e sur la valeur un param�tre
$\theta \in \Scal \subseteq \RR$ caract�risant la loi $\PP_X$ de l'�chantillon
al�atoire. Il s'agit alors de tester
\begin{equation}
  \label{eq:h0}
  \HH_0: \theta = \theta_0,
\end{equation}
o� $\theta_0 \in \Scal$ est une valeur d�terministe fix�e � l'avance.

L'hypoth�se nulle peut cependant �tre de nature plus complexe, par exemple :
\begin{itemize}
\item � Deux variables statistique $X$ et $Y$ sont ind�pendantes � (c'est le
  cas du test d'ind�pendance du $\chi^2$, cf. section~\ref{sec:chi2}).
\item � Deux �chantillons $(x_1, x_2, \dots, x_n)$ et $(y_1, y_2, \dots, y_n)$
  sont des r�alisations de la m�me distribution � (c'est le cas du test de
  Wilcoxon-Mann-Whitney, qui d�passe le cadre de ce programme) ;
\end{itemize}

\paragraph{Pr�somption d'innocence} De m�me que le principe de la pr�somption
d'innocence veut que l'on recueille suffisamment de preuves pour rejeter
l'innocence, en th�orie des tests statistiques il y a pr�somption de
$\HH_0$. Il s'agit donc de savoir si l'�chantillon observ� (les preuves) est
suffisant pour rejeter $\HH_0$, ce dont on conclura $\HH_1.$ Par contre, si
l'on ne rejette pas $\HH_0$, cela peut venir soit de ce que $\HH_0$ est vraie,
soit de ce que nous n'avons pas suffisamment de donn�es pour rejeter
$\HH_1$. Ainsi, $\HH_0$ doit �tre une hypoth�se raisonnable, mais que l'on
aimerait avoit des raisons de r�futer.  

Dans le cadre d'une exp�rience scientifique, l'hypoth�se $\HH_0$ correspond
ainsi � l'�tat actuel des connaissances. Le but d'un test statistique est de
d�terminer si les donn�es qui semblent contredire cette hypoth�se sont
effectivement suffisamment improbables sous $\HH_0$ pour justifier de la
r�futer.
Dans le cadre d'un essai clinique, par exemple, l'hypoth�se $\HH_0$ se doit
d'�tre une d�favorable au nouveau m�dicament (� le nouveau m�dicament est
inefficace � ou � le nouvea m�dicament n'est pas plus efficace que les
traitements connus �). Le but du test statistique est de d�terminer si les
donn�es r�colt�es jusqu'� pr�sent sont suffisantes pour r�futer cette
hypoth�se.

\begin{exemple}
  Dans le cas de notre lancer de pi�ce,
  \begin{itemize}
  \item $X$ est une variable al�atoire discr�te qui suit une loi de Bernoulli
    de param�tre $\pi$. $\PP_X(1) = \pi$ et $\PP_X(0) = 1-\pi$ ;
  \item l'�chantillon al�atoire est un vecteur $(X_1, X_2, \dots, X_n)$ de $n$
    composantes, iid de m�me loi que $X$ ;
  \item une s�rie de lancers est une r�alisation $(x_1, x_2, \dots, x_n)$ de ce
    vecteur al�atoire. Dans le cas de 5 lancers tous tombant sur � pile �,
    cet �chantillon est $(1, 1, 1, 1, 1)$ et $n=5.$
  \item l'hypoth�se nulle est $\HH_0: \pi = 0.5.$
  \end{itemize}
\end{exemple}

Dans le cas o� l'on cherche � tester la valeur d'un param�tre $\theta$ d'une
population, l'hypoth�se alternative peut prendre deux formes :
\begin{itemize}
\item $\theta \neq \theta_0$, ou en d'autres termes, 
  \begin{equation}
    \label{eq:h1_bilateral}
    \HH_1: \theta < \theta_0 \text{ ou } \theta > \theta_0.
  \end{equation}
  On parle alors de test \textbf{bilat�ral.}
\item Si seulement l'une des deux parties de cette hypoth�se alternative nous
  int�resse, ou est possible, on parle de test \textbf{unilat�ral.} Il s'agit
  alors de tester soit
  \begin{equation}
    \label{eq:h1_unilateral_gauche}
    \HH_1: \theta < \theta_0,
  \end{equation}
  soit
  \begin{equation}
    \label{eq:h1_unilateral_droite}
    \HH_1:  \theta > \theta_0.
  \end{equation}
\end{itemize}

De m�me que l'on �labore $\HH_0$ de sorte � ce qu'elle soit la plus plausible
avant d'avoir observ� les donn�es, on �labore $\HH_1$ en fonction de ce que
l'on esp�re d�couvrir. 

Reprenons l'exemple d'un essai clinique sur un nouveau m�dicament. Si
l'hypoth�se $\HH_0$ est � le nouveau m�dicament n'a pas d'effet �, on peut
poser l'hypoth�se alternative $\HH_1$ : � le nouveau traitement a un effet
positif sur l'�tat des patients �. On esp�re ici non seulement rejeter
l'hypoth�se nulle, mais aussi sugg�rer une efficacit� du traitement. Cette
hypoth�se est plus pr�cise que l'hypoth�se alternative selon laquelle � le
nouveau traitement a un effet sur l'�tat des patients �, cet effet pouvant �tre
n�gatif.

\begin{exemple}
  Dans le cas de notre lancer de pi�ce, l'hypoth�se alternative dans le cadre
  d'un test bilat�ral est
  \[
    \HH_1: \pi \neq 0.5.
  \]
  Si nous rejetons $\HH_0,$ notre conclusion sera que la pi�ce n'est pas �quilibr�e.

  Dans le cadre d'un test bilat�ral, par exemple 
  \[
    \HH_1: \pi > 0.5,
  \]
  si nous rejetons $\HH_0,$ notre conclusion sera que la pi�ce n'est pas
  �quilibr�e, et qu'elle favorise � pile �.

  Il ne s'agit donc pas du m�me test.
\end{exemple}


\subsection{Statistique de test et p-valeur}
Une \textbf{statistique de test} $T$ est une statistique de l'�chantillon
al�atoire. Il s'agit donc d'une variable al�atoire r�elle, fonction de
$(X_1, X_2, \dots, X_n) : T = g(X_1, X_2, \dots, X_n)$ Cette statistique de
test sert � mesurer � quel point un �chantillon � d�vie � de l'hypoth�se nulle.

Une statistique de test est ainsi choisie de sorte � avoir une loi diff�rente
sous $\HH_0$ et sous $\HH_1$, et de sorte � ce que sa loi sous $\HH_0$ soit
connue : c'est ce qui permettra de d�terminer un crit�re de rejet de $\HH_0$
garantissant le niveau de signification choisi.

La plupart des test statistiques reposent sur des statistique de test dont le
d�veloppement a �t� long et minutieux. Le choix entre plusieurs statistiques
candidates pour un m�me probl�me est un choix difficile, qui repose entre
autres sur la validit� des hypoth�ses sur la distribution de l'�chantillon
al�atoire ou sur sa taille qui permettent de d�terminer sa loi sous $\HH_0$.

\paragraph{Remarque} Pour des tests portant sur un param�tre
($\HH_0: \theta = \theta_0$), la statistique de test est souvent bas�e sur la
diff�rence entre un estimateur de ce param�tre et sa valeur sous $\HH_0$.

\begin{exemple}
  Reprenons l'exemple du lancer de pi�ce.

  Dans la section~\ref{sec:principe_test}, nous avons choisi comme statistique
  de test $T$ le nombre de pile obtenus dans l'�chantillon :
  \[
    T = \sum_{i=1}^n X_i.
  \]

  Sous $\HH_0$, autrement dit si $\pi=0.5$, la loi de $T$ est d�termin�e par 
  \[
    \PP(T=k) = \PP\left(\sum_{i=1}^n X_i = k\right) \text{ pour } k=0, 1,
    \dots, n.
  \]
  On reconnait ici une loi bin�miale de param�tres $n$ et $\pi.$
\end{exemple}


\subsection{Niveau de signification}
Nous avons maintenant pos� $\HH_0$, $\HH_1$, et une statistique de test $T$
dont nous connaissons la loi $\PP_{T0}$ sous $\HH_0$. Il nous faut maintenant
d�terminer la \textbf{zone de rejet} du test, autrement dit l'ensemble
$\Ical \subseteq \RR$ de ses valeurs qui conduisent � rejeter $\HH_0$.

Pour ce faire, nous avons besoin de fixer le \textbf{niveau de signification,}
$0 < \alpha < 1$, qui est la probabilit� de rejeter l'hypoth�se nulle alors
qu'elle est correcte. Ce seuil est fix� � l'avance, g�n�ralement parmi
$\alpha = 1\%$, $\alpha = 5\%$ ou $\alpha = 10\%$, et d�termine � quel point le
test est strict.

Ainsi, il s'agit de d�terminer $\Ical \subseteq \RR$ de sorte � ce que
$\PP_{T0}(T \in \Ical) = \alpha.$

\begin{exemple}
  Dans l'exemple du lancer de pi�ce, nous avons choisi le nombre de pile comme
  statistique de test $T$. Sous $\HH_0 : \pi = 0.5$, $T$ suit une loi binomiale
  de param�tres $n$ (le nombre de lancers) et $\pi$.

  Posons $\alpha = 5\%.$

  Consid�rons le test unilat�ral $\HH_1 : \pi > 0.5$. Si nous rejetons $\HH_0$,
  nous en conclurons que la pi�ce est biais�e en faveur du c�t� pile. Cela
  signifie que nous souhaitons rejeter $\HH_0$ quand le nombre de pile dans
  l'�chantillon est grand. Il est ici naturel de consid�rer une zone de rejet
  de la forme $\Ical = \mathopen]t_0, n\mathclose].$ En d'autres termes, nous allons rejeter
  $\HH_0$ si la r�alisation $t$ de $T$ sur notre �chantillon est plus grande
  qu'un seuil $t_0,$ fix� tel que $\PP_{T0}(T > t_0) = \alpha.$ 

  En d'autres termes, si $F_{T0}$ est la fonction de r�partition de $T$ sous
  $\HH_0$, $t_0$ est fix� de sorte � ce que $F_{T0}(t_0) = \alpha$. Dans notre
  exemple avec $n=5$ et $\alpha=0.05$, cela fixe $t_0 = 4.$

  Le test consiste donc � rejeter l'hypoth�se nulle si tous les 5 lancers
  aboutissent � pile.

  Consid�rons maintenant le test unilat�ral $\HH_1 : \pi < 0.5.$ Rejeter
  $\HH_0$ conduit � conclure que la pi�ce est biais�e en faveur du c�t�
  face. Nous consid�rons maintenant une zone de rejet de la forme
  $\Ical = \mathopen[0, t_0 \mathclose[,$ et $t_0$ est d�termin� par
  $\PP_{T0}(T < t_0) = \alpha.$ Avec $n=5$ et $\alpha=0.05$, cela fixe
  $t_0=1$. Le test consiste donc � rejeter l'hypoth�se nulle si aucun des 5
  lancers n'aboutit � pile.

  Enfin, consid�rons le test bilat�ral $\HH_1 : \pi \neq 0.5.$ Rejeter $\HH_0$
  conduit � conclure que la pi�ce est biais�e, en faveur de l'un ou de l'autre
  de ses c�t�s. Nous consid�rons alors une zone de rejet de la forme
  $\Ical = \mathopen[0, t_l \mathclose[ \; \cup \; \mathopen]t_r, n
  \mathclose].$
  Il nous faut donc choisir $t_l$ et $t_r$ de sorte � ce que
  $\PP_{T0}(T < t_l) + \PP_{T0}(T > t_r) = \alpha.$ Il est assez naturel de
  fixer alors $\PP_{T0}(T < t_l) = \PP_{T0}(T > t_r) = \frac{\alpha}{2}.$ Avec
  $n=5$ et $\alpha=0.05$, on obtient $t_l = 0$ et $t_r = 5$ et il n'est donc
  jamais possible de rejeter l'hypoth�se nulle.

  Le test que nous venons de d�finir s'appelle le test binomial. 

  \paragraph{Remarque importante} On observe ici que, parmi les trois
  hypoth�ses alternatives envisag�es, seul le test statistique unilat�ral
  $\HH_1: \pi > 0.5$ nous permet de rejeter l'hypoth�se nulle. C'est une
  observation g�n�rale : un test unilat�ral est plus puissant qu'un test
  bilat�ral ; cependant il n'est utile que si on sait de quel c�t� le d�finir.

\end{exemple}

Dans le cas d'un test sur la valeur d'un param�tre $\theta$, c'est-�-dire avec
pour hypoth�se nulle $\HH_0: \theta = \theta_0$, la zone de rejet sera de la
forme
\begin{itemize}
\item $\Ical = \mathopen]t_r, +\infty \mathclose[$ pour le test unilat�ral � droite
  $\HH_1 : \theta > \theta_0$ ;
\item $\Ical = \mathopen]-\infty, t_l \mathclose[$ pour le test unilat�ral � gauche
  $\HH_1 : \theta > \theta_0$ ;
\item $\Ical = \mathopen]-\infty, t_l \mathclose[ \cup
  \mathopen]t_r, +\infty \mathclose[$ pour le test bilat�ral
  $\HH_1 : \theta \neq \theta_0$. 
\end{itemize}

On cherchera souvent � utiliser une statistique de test symm�trique, de sorte �
pouvoir utiliser $t_r = - t_l$.  Dans ce cas $t_0 = t_t$ est appel�e
\textbf{valeur critique} du test et est telle que
\begin{itemize}
\item $\PP_{T0}(T > t_0) = \alpha$ pour le test unilat�ral � droite ; 
\item $\PP_{T0}(T < - t_0) = \alpha$ pour le test unilat�ral � gauche ; 
\item $\PP_{T0}(|T| > t_0) = \alpha$ pour le test bilat�ral. 
\end{itemize}
Dans ce contexte, �tant donn� un �chantillon $(x_1, x_2, \dots, x_n)$ et la
r�alisation $t$ de $T$ sur cet �chantillon, on appelle \textbf{p-valeur} la
probabilit� $\PP_{T0}(T > t)$ pour un test unilat�ral � droite (respectivement,
$\PP_{T0}(T < -t)$ pour un test unilat�ral � gauche, et $\PP_{T0}(\abs{T} > t)$
pour un test bilat�ral). L'hypoth�se nulle est rejet�e si la p-valeur est plus
petite que le niveau de signification. 

En d'autres termes, la p-valeur peut �tre interpr�t�e comme la probabilit�
d'obtenir, sous l'hypoth�se nulle, un r�sultat au moins aussi extr�me que celui
observ�.

On rapporte ainsi g�n�ralement comme r�sultat d'un test non pas la statistique
de test r�alis�e sur l'�chantillon observ�, mais la p-valeur correspondante.

On lira ainsi dans des publications scientifiques des assertions suivies de �
($p < 0.05$) �, siginifiant que l'assertion en question est l'hypoth�se
alternative d'un test dont l'hypoth�se nulle a �t� rejet�e avec une p-valeur
inf�rieure � $5\%$

\begin{exemple}
  Le test que nous avons d�fini dans l'exemple de la pi�ce de monnaie s'appelle
  le test binomial. Il est impl�ment� dans \texttt{scipy.stats} :
  \begin{lstlisting}[language=Python]
    t = 5 # nb pile 
    n = 5 # taille �chantillons 
    pi = 0.5 
    import scipy.stats as st 
    st.binom_test(t, n, pi, alternative='greater') # unilat�ral � droite
  \end{lstlisting}
\end{exemple}

\subsection{Erreurs de premi�re et deuxi�me esp�ce}
Deux types d'erreurs sont possibles quand on fait un test d'hypoth�se :
\begin{itemize}
\item Rejeter l'hypoth�se nulle alors qu'elle est correcte : on parle d'une
  \textbf{erreur de premi�re esp�ce}, ou \textbf{erreur de Type I}.
\item Accepter l'hypoth�se nulle alors qu'elle est en fait fausse : on parle
  d'une \textbf{erreur de deuxi�me esp�ce}, ou \textbf{erreur de Type II}.
\end{itemize}

\paragraph{Moyen mn�motechnique} Ces deux types d'erreur sont num�rot�es dans
le m�me ordre que dans l'histoire du gar�on qui criait au loup : d'abord les
villageois pensaient qu'il y avait un loup alors qu'il n'y en avait pas (erreur
de premi�re esp�ce), mais � la fin les villageois pensaient qu'il n'y avait pas
de loup alors qu'il y en avait un (erreur de deuxi�me esp�ce). Ici l'hypoth�se
nulle est l'hypoth�se correspondant � l'�tat � par d�faut � du village, �
savoir sans loup.

Le niveau de signification $\alpha$ est ainsi la probabilit� de commettre une
erreur de premi�re esp�ce.

La probabilit� de commettre une erreur de deuxi�me esp�ce est g�n�ralement not�
$\beta$. La probabilit� de rejeter $\HH_0$ � raison, $1-\beta$, est appel�e la
\textbf{puissance} du test.


\section{Comparaison d'une moyenne observ�e � une moyenne th�orique}
\label{sec:test_moyenne}
Dans cette section, nous allons d�rouler un autre exemple de test statistique. 

Nous souhaitons tester l'hypoth�se selon laquelle les pigeons du Jardin du
Luxembourg ont un poids moyen de 300g. Nous disposons de mesures pour 40
pigeons, captur�s et pes�s par des �l�ves de l'�cole, dont la moyenne est de
350g et l'�cart-type 30g.

D�finissons une variable al�atoire r�elle, de carr� int�grable pour nous
simplifier la vie, $X$. $X$ mod�lise le poids d'un pigeon. Posons $\mu$
l'esp�rance de $X$ et $\sigma^2$ sa variance.

Nous posons $n=40$ ; les poids des 40 pigeons, $(x_1, x_2, \dots, x_n)$, sont
la r�alisation de l'�chantillon al�atoire $(X_1, X_2, \dots, X_n)$ compos� de
variables al�atoires ind�pendantes et identiquement distribu�es de m�me loi que
$X$.

Nous posons l'hypoth�se nulle � tester
\[
  \HH_0 : \mu = \mu_0, 
\]
avec $\mu_0 = 300g.$

Nous n'avons aucun a priori sur le poids des pigeons du Jardin du Luxembourg,
et formulons donc l'hypoth�se alternative bilat�rale
\[
  \HH_1 : \mu \neq \mu_0.
\]

Pour tester $\HH_0$, nous souhaitons d�terminer la probabilit� d'observer une
moyenne empirique $\hat{m}$ de 350g si l'esp�rance de $X$ est de 300g.  En
posant $M_n$ la moyenne empirique de l'�chantillon, nous souhaitons d�terminer
$\PP(M_n=\hat{m}|\mu=\mu_0)$.

Le th�or�me central limite nous indique que 
\[
  \frac{\sqrt{n} (M_n - \mu)}{\sigma}  \cvloi \Ncal(0, 1).
\]

Nous ne connaissons pas la variance $\sigma^2$ de $X$ ; cependant nous pouvons
l'estimer gr�ce � l'�cart-type empirique $\hat{\sigma} = 30g,$ et utiliser 
\[
  \frac{\sqrt{n} (M_n - \mu)}{\hat \sigma}  \cvloi \Ncal(0, 1).
\]
Nous ne rempla�ons pas $\mu$ par son estimation $\hat{m}$ : ce n'aurait aucun
sens, car nous cherchons justement � tester sa valeur.





TODO test pour comparer la moyenne de deux �chantillons, par exemple cas et
contr�les dans un essai clinique.







\subsection{Intervalle de confiance}
TODO REF Probabilit�s V

\section{Test d'ind�pendance du $\chi^2$}
\label{sec:chi2}

\section{Tests d'hypoth�ses multiples}
\label{sec:mht}


\begin{plusloin}
\item La \textbf{puissance} d'un test statistique est la probabilit� de rejeter
  $\HH_0$ si elle est fausse. C'est le compl�ment � 1 de l'erreur de deuxi�me
  esp�ce (erreur de Type II), qui consiste � ne pas rejeter $\HH_0$ alors
  qu'elle et fausse. La puissance est g�n�ralement d�licate � �valuer car elle
  n�cessite de sp�cifier correctement $\HH_1$. 
\item On dit d'un test statistique qu'il est sans biais si TODO
\item On dit d'un test statistique qu'il converge si TODO
\item Si la distribution de la statistique de test sous l'hypoth�se nulle n'est
  pas connue, on peut recourir � sa distribution empirique. Cette derni�re est
  obtenue par une technique de permutations. TODO
\item TODO ``Controverse'' autour des p-valeurs
\end{plusloin}


\chapter{Estimation de densit�}
%%-*- coding: iso-latin-1 -*-
\label{chap:estimation}

\paragraph{Notions :} �chantillon al�atoire, estimateur, estimation, biais d'un
estimateur, convergence d'un estimateur, estimation par maximisation de la
vraisemblance.

\paragraph{Objectifs p�dagogiques :}
\begin{itemize}
\item Choisir un estimateur, en particulier en d�terminant des propri�t�s
  telles que son biais, sa variance, ou sa convergence.
\item Proposer un estimateur, en particulier par maximisation de la
  vraisemblance.
\end{itemize}


\section{Inf�rence statistique}
Alors que la statistique descriptive se contente de \textit{d�crire} une
population ou un �chantillon de celle-ci, l'inf�rence statistique cherche �
tirer des conclusions sur une population � partir de l'�tude d'un �chantillon
de celle-ci. % Pour cela, il est n�cessaire de s'int�resser :
% \begin{itemize}
% \item aux techniques d'\textbf{�chantillonnage} (cf section~\ref{sec:echantilonnage})
%   permettant de construire des �chantillons d'une population ;
% \item � la \textbf{mod�lisation} permettant de supposer un mod�le probabiliste
%   sur la population ;
% \item aux techniques d'\textbf{estimation} (cf section~\ref{sec:estimation})
%   permettant de d�terminer (approximativement) un param�tre d'une population �
%   partir d'un �chantillon de celle-ci ;
% \item aux \textbf{tests d'hypoth�se} (cf chapitre~\ref{chap:tests}) permettant de valider
%   ou d'infirmer des hypoth�ses sur la population.
% \end{itemize}

\section{�chantillonnage}
\label{ref:echantilonnage}

Lorsque la population � �tudier est trop grande pour qu'il soit possible
d'observer chacun de ses individus, on �tudie alors une partie seulement de la
population. Cette partie est appel�e \textbf{�chantillon}. On parle alors de
\textbf{sondage}, par opposition � un \textbf{recensement}, qui consiste �
�tudier tous les individus d'une population.

\paragraph{Hypoth�ses de l'�chantillonnage}
Pour tirer parti d'un �chantillon, nous allons avoir besoin des hypoth�ses suivantes :
\begin{itemize}
\item La taille de la population est infinie ;
\item Les variables mesur�es sur la population peuvent �tre consid�r�es comme
  des variables al�atoires, dont les mesures sont des r�alisations. Les lois de
  probabilit� suivies par ces variables peuvent appartenir � une famille connue
  (e.g. loi gaussienne, loi de Poisson, etc.) ou �tre totalement
  inconnues. Dans le premier cas, on parlera de \textbf{statistique
    inf�rentielle param�trique} ; dans le deuxi�me, de \textbf{statistique
    inf�rentielle non-param�trique}.
\end{itemize}

\paragraph{Objectifs de la statistique inf�rentielle}
La statistique inf�rentielle a alors pour but d'\textbf{identifier les lois de
  probabilit� des variables al�atoires} en d�crivant les variables. Cela peut
prendre les formes suivantes :
\begin{itemize}
\item L'estimation, qui permet d'approcher les param�tres des lois (param�tre
  $p$ d'une loi de Bernoulli, indice et param�tre d'�chelle d'une loi Gamma) ou
  certaines de leurs caract�ristiques (esp�rance, variance, moments d'ordre
  sup�rieur, quartiles, etc.). C'est le sujet de ce chapitres.
\item Les tests d'hypoth�se, qui permettent d'infirmer ou de confirmer des
  hypoth�ses faites sur ces lois, leurs param�tres ou leurs
  caract�ristiques. Il s'agit par exemple de d�cider s'il est plausible que
  l'esp�rance d'une variable soit sup�rieure � une certaine valeur ; ou qu'une
  variable suive une loi normale. C'est le sujet du prochain chapitre.
\end{itemize}

\subsection{�chantillonnage al�atoire}
Dans la suite de ce chapitre, nous allons consid�rer que l'�chantillon obtenu
par sondage est obtenu par \textbf{�chantillonnage al�atoire simple} : on
pr�l�ve des individus dans la population au hasard, sans remise. Chaque
individu de la population a la m�me probabilit� $1/N$ d'�tre pr�lev�, o� $N$
est la taille de la population (on rappelle que $N \rightarrow \infty$) et ils
sont pr�lev�s ind�pendamment les uns des autres.

\paragraph{Remarque} D'autres techniques d'�chantillonnage sont possibles,
comme l'�chantillonnage al�atoire \textit{stratifi�}, dans lequel la population
est partitionn�e en strates selon une caract�ristique (par exemple, par tranche
d'�ge), et l'�chantillon est obtenu en proc�dant � un �chantillonnage al�atoire
simple dans chacune des strates, permettant d'obtenir pour chaque strate un
�chantillon de taille proportionnelle � la taille de strate dans la
population. Ainsi, les individus n'ont pas tous la m�me probabilit� d'�tre
tir�s : celle-ci d�pend de la taille de la strate � laquelle ils appartiennent.

% \begin{encadre}
  {Deux �chantillons $(x_1, x_2, \dots, x_n)$ et
  $(x^\prime_1, x^\prime_2, \dots, x^\prime_n)$ de tailles identiques $n$ de la
  m�me population seront donc diff�rents. On mod�lise cette variabilit� en
  consid�rant que chacun des individus $x_i$ ou $x^\prime_i$ est la r�alisation
  d'une m�me variable al�atoire $X_i$, o� $(X_1, X_2, \dots, X_n)$ est un
  vecteur al�atoire, dont les composantes sont ind�pendantes et identiquement
  distribu�es. 
  \begin{itemize}
  \item $(X_1, X_2, \dots, X_n)$ est appel� \textbf{�chantillon al�atoire} ;
  \item $(x_1, x_2, \dots, x_n)$ et
    $(x^\prime_1, x^\prime_2, \dots, x^\prime_n)$ sont deux �chantillons,
    c'est-�-dire deux \textit{r�alisations} de cet �chantillon al�atoire.
  \end{itemize}}
%\end{encadre}

Un indicateur statistique de l'�chantillon est alors la r�alisation d'une
variable al�atoire fonction de l'�chantillon al�atoire.

\begin{exemple} La moyenne d'un �chantillon,
$\bar{x} = \frac1n \sum_{i=1}^n x_i,$ est la r�alisation d'une variable
al�atoire $M_n$ d�finie par
\[
  M_n = \frac1n \sum_{i=1}^n X_i,
\]
qui est une fonction de l'�chantillon al�atoire $(X_1, X_2, \dots, X_n)$.
\end{exemple}

\section{Estimation ponctuelle}
Soit $(\Omega, \Acal, \PP)$ un espace probabilis�, $E$ un espace mesurable, et
$X$ une variable al�atoire � valeurs dans $E$. En pratique, dans la suite de ce
chapitre, nous consid�rerons des variables al�atoires r�elles ($E = \RR$ ou une
partie de $\RR$ telle que $\RR_+$ ou $\NN$), mais les id�es qui y sont
pr�sent�es peuvent �tre �tendues � $\RR^d$ ou � des espaces plus sophistiqu�s.

Soit $(X_1, X_2, \dots, X_n)$ un �chantillon al�atoire. Les $X_i$ sont
ind�pendants et identiquement distribu�es, de m�me loi $\PP_X$ que $X.$ Soit
$(x_1, x_2, \dots, x_n)$ un �chantillon, autrement dit une r�alisation de cet
�chantillon al�atoire.

Soit $\theta \in \RR$ une quantit� d�terministe (i.e. il ne s'agit pas d'une
variable al�atoire), qui d�pend uniquement de $\PP_X.$ Le but de l'estimation
ponctuelle est d'approcher au mieux la valeur de $\theta$. 

\begin{exemple} Si l'on fait l'hypoth�se que $X$ suit une loi
exponentielle (statistique inf�rentielle param�trique), on peut chercher �
estimer le param�tre $\theta$ de cette loi. On peut aussi chercher � estimer
l'esp�rance de $\PP_X,$ un de ses moments, un quantile, etc.
\end{exemple}

\subsection{D�finition d'un estimateur}
On appelle \textbf{estimateur} de $\theta$ une statistique de l'�chantillon
al�atoire $(X_1, X_2, \dots, X_n),$ c'est � dire une variable al�atoire
fonction de $(X_1, X_2, \dots, X_n) :$ un estimateur $\Theta_n$ de $\theta$
peut �tre d�fini par 
\[
  \Theta_n = g(X_1, X_2, \dots, X_n), \qquad g: E \rightarrow \RR.
\]

�tant donn� un �chantillon $(x_1, x_2, \dots, x_n)$ de $X$, on appelle
\textbf{estimation} de $\theta$ la valeur
\[
  \hat{\theta}_n = g(x_1, x_2, \dots, x_n) \in \RR,
\]
qui est donc une r�alisation de $\Theta_n$.

\paragraph{R�sum�}
�tant donn� une variable al�atoire r�elle $X$ � valeurs dans $E,$ un entier
$n \in \NN^*$, et une valeur $\theta$ � estimer qui ne d�pend que de la loi de
$X,$
\begin{itemize}
\item un �chantillon al�atoire $(X_1, X_2, \dots, X_n)$ est un vecteur
  al�atoire, dont les composantes sont iid de m�me loi que $X$ ;
\item un �chantillon $(x_1, x_2, \dots, x_n) \in \RR^n$ est une r�alisation de
  ce vecteur al�atoire ;
\item un estimateur de $\theta$ est une variable al�atoire $\Theta_n$ fonction
  de $(X_1, X_2, \dots, X_n)$ : \\ $\Theta_n = g(X_1, X_2, \dots, X_n)$, avec $g: E \rightarrow \RR$ ;
\item une estimation de $\theta$ est une r�alisation $\hat{\theta}_n$ de
  $\Theta_n$ : $\hat{\theta}_n = g(x_1, x_2, \dots, x_n) \in \RR.$
\end{itemize}

\subsection{Exemple : estimation de la moyenne par la moyenne empirique}
Consid�rons maintenant que $X$ est de carr� int�grable ($X \in \Lcal^2$),
d'esp�rance $m$ et de variance $\sigma^2 > 0$.

La \textbf{moyenne empirique} de $X$ est une variable al�atoire $M_n$, d�finie
par
\[
  M_n = \frac1n \sum_{i=1}^n X_i.
\]

$M_n$ est un estimateur de $m$ : �tant donn� un �chantillon
$(x_1, x_2, \dots, x_n),$ la valeur $\hat{m}_n = \frac1n \sum_{i=1}^n x_i$ est
une estimation de $m$.

� ce stade, rien ne nous permet de dire que $M_n$ est un \textit{bon}
estimateur de $m$ ; en effet, nous pourrions aussi d�finir
$\frac2n \sum_{i=1}^n X_i$ comme estimateur de la moyenne. Quelles sont les
\textit{propri�t�s} de $M_n$ qui nous font pr�f�rer poser $M_n$ comme nous
l'avons fait ? Quelques indices :

\begin{itemize}
\item $\EE[M_n] = m.$ Nous verrons que l'on dit que $M_n$ est un estimateur
  \textit{non-biais�} de $m$ (cf. section~\ref{sec:biais_estimateur}) ;
\item $\VV[M_n] = \frac{\sigma^2}{n}$ (voir preuve
  section~\ref{sec:estimation_proofs}) : plus l'�chantillon est grand, plus la
  variance de l'estimateur est faible, autrement dit plus sa r�alisation
  $\hat{m}_n$ sera proche de son esp�rance $m$. On parle ici de la
  \textit{pr�cision} de $M_n$ (cf. section~\ref{sec:precision_estimateur}) ;
\item Par la loi faible des grands nombres, $M_n \cvproba m.$ Nous
  verrons que l'on dit que $M_n$ est un estimateur \textit{convergent} de $m$
  (cf. section~\ref{sec:convergence_estimateur}) ;
\item Par la loi forte des grands nombres, $M_n \cvps m.$ Nous
  verrons que l'on dit que $M_n$ est un estimateur \textit{fortement convergent} de $m$
  (cf. section~\ref{sec:convergence_estimateur}.)
\end{itemize}



\section{Propri�t�s d'un estimateur}

\subsection{Biais d'un estimateur}
\label{sec:biais_estimateur}
Le \textbf{biais} d'un estimateur $\Theta_n$ de la quantit� $\theta$ est d�fini par 
\[
  B(\Theta_n) = \EE[\Theta_n] - \theta.
\]

$\Theta_n$ est dit \textbf{non-biais�} si $B(\Theta_n) = 0$, autrement dit si
son esp�rance vaut $\theta$.

La figure~\ref{fig:biais_variance} illustre les distributions de 3 estimateurs
d'une m�me quantit� $\theta$. On suppose ici que ce sont des gaussiennes. Les
estimateurs $\Theta$ et $\Theta^{\prime\prime}$ sont
non-biais�s. $\Theta^\prime$ est biais� : son esp�rance vaut
$\theta + \epsilon$.

\begin{figure}[h]
  \centering
  \includegraphics[width=0.5\textwidth]{figures/estimation/biais_variance}
  \caption{Distribution de 3 estimateurs de $\theta$.}
  \label{fig:biais_variance}
\end{figure}

\subsection{Exemple : Estimation non-biais�e de la variance}



\subsection{Pr�cision d'un estimateur}
\label{sec:precision_estimateur}

\begin{itemize}
\item estimateur non biais� : variance
\item exemple moyenne empirique
\item erreur quadratique moyenne  
\item exemple moyenne empirique
\end{itemize}



\subsection{Convergence d'un estimateur}
\label{sec:convergence_estimateur}

On utilise en anglais le terme de ``\textit{consistent}'', ce qui conduit les
francophones � parfois parler d'estimateur consistant plut�t que convergent.

\section{Estimation par maximum de vraisemblance}


\section{Preuves}
\label{sec:estimation_proofs}
\paragraph{Variance de la moyenne empirique} Soit $X$ une variable al�atoire
r�elle de carr� int�grable, d'esp�rance $m$ et de variance $\sigma^2$. Soient
$X_1, X_2, \dots, X_n$ ind�pendantes et identiquement distribu�es, de m�me loi
que $X$. 

Par d�finition de la variance, $\sigma^2 = \EE[X^2] - \EE[X]^2$ donc
$\EE[X^2] = \sigma^2 + m^2$.

Posons $M_n = \frac1n \sum_{i=1}^n X_i.$
\begin{align*}
  \VV[M_n] &= \EE[M_n^2] - \EE[M_n]^2 
   = \EE\left[\left(\frac1n \sum_{i=1}^n X_i\right)^2\right] - m^2 \\
  & = \frac1{n^2} \EE\left[ \sum_{i=1}^n X_i \sum_{j=1}^n X_j\right] - m^2 \\
  & = \frac1{n^2} \EE\left[ \sum_{i=1}^n \left(X_i^2 + \sum_{j \neq i }^n X_i X_j \right) \right] - m^2 \\
  & = \frac1{n} \left(\EE[X^2] + \sum_{j \neq i }^n \EE[X]^2 \right) - m^2 \\
  & = \frac1{n} \left(\sigma^2 + m^2 + (n-1) m^2  \right) - m^2 = \frac{\sigma^2}{n}.
\end{align*}



\begin{plusloin}
\item Plus la variance d'un estimateur est faible, plus cet estimateur
  peut-�tre consid�r� comme pr�cis. Un estimateur est dit \textit{efficace}
  s'il est non-biais� et que sa variance tend vers la \textit{borne de
    Cram�r-Rao} quand la taille de l'�chantillon tend vers l'infini. La borne
  de Cram�r-Rao est une borne inf�rieure de la variance d'un estimateur,
  obtenue en prenant un point de vue de th�orie de l'information sur
  l'estimation statistique.
\end{plusloin}





\part{Analyse exploratoire}
\chapter{R�duction de dimension}
%%-*- coding: iso-latin-1 -*-
\label{chap:dimred}

\todo{
  \begin{itemize}
  \item Coh�rence des notations avec les chapitres pr�c�dents.
  \item Lien avec le cours d'optimisation.
  \item All�ger / �laguer.
  \end{itemize}
}


\paragraph{Notions :} s�lection de variables ; extraction de variables ;
filtrage ; analyse en composantes principales ; analyse en composantes principales probabiliste.
\paragraph{Objectifs p�dagogiques :} 
\begin{itemize}      
  \setlength{\itemsep}{3pt}
\item Expliquer l'int�r�t de r�duire la dimension d'un jeu de donn�es ;
\item Faire la diff�rence entre la s�lection de variables et l'extraction de variables ;
\item Mettre en \oe{}uvre des m�thodes de s�lection de variable par filtrage ;
\item Projeter des donn�es sur un espace de plus petite dimension ;%par factorisation de matrice ; 
\item Mettre en \oe{}uvre des m�thodes d'extraction de variables.
\end{itemize}

\section{Notations}
Nous avons jusqu'� pr�sent travaill� sur des s�ries statistiques contenant une
seule variable. Cependant, dans la majorit� des probl�mes de sciences des
donn�es, nous disposons de plusieurs variables pour d�crire chaque individu.

� partir de maintenant, l'objet de nos �tudes ne sera plus un �chantillon
$(x_1, x_2, \dots, x_n),$ mais une matrice $X \in \RR^{n \times p}$
repr�sentant $n$ individus, ou observations, d�crites chacune par $p$
variables.

Nous essaierons de nous en tenir aux notations suivantes :
\begin{itemize}
\item Les lettres minuscules ($x$) repr�sentent un scalaire ;
\item les lettres minuscules surmont�es d'une fl�che ($\xx$) repr�sentent un
  vecteur ;
\item les lettres majuscules ($X$) repr�sentent une matrice, un �v�nement ou
  une variable al�atoire ;
\item les lettres calligraphi�es ($\XX$) repr�sentent un ensemble ou un espace ;
\item les {\it indices} correspondent � une variable tandis que les {\it
    exposants} correspondent � une observation : $x^i_j$ est la $j$-�me
  variable de la $i$-�me observation, et correspond � l'entr�e $X_{ij}$ de la
  matrice $X$ ;
\item $n$ est un nombre d'observations et $p$ un nombre de variables.
\end{itemize}

\section{Motivation}
Le but de la r�duction de dimension est de transformer une repr�sentation
$X \in \RR^{n \times p}$ des donn�es en une repr�sentation
$X^* \in \RR^{n \times m}$ o� $m \ll p$. Les raisons de cette d�marche sont
multiples : 

\begin{itemize}
\item \textbf{Visualiser les donn�es :} ce n'est pas t�che ais�e avec un nombre
  tr�s grand de variables. Comment visualiser $n$ points en plus de 2 ou 3
  dimensions ? Limiter les variables � un faible nombre de dimensions permet de
  visualiser les donn�es plus facilement, quitte � perdre un peu d'information
  lors de la transformation.
\item \textbf{R�duire les co�ts algorithmiques du traitement des donn�es :} d'un
  point de vue purement computationnel, r�duire la dimension des donn�es permet
  de r�duire d'une part l'espace qu'elles prennent en m�moire et d'autre part
  les temps de calcul. De plus, si certaines variables sont inutiles, ou
  redondantes, il n'est pas n�cessaire de les obtenir pour de nouvelles
  observations : cela permet de r�duire le co�t d'acquisition des donn�es.
\item \textbf{Am�liorer la qualit� du traitement des donn�es :} les algorithmes
  d'apprentissage supervis� ou de clustering sont g�n�ralement plus performants
  sur un faible nombre de variables. Tout d'abord, si certaines des variables
  ne sont pas pertinentes, elles risquent d'induire du biais dans les mod�les
  appris. De plus, les raisonnements d�velopp�s en faible dimension pour
  construire un algorithme d'apprentissage supervis� ne s'appliquent pas
  n�cessairement en haute dimension. C'est un ph�nom�ne connu sous le nom de
  {\it fl�au de la dimension}, ou {\it curse of dimensionality} en
  anglais. Pour plus de d�tails, reportez-vous � la section~\ref{sec:dimcurse}.
\end{itemize}


Deux possibilit�s s'offrent � nous pour r�duire la dimension de nos donn�es :
\begin{itemize}
\item la \textbf{s�lection de variables}, qui consiste � �liminer un nombre
  $p-m$ de variables de nos donn�es ;
\item l'\textbf{extraction de variables}, qui consiste � {\it cr�er} $m$
  nouvelles variables � partir des $p$ variables dont nous disposons
  initialement.
\end{itemize}

\paragraph{S�lection de variables} La s�lection de variables consiste �
�liminer des variables peu informatives. Dans le cas non-supervis�, il s'agit
par exemple d'�liminer des variables dont la variance est tr�s faible (leur
valeur �tant � peu pr�s la m�me pour chaque individu, elle n'apporte aucune
information permettant de distinguer deux individus) ou qui sont corr�l�es �
une autre variable (elles apportent alors la m�me information).

Dans le cas supervis�, il s'agit aussi d'�liminer des variables qui ne sont pas
pertinentes par rapport � la t�che de pr�diction. On peut par exemple
\begin{itemize}
\item �liminer, par exemple � l'aide d'un test du chi2 comme vu dans la PC1,
  les variables ind�pendantes de l'�tiquette � pr�dire. Remarquez n�anmoins que
  deux variables chacune ind�pendante de l'�tiquette peuvent �tre tr�s
  informatives quand on les consid�re simultan�ment\footnote{Consid�rez par exemple, en deux dimensions, un probl�me de classification dans lequel l'�tiquette $y$ est donn�e par $y = x_1 \oplus x_2$} ;
\item chercher � �liminer des variables qui n'am�liorent pas la performance
  d'un algorithme pr�cis.
\end{itemize}
Nous reviendrons sur la
s�lection de variables supervis�e quand nous parlerons du lasso
(section~\ref{sec:lasso}).


%La suite de ce chapitre d�taille des exemples de ces deux approches.

\section{Analyse en composantes principales}
\label{sec:pca}
La m�thode la plus classique pour r�duire la dimension d'un jeu de donn�es par
extraction de variables est l'\textbf{analyse en composantes principales}, ou
{\it ACP}. On parle aussi souvent de {\it PCA}, de son nom anglais {\it
  Principal Component Analysis}.

\subsection{Maximisation de la variance}
L'id�e centrale d'une ACP est de repr�senter les donn�es de sorte � maximiser
leur variance selon les nouvelles dimensions, afin de pouvoir continuer �
distinguer les exemples les uns des autres dans leur nouvelle repr�sentation
(cf. figure~\ref{fig:data_variance}).
\begin{figure}[h]
  \centering
  \includegraphics[width=0.4\textwidth]{figures/dimred/data_variance}
  \caption{La variance des donn�es en deux dimensions est maximale selon l'axe
    indiqu� par la fl�che.}
  \label{fig:data_variance}
\end{figure}

Formellement, une nouvelle repr�sentation de $\XX$ est d�finie par une base
orthonorm�e sur laquelle projeter la matrice de donn�es $X$.

Une \textbf{analyse en composantes principales}, ou {\it ACP}, de la matrice
$X \in \RR^{n \times p}$ est une transformation lin�aire orthogonale qui permet
d'exprimer $X$ dans une nouvelle base orthonorm�e, de sorte que la plus grande
variance de $X$ par projection s'aligne sur le premier axe de cette nouvelle
base, la seconde plus grande variance sur le deuxi�me axe, et ainsi de suite.

Les axes de cette nouvelle base sont appel�s les \textbf{composantes
  principales}, abr�g�es en {PC} pour {\it Principal Components}.

\begin{attention}
  Dans la suite de cette section, nous supposons que les variables ont �t� {\it
    standardis�es} de sorte � toutes avoir une moyenne de 0 et une variance de
  1, pour �viter que les variables qui prennent de grandes valeurs aient plus
  d'importance que celles qui prennent de faibles valeurs. C'est un pr�-requis
  de l'application de l'ACP.  Cette standardisation s'effectue en centrant la
  moyenne et en r�duisant la variance de chaque variable :
  \begin{equation}
    \label{eq:standardization}
    x^i_j \leftarrow \frac{x^i_j - \bar{x_j}}{\sqrt{\frac1n 
        \sum_{l=1}^n (x^l_j - \bar{x_j})^2}},
  \end{equation}
  o� $\bar{x_j} = \frac1n \sum_{l=1}^n x^l_j.$ On dira alors que $X$ est {\it
    centr�e} : chacune de ses colonnes a pour moyenne 0.
\end{attention}

\paragraph{Th�or�me} 
Soit $X \in \RR^{n \times p}$ une matrice {\it centr�e} de covariance
$\Sigma = \frac1n X^\top X$. Les composantes principales de $X$ sont les
vecteurs propres de $\Sigma$, ordonn�s par valeur propre d�croissante.

\paragraph{Preuve}
Commen�ons par d�montrer que, pour tout vecteur $\ww \in \RR^p$, la variance de
la projection de $X$ sur $\ww$ vaut $w^\top \Sigma w.$

La projection de $X \in \RR^{n \times p}$ sur $\ww \in \RR^p$ est le vecteur
$\zz = X w.$ Comme $X$ est centr�e, la moyenne de $\zz$ vaut
\begin{equation*}
  \frac1n \sum_{i=1}^n z_i = \frac1n \sum_{i=1}^n \sum_{j=1}^p x^i_j w_j = 
  \frac1n \sum_{j=1}^p  w_j \sum_{i=1}^n x^i_j = 0.
\end{equation*}
Sa variance vaut
\begin{equation*}
  \text{Var}[\zz] = \frac1n \sum_{i=1}^n zz_i^2 = \frac1n \ww^\top X^\top X \ww
  = \ww^\top \Sigma \ww.
\end{equation*}    

Appelons maintenant $\ww_1 \in \RR^p$ la premi�re composante
principale. $\ww_1$ est orthonorm� et tel que la variance de $X \ww_1$ soit
maximale :
\begin{equation}
  \label{eq:pc1}
  \ww_1 = \argmax_{\ww \in \RR^p} \ww^\top \Sigma \ww \text{ avec } \ltwonorm{\ww_1}=1.
\end{equation}
Il s'agit d'un probl�me d'optimisation quadratique sous contrainte d'�galit�,
que l'on peut r�soudre en introduisant le multiplicateur de Lagrange
$\alpha_1 > 0$ et en �crivant le lagrangien
\begin{equation*}
  L(\alpha_1, \ww) = \ww^\top \Sigma \ww - \alpha_1 
  \left( \ltwonorm{\ww} - 1 \right).
\end{equation*}
Par dualit� forte, % $\min_{\ww \in \RR^p} w^\top \Sigma w = \max_{\alpha_1}
    % \inf_{\ww \in \RR^p} L(\alpha_1, \ww).$
le maximum de $\ww^\top \Sigma \ww$ sous la contrainte $\ltwonorm{\ww_1}=1$ est
�gal � $\min_{\alpha_1} \sup_{\ww \in \RR^p} L(\alpha_1, \ww).$ Le supremum du
lagrangien est atteint en un point o� son gradient s'annule, c'est-�-dire qui
v�rifie
\begin{equation*}
  2 \Sigma \ww - 2 \alpha_1 \ww = 0.
\end{equation*}
Ainsi, $\Sigma \ww_1 = \alpha_1 \ww_1$ et $(\alpha_1, \ww_1)$ forment un couple
(valeur propre, vecteur propre) de $\Sigma$.

Parmi tous les vecteurs propres de $\Sigma$, $\ww_1$ est celui qui maximise la
variance $\ww_1^\top \Sigma \ww_1 = \alpha_1 \ltwonorm{\ww_1} = \alpha_1.$
Ainsi, $\alpha_1$ est la plus grande valeur propre de $\Sigma$ (rappelons que
$\Sigma$ �tant d�finie par $X^\top X$ est semi-d�finie positive et que toutes
ses valeurs propres sont positives.)

La deuxi�me composante principale de $X$ v�rifie
\begin{equation}
  \label{eq:pc2}
  \ww_2 = \argmax_{\ww \in \RR^p} \ww^\top \Sigma \ww \text{ avec } \ltwonorm{\ww_2}=1
  \text{ et } \ww^\top\ww_1=0.
\end{equation}
Cette derni�re contrainte nous permet de garantir que la base des composantes
principales est orthonorm�e.

Nous introduisons donc maintenant deux multiplicateurs de Lagrange
$\alpha_2 > 0$ et $\beta_2 > 0$ et obtenons le lagrangien
\begin{equation*}
  L(\alpha_2, \beta_2, \ww) = \ww^\top \Sigma \ww - \alpha_2 
  \left(\ltwonorm{\ww}^2 - 1 \right)
  - \beta_2 \ww^\top\ww_1.
\end{equation*}
Comme pr�c�demment, son supremum en $\ww$ est atteint en un point o� son
gradient s'annule :
\begin{equation*}
  2 \Sigma \ww_2 - 2 \alpha_2 \ww_2 - \beta_2 \ww_1 = 0.
\end{equation*}
En multipliant � gauche par $\ww_1^\top$, on obtient
\begin{equation*}
  2 \ww_1^\top \Sigma \ww_2 - 2 \alpha_2 \ww_1^\top \ww_2 - 
  \beta_2 \ww_1^\top \ww_1 = 0
\end{equation*}
d'o� l'on conclut que $\beta_2=0$ et, en rempla�ant dans l'�quation pr�c�dente,
que, comme pour $\ww_1$, $2 \Sigma \ww_2 - 2 \alpha_1 \ww_2 = 0$.  Ainsi
$(\alpha_2, \ww_2)$ forment un couple (valeur propre, vecteur propre) de
$\Sigma$ et $\alpha_2$ est maximale : il s'agit donc n�cessairement de la
deuxi�me valeur propre de $\Sigma$.
    
Le raisonnement se poursuit de la m�me mani�re pour les composantes principales
suivantes. \hfill $\square$

  
\paragraph{Remarque} Alternativement, on peut prouver ce th�or�me
en observant que $\Sigma$, qui est par construction d�finie positive, est
diagonalisable par un changement de base orthonorm�e :
$\Sigma = Q^\top \Lambda Q$, o� $\Lambda \in \mathbb{R}^{p \times p}$ est une
matrice diagonale dont les valeurs diagonales sont les valeurs propres de
$\Sigma$.  Ainsi,
\begin{equation*}
  \ww_1^\top \Sigma \ww_1 = \ww_1^\top Q^\top \Lambda Q \ww_1 = 
  \left(Q \ww_1 \right)^\top \Lambda \left(Q \ww_1 \right).
\end{equation*}
Si l'on pose $\vv = Q \ww_1$, il s'agit donc de trouver $\vv$ de norme 1 ($Q$
�tant orthonorm�e et $\ww_1$ de norme 1) qui maximise
$\sum_{j=1}^p v_j^2 \lambda_j$.  Comme $\Sigma$ est d�finie positive,
$\lambda_j \geq 0 \; \forall j=1, \dots, p$. De plus, $\ltwonorm{\vv} = 1$
implique que $0 \leq v_j^1 \leq 1 \; \forall j=1, \dots, p.$ Ainsi,
$\sum_{j=1}^p v_j^2 \lambda_j \leq \max_{j=1, \dots, p} \lambda_j \sum_{j=1}^p
v_j^2 \leq \max_{j=1, \dots, p} \lambda_j$
et ce maximum est atteint quand $v_j=1$ et $v_k=0 \; \forall k \neq j$. On
retrouve ainsi que $\ww_1$ est le vecteur propre correspondant � la plus grande
valeur propre de $\Sigma$, et ainsi de suite.


\subsection{D�composition en valeurs singuli�res}
\paragraph{Th�or�me} 
Soit $X \in \RR^{n \times p}$ une matrice {\it centr�e}. Les composantes
principales de $X$ sont ses vecteurs singuliers � droite ordonn�s par valeur
singuli�re d�croissante.

\paragraph{Preuve}
Si l'on �crit $X$ sous la forme $U D V^\top$ o� $U \in \RR^{n \times n}$ et
$V \in \RR^{p \times p}$ sont orthogonales, et $D \in \RR^{n \times p}$ est
diagonale, alors
\begin{equation*}
  \Sigma = X^\top X = V D U^\top U D V^\top = V D^2 V^\top
\end{equation*}
et les valeurs singuli�res de $X$ (les entr�es de $D$) sont les racines carr�es
des valeurs propres de $\Sigma$, tandis que les vecteurs singuliers � droite de
$X$ (les colonnes de $V$) sont les vecteurs propres de $\Sigma$. \hfill $\square$

  
En pratique, les impl�mentations de la d�composition en valeurs singuli�res (ou
SVD) sont num�riquement plus stables que celles de d�composition spectrale. On
pr�f�rera donc calculer les composantes principales de $X$ en calculant la SVD
de $X$ plut�t que la d�composition spectrale de $X^\top X$.

\subsection{Choix du nombre de composantes principales}
R�duire la dimension des donn�es par une ACP implique de {\it choisir} un
nombre de composantes principales � conserver. Pour ce faire, on utilise la
\textbf{proportion de variance expliqu�e} par ces composantes : la variance de $X$
s'exprime comme la trace de $\Sigma$, qui est elle-m�me la somme de ses valeurs
propres.

Ainsi, si l'on d�cide de conserver les $m$ premi�res composantes principales de
$X$, la proportion de variance qu'elles expliquent est
\begin{equation*}
  \frac{\alpha_1 + \alpha_2 + \dots + \alpha_m}{\text{Tr}(\Sigma)}
\end{equation*}
o� $\alpha_1 \geq \alpha_2 \geq \dots \geq \alpha_p$ sont les valeurs propres
de $\Sigma$ par ordre d�croissant.
  
Il est classique de s'int�resser � l'�volution, avec le nombre de composantes,
soit de la proportion de variance expliqu�e par chacune d'entre elles, soit �
cette proportion cumul�e, que l'on peut repr�senter visuellement sur un {\it
  scree plot} (figure~\ref{fig:scree_plot}). Ce graphe peut nous servir �
d�terminer :
\begin{itemize}
\item soit le nombre de composantes principales qui explique un pourcentage de
  la variance que l'on s'est initialement fix� (par exemple, sur la
  figure~\ref{fig:scree_plot_cumul}, $95\%$) ;
\item soit le nombre de composantes principales correspondant au � coude � du
  graphe, � partir duquel ajouter une nouvelle composante principale ne semble
  plus informatif.
\end{itemize}

\begin{figure}[h]
  \begin{subfigure}[t]{0.48\textwidth}
    \centering
    \includegraphics[width=\textwidth]{figures/dimred/scree_plot}
    \caption{Pourcentage de variance expliqu� par chacune des composantes
      principales. � partir de 6 composantes principales, ajouter de nouvelles
      composantes n'est plus vraiment informatif.}
    \label{fig:scree_plot}
  \end{subfigure} \hfill
  \begin{subfigure}[t]{0.48\textwidth}
    \includegraphics[width=\textwidth]{figures/dimred/scree_plot_cumul}  
    \caption{Pourcentage cumul� de variance expliqu�e par chacune des
      composantes principales. Si on se fixe une proportion de variance
      expliqu�e de $95\%$, on peut se contenter de 10 composantes principales.}
    \label{fig:scree_plot_cumul}
  \end{subfigure}
  \caption{Pour choisir le nombre de composantes principales, on utilise le
    pourcentage de variance expliqu�e.}
\end{figure}

\section{Factorisation de la matrice des donn�es}
Soit un nombre $m$ de composantes principales, calcul�es par une ACP,
repr�sent�es par une matrice $W \in \RR^{p \times m}$. La repr�sentation
r�duite des $n$ observations dans le nouvel espace de dimension $m$ s'obtient
en projetant $X$ sur les colonnes de $W$, autrement dit en calculant
\begin{equation}
  \label{eq:reduced_rep}
  H = W^\top X.
\end{equation}

La matrice $H \in \RR^{m \times n}$ peut �tre interpr�t�e comme une
\textbf{repr�sentation latente} (ou cach�e, {\it hidden} en anglais d'o� la
notation $H$) des donn�es. C'est cette repr�sentation que l'on a cherch� �
d�couvrir gr�ce � l'ACP.

Les colonnes de $W$ �tant des vecteurs orthonorm�s (il s'agit de vecteurs
propres de $X X^\top$), on peut multiplier l'�quation~\ref{eq:reduced_rep} �
gauche par $W$ pour obtenir une {\it factorisation} de $X$ :
\begin{equation}
  \label{eq:pca_factor}
  X = W H.
\end{equation}
On appelle ainsi les lignes de $H$ les \textbf{facteurs} de $X$.


Cette factorisation s'inscrit dans le cadre plus g�n�ral de \textbf{l'analyse
  factorielle}.

Supposons que les observations $\{\xx^1, \xx^2, \dots, \xx^n\}$ soient les
r�alisations d'une variable al�atoire $p$-dimensionnelle $\xx$, g�n�r�e par le
mod�le
\begin{equation}
  \label{eq:fa_model}
  \xx = W \hh + \epsilon,
\end{equation}
o� $\hh$ est une variable al�atoire $m$-dimensionnelle qui est la
repr�sentation latente de $\xx$, et $\epsilon$ un bruit gaussien :
$\epsilon \sim \Ncal(0, \Psi).$

Supposons les donn�es centr�es en $0$, et les variables latentes
$\hh^1, \hh^2, \dots, \hh^n$ (qui sont des r�alisations de $\hh$) ind�pendantes
et gaussiennes de variance unitaire, c'est-�-dire $\hh \sim \Ncal(0, I_m)$ o�
$I_m$ est la matrice identit� de dimensions $m \times m$. Alors $W \hh$ est
centr�e en $0$, et sa covariance est $WW^\top.$ Alors
$\xx \sim \Ncal(0, WW^\top + \Psi)$.

Si on consid�re de plus que $\epsilon$ est {\it isotropique}, autrement dit que
$\Psi = \sigma^2 I_p$, alors $\xx \sim \Ncal(0, WW^\top + \sigma^2 I_p)$, on
obtient ce que l'on appelle \textbf{ACP probabiliste}. Les
param�tres $W$ et $\sigma$ de ce mod�le peuvent �tre obtenus par maximum de
vraisemblance.

L'ACP classique est un cas limite de l'ACP probabiliste, obtenu quand la
covariance du bruit devient infiniment petite ($\sigma^2 \rightarrow 0$).

Plus g�n�ralement, on peut faire l'hypoth�se que les variables observ�es
$x_1, x_2, \dots, x_p$ sont conditionnellement ind�pendantes �tant donn�es les
variables latentes $h_1, h_2, \dots, h_m.$ Dans ce cas, $\Psi$ est une matrice
diagonale, $\Psi = \text{diag}(\psi_1, \psi_2, \dots, \psi_p)$, o� $\psi_j$
d�crit la variance sp�cifique � la variable $x_j$. Les valeurs de $W$, $\sigma$
et $\psi_1, \psi_2, \dots, \psi_p$ peuvent encore une fois �tre obtenues par
maximum de vraisemblance. C'est ce que l'on appelle \textbf{l'analyse
  factorielle}.

Dans l'analyse factorielle, nous ne faisons plus l'hypoth�se que les nouvelles
dimensions sont orthogonales. En particulier, il est donc possible d'obtenir
des dimensions d�g�n�r�es, autrement dit des colonnes de $W$ dont toutes les
coordonn�es sont $0$.

% \subsection{Approches non lin�aires}
% De nombreuses autres approches ont �t� propos�es pour r�duire la dimension des
% donn�es de mani�re non lin�aire. Parmi elles, nous en abordons ici
% quelques-unes parmi les plus populaires ; les expliquer de mani�re d�taill�e
% d�passe le propos de cet ouvrage introductif.

% \subsubsection{Analyse en composantes principales � noyau}
% Nous commencerons par noter que l'analyse en composantes principales se pr�te �
% l'utilisation de l'astuce du noyau (cf. section~\ref{sec:kernel_trick}). La
% m�thode qui en r�sulte est appel�e {\it kernel PCA}, ou {\it kPCA}.

% \subsubsection{Positionnement multidimensionnel}
% Le {\it positionnement multidimensionnel}, ou {\it multidimensional scaling}
% ({\it MDS})~\citep{cox1994}, se base sur une matrice de {\it dissimilarit�}
% $D \in \RR^{n \times n}$ entre les observations : il peut s'agir d'une distance
% m�trique, mais ce n'est pas n�cessaire. Le but de l'algorithme est alors de
% trouver une repr�sentation des donn�es qui pr�serve cette dissimilarit� :
% \begin{equation}
%   \label{eq:mds}
%   X^* = \argmin_{Z \in \RR^{n \times m}} \sum_{i=1}^n \sum_{l=i+1}^n \left( 
%     \ltwonorm{\zz^i - \zz^l} - D_{il}\right)^2.
% \end{equation}
% Si l'on utilise la distance euclidienne comme dissimilarit�, alors le MDS est
% �quivalent � une ACP.

% Le positionnement multidimensionnel peut aussi s'utiliser pour positionner dans
% un espace de dimension $m$ des points dont on ne conna�t pas les
% coordonn�es. Il s'applique par exemple tr�s bien � repositionner des villes sur
% une carte � partir uniquement des distances entre ces villes.

% Une des limitations de MDS est de ne chercher � conserver la distance entre les
% observations que globalement. Une fa�on efficace de construire la matrice de
% dissimilarit� de MDS de sorte � conserver la structure locale des donn�es est
% l'algorithme {\it IsoMap}~\citep{tenenbaum2000}. Il s'agit de construire un
% graphe de voisinage entre les observations en reliant chacune d'entre elles �
% ses $k$ plus proches observations voisines. Ces ar�tes peuvent �tre pond�r�e
% par la distance entre les observations qu'elles relient. Une dissimilarit�
% entre observations peut ensuite �tre calcul�e sur ce graphe de voisinage, par
% exemple via la longueur du plus court chemin entre deux points.

% \subsubsection{t-SNE}
% Enfin, l'algorithme {\it t-SNE}, pour {\it t-Student Neighborhood Embedding},
% propos� en 2008 par Laurens van der Maaten and Geoff Hinton, propose
% d'approcher la distribution des distances entre observations par une loi de
% Student~\citep{vandermaaten2008}. Pour chaque observation $\xx^i$, on d�finit
% $P_i$ comme la loi de probabilit� d�finie par
% \begin{equation}
%   P_i(\xx) = \frac1{\sqrt{2 \pi \sigma^2}} \exp \left( - \frac{
%       \ltwonorm{\xx - \xx^i}^2}{2 \sigma^2} \right).
%   \label{eq:tsne_distr}    
% \end{equation}
% t-SNE consiste alors � r�soudre
% \begin{equation}
%   \label{eq:tsne}
%   \argmin_{Q} \sum_{i=1}^n KL(P_i||Q_i)
% \end{equation}
% o� $KL$ d�note la divergence de Kullback-Leibler (voir
% section~\ref{sec:cross_entropy}) et $Q$ est choisie parmi les distributions de
% Student de dimension inf�rieure � $p$. Attention, cet algorithme trouve un
% minimum local et non global, et on pourra donc obtenir des r�sultats diff�rents
% en fonction de son initialisation. De plus, sa complexit� est quadratique en le
% nombre d'observations.
% \end{cours}

% \begin{pointsclefs}
% \item R�duire la dimension des donn�es avant d'utiliser un algorithme
%   d'apprentissage supervis� permet d'am�liorer ses besoins en temps et en
%   espace, mais aussi ses performances.
% \item On distingue la s�lection de variables, qui consiste � �liminer des
%   variables redondantes ou peu informatives, de l'extraction de variable, qui
%   consiste � g�n�rer une nouvelle repr�sentation des donn�es.
% \item Projeter les donn�es sur un espace de dimension 2 gr�ce �, par exemple,
%   une ACP ou t-SNE, permet de les visualiser.
% \item De nombreuses m�thodes permettent de r�duire la dimension des variables. 
% \end{pointsclefs}

\section{Compl�ments}
\subsection{Fl�au de la dimension}
\label{sec:dimcurse}
En haute dimension, les individus ont tendance � tous �tre �loign�s les uns des
autres. Pour comprendre cette assertion, pla�ons-nous en dimension $p$ et
consid�rons l'hypersph�re $\Scal(\xx, R)$ de rayon $R \in \RR_+^*$ centr�e sur
une observation $\xx$, ainsi que l'hypercube $\Ccal(\xx, R)$ circonscrit �
cette hypersph�re. Le volume de $\Scal(\xx)$ vaut
$\frac{2 R^p \pi^{p/2}}{p \Gamma(p/2)}$, tandis que celui de $\Ccal(\xx, R)$,
dont le c�t� a pour longueur $2R$, vaut $2^p R^p$. Ainsi
\begin{equation*}
  \lim_{p \rightarrow \infty} \frac{\text{Vol}(\Ccal(\xx, R))}{
    \text{Vol}(\Scal(\xx, R))} = 0.
\end{equation*}
Cela signifie que la probabilit� qu'un exemple situ� dans $\Ccal(\xx, R)$
appartienne � $\Scal(\xx, R)$, qui vaut $\frac{\pi}4 \approx 0.79$ lorsque
$p=2$ et $\frac{\pi}6 \approx 0.52$ lorsque $p=3$, devient tr�s faible quand
$p$ est grand : les donn�es ont tendance � �tre �loign�es les unes des autres.

Cela implique que les algorithmes d�velopp�s en utilisant une notion de
similarit� ou distance entre individus ne fonctionnent pas n�cessairement en
grande dimension. Ainsi, r�duire la dimension peut �tre n�cessaire � la
construction de bons mod�les d'apprentissage.


\begin{plusloin}
\item \todo{Plus de variantes de l'analyse factorielle, e.g. NMF}
\item \todo{Representation learning}
\item \todo{MDS, IsoMap, tSNE}
% \item Le tutoriel de \citet{shlens2014} est une introduction d�taill�e � l'analyse en
%   composantes principales.
% \item Pour une revue des m�thodes de s�lection de variables, on pourra se
%   r�ferer �~\citet{guyon2003}.
% \item Pour plus de d�tails sur la NMF, on pourra par exemple se tourner
%   vers~\citet{lee1999}
% \item Pour plus de d�tails sur les m�thodes des s�lection de sous-ensemble de
%   variables (wrapper methods), on pourra se r�f�rer �
%   l'ouvrage de~\citet{miller1990}
% \item Une page web est d�di�e � Isomap:
%   \url{http://web.mit.edu/cocosci/isomap/isomap.html}.
% \item Pour plus de d�tails sur l'utilisation de t-SNE, on pourra se r�f�rer �
%   la page \url{https://lvdmaaten.github.io/tsne/} ou � la publication
%   interactive de~\citet{wattenberg2016}.
\end{plusloin}

% \section*{Bibliographie}
% \vspace{-25pt}
% \begin{thebibliography}{99}
% \bibitem[\protect\astroncite{Cox et Cox}{1994}]{cox1994} Cox, T.~F. et Cox,
%   M. A.~A. (1994).  \newblock {\em Multidimensional Scaling}.  \newblock
%   Chapman and Hall., London.

% \bibitem[\protect\astroncite{Guyon et Elisseeff}{2003}]{guyon2003} Guyon,
%   I. et Elisseeff, A. (2003).  \newblock An introduction to variable and
%   feature selection.  \newblock {\em Journal of Machine Learning Research},
%   3:1157--1182.

% \bibitem[\protect\astroncite{Hinton}{2002}]{hinton2002} Hinton, G.~E. (2002).
%   \newblock Training product of experts by minimizing contrastive divergence.
%   \newblock {\em Neural Computation}, 14:1771--1800.

% \bibitem[\protect\astroncite{Hinton et Salakhutdinov}{2006}]{hinton2006}
%   Hinton, G.~E. et Salakhutdinov, R.~R. (2006).  \newblock Reducing the
%   dimensionality of data with neural networks.  \newblock {\em Science},
%   313:504--507.

% \bibitem[\protect\astroncite{Kozachenko et Leonenko}{1987}]{kozachenko1987}
%   Kozachenko, L.~F. et Leonenko, N.~N. (1987).  \newblock A statistical
%   estimate for the entropy of a random vector.  \newblock {\em Problemy
%     Peredachi Informatsii}, 23:9--16.

% \bibitem[\protect\astroncite{Lee et Seung}{1999}]{lee1999} Lee, D.~D. et
%   Seung, H.~S. (1999).  \newblock Learning the parts of objects by non-negative
%   matrix factorization.  \newblock {\em Nature}, 401(6755):788--791.

% \bibitem[\protect\astroncite{Miller}{1990}]{miller1990} Miller, A.~J. (1990).
%   \newblock {\em Subset Selection in Regression}.  \newblock Chapman and Hall.,
%   London.

% \bibitem[\protect\astroncite{Shlens}{2014}]{shlens2014} Shlens, J. (2014).
%   \newblock A {Tutorial} on {Principal} {Component} {Analysis}.  \newblock {\em
%     arXiv [cs, stat]}.  \newblock arXiv: 1404.1100.

% \bibitem[\protect\astroncite{Smolensky}{1986}]{smolensky1986} Smolensky,
%   P. (1986).  \newblock Information processing in dynamical systems:
%   foundations of harmony theory.  \newblock In {\em Parallel Distributed
%     Processing: Explorations in the Microstructure of Cognition}, volume 1:
%   Foundations, chapter~6, pages 194--281. MIT Press, Cambridge, MA.

% \bibitem[\protect\astroncite{Tenenbaum et~al.}{2000}]{tenenbaum2000} Tenenbaum,
%   J.~B., de~Silva, V., et Langford, J.~C. (2000).  \newblock A global
%   geometric framework for nonlinear dimensionality reduction.  \newblock {\em
%     Science}, 290(5500):2319--2323.

% \bibitem[\protect\astroncite{Tipping et Bishop}{1999}]{tipping1999} Tipping,
%   M.~E. et Bishop, C.~M. (1999).  \newblock Probabilistic principal components
%   analysis.  \newblock {\em Journal of the Royal Statistical Society Series B},
%   61:611--622.

% \bibitem[\protect\astroncite{van~der Maaten et Hinton}{2008}]{vandermaaten2008}
%   van~der Maaten, L. et Hinton, G. (2008).  \newblock Visualizing data using
%   t-{SNE}.  \newblock {\em Journal of Machine Learning Research}, 9:2579--2605.

% \bibitem[\protect\astroncite{Wattenberg et~al.}{2016}]{wattenberg2016}
%   Wattenberg, M., Vi�gas, F., et Johnson, I. (2016).  \newblock How to use
%   t-{SNE} effectively.  \newblock {\em Distill}.  \newblock
%   \url{http://distill.pub/2016/misread-tsne}.
% \end{thebibliography}



%%% Local Variables:
%%% mode: latex
%%% TeX-master: "sdd_2020_poly"
%%% End:


\chapter{Bonnes pratiques}
%%-*- coding: iso-latin-1 -*-
\label{chap:pratiques}

\paragraph{Notions :} visualisation de donn�es, repr�sentativit� des donn�es,
�quit� des algorithmes, confidentialit� des donn�es, anonymisation,
responsabilit�.

\paragraph{Objectifs p�dagogiques :} 
\begin{itemize}      
  \setlength{\itemsep}{3pt}
\item S'interroger sur la pertinence d'une analyse de donn�es et la validit�
  des conclusions qui en sont tir�es.
\end{itemize}

La science des donn�es n'est pas uniquement une discipline technique : comme
souvent en ing�nierie, nous ne pouvons pas dissocier les calculs que nous
faisons de la question pos�e ni de leur utilisation. Ce chapitre n'a pas
vocation � �tre un cours d'�thique\footnote{L'�thique peut �tre
  d�finie comme l'�tude de la justification d'ue action � partir de normes,
  r�gles juridiques ou d�ontologiques, valeurs morales, intuitions et
  traditions qui peuvent �tre multiples et contradictoires au sein d'une m�me
  socit�t�.}, mais � vous donner quelques points d'entr�e pour vous amener �
vous poser des questions sur l'usage de la science des donn�es, de
l'apprentissage automaique et de l'intelligence artificielle. Pour cette
raison, vous trouverez plus de liens externes qu'� l'habitude � travers le
texte de ce chapitre, pointant tant vers des publications scientifiques que des
blogs de vulgarisation ou des articles de presse grand public. N'h�sitez pas �
poursuivre vos propres lectures sur le sujet.

Nous motiverons ce chapitre par deux citations : la premi�re, attribu�e �
Benjamin Disraeli par Mark Twain, ``\textit{There are three kinds of lies:
  lies, damned lies, and statistics}'', et la seconde, attribu�e � George
Box, ``\textit{All models are wrong, but some are useful}''.

\section{Visualisation de donn�es}
La fa�on dont vous choisissez de repr�senter vos donn�es ou vos r�sultats a un
impact fort sur le message que vous essayez de faire passer. 

Mi-mai 2020, le
Department of Public Health de l'�tat de G�orgie (�tats-Unis d'Am�rique) a
publi� le diagramme en barres de la figure~\ref{fig:georgia_wtf_barplot}. Regardez bien l'axe des abscisses : le message vous semble-t-il le m�me quand
les dates sont ordonn�es de mani�re chronologique, comme sur la figure~\ref{fig:georgia_fixed_barplot} ?

\begin{figure}[h]
  \centering
  \begin{subfigure}[t]{0.47\textwidth}
    \centering
    \includegraphics[width=\textwidth]{figures/pratiques/georgia_wtf_barplot}
    \caption{Premi�re version du diagramme en barres.}
    \label{fig:georgia_wtf_barplot}
  \end{subfigure} \hfill
  \begin{subfigure}[t]{0.47\textwidth}
    \includegraphics[width=\textwidth]{figures/pratiques/georgia_fixed_barplot}  
    \caption{Deuxi�me version du diagramme en barres.}
    \label{fig:georgia_fixed_barplot}
  \end{subfigure}  
  \caption{Deux variantes du m�me diagramme en barres publi�es par le
    Department of Public Health de l'�tat de G�orgie � propos du nombre de cas
    de CoVid19.}
  %\label{fig:georgia_barplot}
\end{figure}

Il est donc tr�s important de vous assurez que vos graphiques soient lisibles et qu'ils traduisent clairement votre message sans d�former les donn�es. La visualisation des donn�es, ou \textit{dataviz}, est un champ d'�tudes � part enti�re.  Nous nous contenterons ici de citer quelques principes parmi
les plus importants.


\subsection{Le choix des axes}
%https://callingbullshit.org/tools/tools_misleading_axes.html 
Le choix des �chelles et intervalles d'un graphique a une influence sur son
interpr�tation.

Pour un diagramme en barres, ne pas faire commencer les axes � 0 peut
artificiellement gonfler les diff�rences entre les diff�rentes barres. Ainsi,
le diagramme de la figure~\ref{fig:bars_start_nonzero} indique que le mod�le 4
est bien sup�rieur aux autres, tandis que celui de la
figure~\ref{fig:bars_start_zero} montre des performances tr�s comparables entre
les diff�rentes m�thodes. (Dans ce cas pr�cis, il serait de toute fa�on
souhaitable de r�p�ter plusieurs fois l'entrainement et l'�valuation, par
exemple avec une validation crois�e (que nous verrons
section~\ref{sec:crossval}) et de produire des barres d'erreurs.)
\begin{figure}[h]
  \centering
  \begin{subfigure}[t]{0.47\textwidth}
    \centering
    \includegraphics[width=\textwidth]{figures/pratiques/bars_start_nonzero}
    \caption{Axe des ordonn�es r�duit.}
    \label{fig:bars_start_nonzero}
  \end{subfigure} \hfill
  \begin{subfigure}[t]{0.47\textwidth}
    \includegraphics[width=\textwidth]{figures/pratiques/bars_start_zero}  
    \caption{Axe des ordonn�es allant de 0 � 1.}
    \label{fig:bars_start_zero}
  \end{subfigure}  
  \caption{Deux fa�ons de pr�senter la comparaison des performances de 4 mod�les.}
  %\label{fig:georgia_barplot}
\end{figure}

� l'inverse, il pourra �tre pr�f�rable pour un diagramme dont le but est non
pas de comparer les valeurs absolues de variables mais plut�t de pr�senter leur
�volution que l'axe des ordonn�es ne commence pas � z�ro. Ainsi, la figure~\ref{fig:line_start_zero} indique une temp�rature tr�s stable, tandis que la figure~\ref{fig:line_start_nonzero} permet de mieux rendre compte des variations.
\begin{figure}[h]
  \centering
  \begin{subfigure}[t]{0.47\textwidth}
    \centering
    \includegraphics[width=\textwidth]{figures/pratiques/line_start_zero}
    \caption{Axe des ordonn�es partant de 0K.}
    \label{fig:line_start_zero}
  \end{subfigure} \hfill
  \begin{subfigure}[t]{0.47\textwidth}
    \includegraphics[width=\textwidth]{figures/pratiques/line_start_nonzero}  
    \caption{Axe des ordonn�es r�duit.}
    \label{fig:line_start_nonzero}
  \end{subfigure}  
  \caption{Deux fa�ons de pr�senter l'�volution des temp�ratures moyenne de la table~\ref{tab:meteo_data}.}
  %\label{fig:georgia_barplot}
\end{figure}

\subsection{\textit{Proportional ink} ou principe de l'encre proportionnelle}
%https://callingbullshit.org/tools/tools_proportional_ink.html
De mani�re g�n�rale, il est recommand�, lorsque l'on utilise des surfaces pour
repr�senter des nombres (par exemple, les rectangles d'un diagramme en barres),
que ces surfaces soient d'aires proportionnelles aux nombres en question. On
retrouve d'ailleurs ici l'id�e de commencer les barres d'un diagramme en
barres � 0.

Il faut cependant faire aussi attention � ce que les surfaces en question soient faciles � comparer visuellement. Un diagramme camembert est ainsi pr�f�rable � un graphique � bulles ; mais un diagramme en barres est g�n�ralement plus lisible qu'un diagramme camembert. La figure~\ref{fig:areas} l'illustre. Il s'agit d'une variante d'une \href{https://www.jstor.org/stable/2288400?seq=1#metadata\_info\_tab\_contents}{exp�rience men�e au d�but des ann�es 1980} et souvent consid�r�e comme fondatrice en \textit{dataviz}.

Remarquez ici que le diagramme en barres serait encore plus lisible sans couleurs (elles n'apportent rien) et en ordonnant les cat�gories par proportion.
\begin{figure}[h]
  \centering
  \begin{subfigure}[t]{0.20\textwidth}
    \includegraphics[width=\textwidth]{figures/pratiques/areas_bubbles}  
    \caption{Graphique � bulles.}
    \label{fig:areas_bubbles}
  \end{subfigure}  \hfill
  \begin{subfigure}[t]{0.33\textwidth}
    \includegraphics[width=\textwidth]{figures/pratiques/areas_pie}  
    \caption{Diagramme camembert.}
    \label{fig:areas_pie}
  \end{subfigure} \hfill
  \begin{subfigure}[t]{0.33\textwidth}
    \centering
    \includegraphics[width=\textwidth]{figures/pratiques/areas_bars}
    \caption{Diagramme en barres.}
    \label{fig:areas_bars}
  \end{subfigure} 
  \caption{Trois fa�ons de repr�senter les proportions de 8 cat�gories. Quelle(s) repr�sentation(s) permettent de les classer ais�ment par ordre croissant ?}
  \label{fig:areas}
\end{figure}

%\subsection{Attention aux r�sum�s}
%https://python-graph-gallery.com/39-hidden-data-under-boxplot/ 


\subsection{Dyschromatopie}
Nous ne percevons pas les couleurs de la m�me fa�on. Une forte proportion de la population est atteinte d'une forme ou d'une autre de dyschromatopie, la plus fr�quente �tant la deut�ranopie (incapacit� de diff�rencier rouge et vert). 

Pour assurer une accessibilit� maximale, utilisez des �chelles de couleurs adapt�es. Il est difficile de s'adapter � \textit{toutes} les dyschromatopies ; n�anmoins le cycle par d�faut de \texttt{matplotlib} est suppos� �tre relativement adapt�. Pour des \textit{heatmaps}, favoriser les �chelles de couleur \textit{viridis} ou \textit{cividis} (voir figure~\ref{fig:pca_plot}). Des outils comme \href{https://www.color-blindness.com/coblis-color-blindness-simulator/}{CBLIS} ou \href{https://www.funkify.org}{Funkify} vous permettent de simuler diff�rentes dyschromatopies pour v�rifier la lisibilit� de vos graphiques.


Vous pouvez aussi augmenter la lisibilit� de vos graphiques en utilisant des
indices suppl�mentaires (�paisseur de trait, hachures, forme des points,
ordonner les l�gendes dans le m�me ordre que les courbes, etc.) et en doublant
vos images d'une description textuelle alternative pour les personnes
non-voyantes.


\begin{figure}[h]
  \centering
  \begin{subfigure}[t]{0.30\textwidth}
    \includegraphics[width=\textwidth]{figures/pratiques/pca_plot_magma}  
    \caption{Magma.}
    \label{fig:pca_plot_magma}
  \end{subfigure}  \hfill
  \begin{subfigure}[t]{0.30\textwidth}
    \includegraphics[width=\textwidth]{figures/pratiques/pca_plot_viridis}  
    \caption{Viridis.}
    \label{fig:pca_plot_viridis}
  \end{subfigure} \hfill
  \begin{subfigure}[t]{0.30\textwidth}
    \centering
    \includegraphics[width=\textwidth]{figures/pratiques/pca_plot_cividis}
    \caption{Cividis.}
    \label{fig:pca_plot_cividis}
  \end{subfigure} 
  \caption{Athl�tes de la PC2, repr�sent�s selon deux composantes, et color�s en fonction de leur classement, selon trois �chelles de couleur diff�rentes.}
  \label{fig:pca_plot}
\end{figure}




\section{�quit� des algorithmes}
Une question importante qui se pose constamment en science des donn�es est
celle de la \textbf{reproduction des biais}. En effet, un mod�le appris sur un
jeu de donn�es peut facilement reproduire des biais de ce jeu de donn�es,
qu'ils soient explicites ou implicites.

Un exemple qui revient souvent est celui d'un algorithme de ressources humaines
utilis� par Amazon. Le mod�le avait tendance � rejeter les candidatures pos�es
par des femmes. En effet, il �tait entra�n� sur des donn�es internes �
l'entreprise, dont les recrutements �taient fortement biais�s en faveur des
hommes. Bien que le genre n'ait pas �t� une variable utilis�e pour d�crire les
candidatures, le mod�le d�tectait dans le texte des CV des informations
corr�l�e dans le jeu d'entra�nement au rejet d'une candidature mais qui
s'av�raient surtout traduire qu'elle �tait pos�e par une femme (�ducation dans
un �tablissement non-mixte r�serv� aux femmes ; appartenance � une �quipe de
sport f�minin, etc.).

Ainsi, ce n'est pas parce qu'un mod�le statistique est purement math�matique
qu'il est impartial ; en particulier, un mod�le ne peut pas �tre de meilleure
qualit� que son jeu d'entra�nement. Il faut donc r�fl�chir � la
\textbf{repr�sentativit�} des donn�es : peut-on bien consid�rer qu'il s'agit
d'un �chantillon al�atoire de la population qui nous int�resse, o� ne
correspondent-elles qu'� une sous-population sp�cifique ?

Un autre exemple de reproduction des biais apparait dans une publication de
2016 qui pr�sente un classifieur capable de distinguer criminels de
non-criminels � partir de simples photos. Cependant, les clich�s de criminels
�taient des photos administratives prises de face, sans sourire, tandis que les
photos de non-criminels �taient des clich�s plus flatteurs : le mod�le
\href{https://callingbullshit.org/case\_studies/case\_study\_criminal\_machine\_learning.html}{d�tectait
  en fait les sourires}. On retrouve tr�s souvent ce type d'erreurs, d�es � un
\textbf{facteur confondant} : on croit arriver � s�parer des images sur leur
contenu alors qu'on utilise principalement leur luminosit� ; ou � trouver des
facteurs g�n�tiques influen�ant le niveau �conomique, alors que celui-ci est
fortement corr�l� dans les donn�es � la couleur de peau ; et ainsi de suite.

La question de l'�quit� des algorithmes est un sous-domaine important de
l'apprentissage automatique, et se pose d'autant plus que ses applications
s'�tendent � des domaines divers et vari�s touchant de nombreux aspects de nos
soci�t�s : recruement mais aussi s�curit�, sant�, justice, etc. 
C'est le sujet par exemple de l'organisation
\href{https://www.fatml.org/}{Fairness, Accountability and Transparency in
  Machine Learning}.

Pour autant, il n'y a pas actuellement (et il n'y aura vraisemblablement
jamais) d'outils ou de proc�dures permettant de garantir cette �quit�. Il est
ainsi n�cessaire de comprendre l'origine possible des biais, ainsi que de
d�velopper des outils pour les mesurer.

Ces derni�res ann�es ont cependant vu l'�mergence de labels, tels que le
\href{http://fdu-label.com/fr/index.html}{Fair Data Use} en France ou
\href{http://aequitas.dssg.io/}{Aequitas} aux USA, proposant une �valuation
�thique des outils num�riques.

\section{Fiabilit�}
Du diagnostic automatis� aux v�hicules autonomes, nous avons de plus en plus
envie de faire confiance � l'intelligence artificielle pour les opportunit�s
qu'elle pr�sente.

Mais comment faire confiance aux mod�les et algorithmes issus de la science des
donn�es ? Plusieurs questions se posent en plus de celle de l'�quit� discut�e plus haut.

\paragraph{V�rifiabilit�} les syst�mes d'IA ont-ils le comportement attendu ?
Les \href{https://fr.wikipedia.org/wiki/M\%C3\%A9thode\_formelle\_(informatique)}{m�thodes
  formelles}
typiquement utilis�es en informatique pour les programmes utilis�s en avionique
ne se pr�tent gu�re aux mod�les de l'apprentissage automatique, m�me si \href{https://formal-paris-saclay.fr/}{de
r�cents travaux �mergent sur le sujet}.

\paragraph{Explicabilit� et interpr�tabilit�} Il s'agit aussi de vastes champs
d'�tude. Si une r�gression lin�aire est relativement interpr�table (cf. PC 3),
des mod�les param�triques plus complexes tels que ceux produits par des r�seaux
de neurones artificiels (voir chapitre~\ref{chap:nonlin}) le
sont beaucoup moins. 

\paragraph{Sp�cification} La description pr�cise du comportement attendu
peut-elle aussi �tre d�licate : quel choix doit faire un v�hicule autonome
entre renverser une fillette et emboutir une moto avec deux passagers ? Le MIT
Media Lab propose par exemple \href{http://moralmachine.mit.edu/hl/fr}{La
  Machine Morale}, une plateforme permettant d'explorer divers dilemnes moraux
pos�s par la prise de d�cision de machines intelligentes.

\paragraph{Robustesse} Les mod�les sont-ils robustes aux attaques ? Depuis
2015, les exemples montrant qu'il est possible d'induire facilement en erreur
un mod�le appris par apprentissage automatique s'accumulent. Ces exemples
incluent l'ajout de bruit
\footnote{https://arxiv.org/abs/1412.6572}{ind�tectable � l'\oe{}il} ou
\href{https://nicholas.carlini.com/code/audio\_adversarial\_examples}{�
  l'oreille}, la \href{https://arxiv.org/abs/1710.08864}{modification d'un seul
  pixel} d'une image, ou
l'\href{https://towardsdatascience.com/poisoning-attacks-on-machine-learning-1ff247c254db}{empoisonnement}
d'un jeu de donn�es, qui consiste � introduire au moment de l'apprentissage un
faible nombre d'exemples mal �tiquet�s ou ing�nieusement calibr�s pour induire
un comportement ind�sirable.

De m�me qu'en cryptographie o� de nouveaux protocoles �mergent pour faire face
� de nouvelles attaques de hackers, l'apprentissage automatique progresse aussi
pour r�pondre aux attaques
adversariales. \href{http://proceedings.mlr.press/v97/simon-gabriel19a.html}{De
  r�cents travaux} montrent m�me qu'en raison du fl�au de la dimension, les
attaques adversariales sont in�vitables en grande dimension.

\paragraph{Reproductibilit�} La d�marche scientifique repose sur la
reproductibilit� des exp�riences. Au probl�me du \textit{p-hacking} abord� au
Chapitre~\ref{chap:tests} s'ajoute celui de la disponibilit� des donn�es, qui
peut �tre limit�e pour des raisons de confidentialit�, ainsi que la question
des \textbf{ressources informatiques} qui peuvent �tre n�cessaires � entra�ner
certains mod�les. Reproduire des r�sultats obtenuse en faisant tourner 800
processeurs graphiques (GPUs) pendant 3 semaines n�cessite des ressources
financi�res importantes (on rejoint ici des questions de co�t �nerg�tique et
�cologique abord�es dans la section~\ref{sec:ecology}).


\paragraph{Responsabilit�} Qui est responsable en cas de faillite d'un
syst�me d'IA : l'IA est-elle responsable ? Ou bien la personne qui l'utilise ?
Ou encore celle qui l'a construite ? La question s'est par exemple pos�e
lorsqu'un v�hicule autonome
\href{https://www.nextinpact.com/news/108432-cause-probable-accident-mortel-uber-tout-monde-en-prend-pour-son-grade.htm}{a
  fauch� une pi�tonne} en mars 2018.



\section{Confidentialit� des donn�es}
Une grande partie des donn�es utilis�es en science des donn�es sont des donn�es
personnelles, c'est-�-dire que les individus qu'elles d�crivent sont des
personnes. Nombre d'entre nous s'inqui�tent de ce que les donn�es qui nous
concernent, qu'elles soient m�dicales, de localisation g�ographique, ou
concernent notre activit� num�rique, soient utilis�es � bon escient.

Les \href{https://risques-tracage.fr/}{discussions autour des applications de
  tra�age de contacts} dans la lutte contre la propagation du coronavirus vont
actuellement bon train, illustrant cette pr�occupation.


En tant que \textit{data scientists}, comment nous assurer que nous ne
compromettons pas la confidentialit� des personnes dont nous manipulons les
donn�es ? Deux types de solutions techniques sont possibles.
\paragraph{D�-identification algorithmique} Il s'agit de s'assurer que l'on ne
puisse pas remonter des donn�es aux individus. Parmi ces techniques,
l'\textbf{anonymisation} consiste � supprimer suffisamment d'informations
identifiantes pour emp�cher la r�identification. Ces informations peuvent �tre
\textbf{directement identifiantes} s'il s'agit de caract�ristiques personnelles
uniques (nom, num�ro de s�curit� sociale, num�ro de t�l�phone, etc.) ou
\textbf{indirectement identifiantes} si elles permettent d'identifier la
personne de mani�re unique quand elles sont crois�es avec d'autres donn�es (code postal, date de naissance et lieu de travail pris ensemble
peuvent �tre indirectement identifiants).  Par contraste, la
\textbf{confidentialit� diff�rentielle}, ou \textit{differential privacy} en
anglais cherche plut�t � garantir que les r�sultats d'une analyse sur une base de
donn�es soient presque identiques qu'un �chantillon soit pr�sent ou non.

\paragraph{S�curit� des bases de donn�es} Cet aspect inclut par exemple le
chiffrement homomorphique permettant d'obtenir les m�mes r�sultats sur donn�es
chiffr�es que non chiffr�es, ne laissant ainsi aux \textit{data scientists} que
l'acc�s aux donn�es chiffr�es, des solutions de calcul distribu� s�curis�es, ou
encore du mat�riel cryptographique permettant d'ex�cuter du code sans que les
donn�es ne soient visibles.

En France, la \href{https://www.cnil.fr/}{Commission Nationale de
  l'Informatique et des Libert�s (CNIL)} encadre l'utilisation des donn�es
personnelles, qui est notamment encadr� par la loi du 14 mai 2018 transposant
le
\href{https://fr.wikipedia.org/wiki/R\%C3\%A8glement\_g\%C3\%A9n\%C3\%A9ral\_sur\_la\_protection\_des\_donn\%C3\%A9es}{R�glement
  G�n�ral sur la Protection des Donn�es (RGPD)} de l'Union Europ�enne.


\section{Enjeux �cologiques}
\label{sec:ecology}
\href{https://www.ademe.fr/sites/default/files/assets/documents/guide-pratique-face-cachee-numerique.pdf}{Selon
  l'ADEME}, le secteur du num�rique est responsable de 4\% des �missions
mondiales de gaz � effet de serre, dont un quart d�s aux data
centers. Entra�ner un r�seau de neurones artificiels avec 213 millions de
param�tres peut g�n�rer \href{https://arxiv.org/abs/1906.02243}{autant
  d'�missions de CO2 que cinq voitures am�ricaines} pendant toute leur
existence, fabrication comprise.  Le
\href{https://mlco2.github.io/impact/}{Machine Learning Emissions Calculator}
est un des outils qui accompagnent la prise de conscience de l'impact
environnemental de la science des donn�es.


\begin{plusloin}
\item Des ouvrages entiers ont �t�s �crits sur la \textit{dataviz}, par exemple \href{https://serialmentor.com/dataviz/}{\textit{Fundamentals of Data Vizualization} de Claus O. Wilke}, le travail d'\href{https://www.edwardtufte.com/tufte/}{Edward Tufte}, ou encore \href{https://informationisbeautiful.net/}{\textit{Information is Beautiful} by David McCandless}.
\item \href{https://hippocrate.tech/}{Le Serment d'Hippocrate pour Data Scientist} de Data for Good.
\item La question de la repr�sentativit� se pose dans de nombreux
  domaines de l'ing�nierie. Les exemples sont nombreux, des
  \href{https://www.huffingtonpost.fr/2017/08/19/ce-distributeur-automatique-ne-distribue-pas-de-savon-aux-mains\_a\_23152387/}{distributeurs
    de savon qui ne d�tectent que les peaux claires} � tous les
  objets plut�t adapt�s aux hommes recens�s par Caroline Criado
  Perez dans
  \href{https://www.liberation.fr/france/2020/03/06/les-femmes-invisibles-dans-un-monde-cree-pour-les-hommes\_1780895}{\textit{Invisible
      Women}}.
  \item Un �pisode de La M�thode Scientifique  intitul� \href{https://april.org/ethique-numerique-des-datas-sous-serment-emission-la-methode-scientifique}{\textit{�thique num�rique, des data sous
    serment}}.
  \item {Fairness and Machine Learning} by Solon
    Barocas, Moritz Hardt and Arvind Narayanan.
  \item � propos de \href{https://www.latribune.fr/supplement/ceux-qui-transforment-la-france/la-justice-predictive-nouvel-outil-pour-les-professionnels-du-droit-837752.html}{justice pr�dictive}, l'article \href{https://www.dalloz-actualite.fr/flash/justice-et-intelligence-artificielle-preparer-demain-episode-i#.XsvEykNS8Xc}{Justice
        et intelligence artificielle : pr�parer demain}.
  \item \href{https://hbr.org/2013/04/the-hidden-biases-in-big-data}{\textit{The Hidden Biases in Big Data}}, Kate Crawford, HBR, April 2013. 
  \item \href{https://salil.seas.harvard.edu/files/salil/files/differential_privacy_primer_nontechnical_audience.pdf}{\textit{Differential privacy: A primer for a non-technical audience}}, A. Wood et al., Vanderbilt Journal of Entertainment and Technology Law.
\end{plusloin}

%%% Local Variables:
%%% mode: latex
%%% TeX-master: "sdd_2020_poly"
%%% End:



\part{Apprentissage supervis�}
\chapter{Minimisation du risque empirique}
%-*- coding: iso-latin-1 -*-
\label{chap:erm}

\paragraph{Notions :} classification, r�gression, espace des hypoth�ses,
minimisation du risque empirique, moindres carr�s, mod�les param�triques
lin�aires
\paragraph{Objectifs p�dagogiques :} 
\begin{itemize}      
  \setlength{\itemsep}{3pt}
\item Formaliser un probl�me d'apprentissage supervis�.
\item D�crire l'espace des hypoth�ses dans le cas d'un mod�le param�trique.
\item  $\bigstar$ Prouver l'�quivalence entre maximisation de la vraisemblance et
  minimisation du risque empirique dans le cas gaussien.
\item Mettre en \oe{}uvre une r�gression lin�aire.
\end{itemize}



Nous nous int�ressons maintenant aux probl�mes d'apprentissage \textbf{supervis�}
: il s'agit de d�velopper des algorithmes qui soient capables d'apprendre des
mod�les \textbf{pr�dictifs}. � partir d'exemples �tiquet�s, ces mod�les seront
capables de pr�dire l'�tiquette de nouveaux objets. Le but de ce chapitre est
de d�velopper les concepts g�n�raux qui nous permettent de formaliser ce type
de probl�mes.



% \paragraph{Comp�tences}


\section{Formalisation d'un probl�me d'apprentissage supervis�}
\label{sec:sup_learn}
% Un probl�me d'\textbf{apprentissage supervis�} peut �tre formalis� de la fa�on
% suivante : �tant donn�es $n$ \textbf{observations}
% $\{\xx^1, \xx^2, \dots, \xx^n\}$, o� chaque observation $\xx^i$ est un �l�ment
% de l'espace des observations $\XX$, et leurs \textbf{�tiquettes}
% $\{y^1, y^2, \dots, y^n\}$, o� chaque �tiquette $y^i$ appartient � l'espace des
% �tiquettes $\YY$, le but de l'apprentissage supervis� est de trouver une
% fonction $f: \XX \rightarrow \YY$ telle que $f(\xx) \approx y,$ pour toutes les
% paires $(\xx, y) \in \XX \times \YY$ ayant la m�me relation que les paires
% observ�es. L'ensemble de $\DD = \{(\xx^i, y^i)\}_{i=1, \dots, n}$ forme le
% \textbf{jeu d'apprentissage}.

%\subsection{Vocabulaire}
Nous supposons maintenant disposer non seulement d'une matrice
$X \in \RR^{n \times p}$ d�crivant $n$ individus en $p$ dimensions, mais aussi
de $n$ \textbf{�tiquettes} $\{y^1, y^2, \dots, y^n\}$. Chaque �tiquette $y^i$
appartient � un espace $\YY.$ Dans ce cours, nous allons consid�rer deuxcas
particuliers pour $\YY:$
\begin{itemize}
\item $\YY = \RR :$ on parle d'un probl�me de \textbf{r�gression} ;
\item $\YY = \{0, 1\} :$ on parle d'un probl�me de \textbf{classification
    binaire}, et les observations dont l'�tiquette vaut $0$ sont appel�es
  \textbf{n�gatives} tandis que celles dont l'�tiquette vaut $1$ sont appel�es
  \textbf{positives}. Dans certains cas, il sera math�matiquement plus simple
  d'utiliser $\YY = \{-1, 1\}$.
\end{itemize}

La matrice $X \in \RR^{n \times p}$ telle que $X_{ij} = x^i_j$ soit la $j$-�me
variable du $i$-�me individu est appel�e \textbf{matrice de donn�es} ou
\textbf{matrice de design}. 

On peut aussi choisir de repr�senter chaque individu et son �tiquette par le
couple $(\xx^i, y^i) \in \RR^{p} \times \YY.$ L'ensemble
$\DD = \{(\xx^i, y^i)\}_{i=1, \dots, n}$ forme alors le \textbf{jeu d'apprentissage}.

Le machine learning �tant issu de plusieurs disciplines et champs
d'applications, on trouvera plusieurs noms pour les m�mes objets.  Ainsi les
variables sont aussi appel�es \textbf{descripteurs}, \textbf{attributs},
\textbf{pr�dicteurs}, ou \textbf{caract�ristiques} (en anglais,
\textit{variables, descriptors, attributes, predictors} ou encore
\textit{features}).  Les \textbf{individus}, ou \textbf{observations} sont
aussi appel�es \textbf{exemples}, \textbf{�chantillons} ou \textbf{points du
  jeu de donn�es} (en anglais, \textit{samples} ou \textit{data
  points}). Enfin, les �tiquettes sont aussi appel�es \textbf{variables cibles}
(en anglais, \textit{labels, targets} ou \textit{outcomes}).

% Ces concepts sont illustr�s sur la figure~\ref{fig:suplearning}.

% \begin{figure}[h]
%   \centering
%   \includegraphics[width=0.7\textwidth]{figures/erm/suplearning}
%   \caption{Les donn�es d'un probl�me d'apprentissage supervis� sont organis�es
%     en une matrice de design et un vecteur d'�tiquettes. Les observations sont
%     repr�sent�es par leurs variables explicatives.}
%   \label{fig:suplearning}
% \end{figure}

Le but de l'apprentissage supervis� est alors de trouver une fonction
$f: \RR^p \rightarrow \YY$ telle que $f(\xx) \approx y,$ qui s'applique non
seulement aux $n$ individus observ�s, mais plus g�n�ralement � tous les
individus d'une population � laquelle on suppose que ces $n$ individus
appartiennent. C'est cette fonction $f$ qui est le \textbf{mod�le pr�dictif}
appris. Un \textbf{algorithme d'apprentissage supervis�} utilise le jeu de
donn�es $\DD$ donn�es pour d�terminer $f$.


Plus formellement, supposons que les couples $(\xx^i, y^i)$ soient les
r�alisations de $n$ vecteurs al�atoires de m�me loi qu'un couple de variables
al�atoire $(X, Y)$, $X$ �tant un vecteur al�atoire $p$-dimensionnel et $Y$ une
variable al�atoire r�elle � valeurs dans $\YY$. Supposons de plus qu'il existe
une fonction $\Phi : \RR^p \rightarrow \YY$ et une variable al�atoire r�elle
$\epsilon$ telle que
  \begin{equation}
  Y = \Phi(X) + \epsilon,
  \label{eq:probabilistic_ml}
\end{equation}
$\epsilon$ repr�sentant un \textbf{bruit}.
Ce bruit peut �tre caus�
\begin{itemize}
\item par des {\it erreurs de mesure} dues � la faillibilit� des capteurs
  utilis�s pour mesurer les variables par lesquelles on repr�sente nos
  donn�es, ou � la faillibilit� des op�rateurs humains qui ont entr� ces
  mesures dans une base de donn�es ;
\item par des {\it erreurs d'�tiquetage} (souvent appel�s {\it teacher's noise}
  en anglais) dues � la faillibilit� des op�rateurs humains qui ont �tiquet�
  les donn�es ;
\item enfin, parce que les variables mesur�es ne suffisent pas � mod�liser le
  ph�nom�ne qui nous int�resse, soit qu'on ne les connaisse pas, soit
  qu'elles soient co�teuses � mesurer.
\end{itemize}
Notre but est d'approcher $\Phi$ par $f$.

Dans le cas d'un probl�me de classification, le mod�le pr�dictif peut prendre
directement la forme d'une fonction $f$ � valeurs dans $\{0, 1\}$, ou utiliser
une fonction interm�diaire $g$ � valeurs r�elles, qui associe � une observation
un score d'autant plus �lev� qu'elle est susceptible d'�tre positive. Ce score
peut par exemple �tre la probabilit� que cette observation appartienne � la
classe positive. On obtient alors $f$ en \textbf{seuillant} $g$ ; $g$ est
appel�e \textbf{fonction de d�cision} \footnote{Dans la librairie
  \texttt{scikit-learn}, on fera ainsi attention � la distinction entre les
  m�thodes \texttt{predict} et \texttt{predict\_proba}.}.

\begin{exemple}
  \paragraph{Filtrage de spam.} On peut poser le filtrage de spam comme un
  probl�me de classification binaire. Les individus sont des emails. Leur
  �tiquette est binaire (positive pour � spam � et n�gative pour � non-spam
  �). Les $p$ variables repr�sentant un email peuvent �tre d�finies comme le
  nombre d'occurrences, pour $p$ mots, de chacun de ces mots dans l'email ($p$
  est ainsi la taille d'un dictionnaire pr�-d�fini)\footnote{C'est ce qu'on
    appelle une repr�sentation \textit{bag-of-words}}. �tant donn� un jeu de
  donn�es de $n$ emails �tiquet�s, un algorithme d'apprentissage retourne une
  fonction $f$ qui, � tout email repr�sent� par un vecteur de $\RR^p$ (en fait,
  $\NN^p$), associe une �tiquette $0$ ou $1$. Ce mod�le
  $f: \RR^p \rightarrow \{0, 1\}$ peut �tre obtenu en seuillant une fonction de
  d�cision $g: \RR^p \rightarrow \RR$.

  Le bruit peut �tre d� aux causes suivantes :
  \begin{itemize}
  \item Des erreurs de mesures peuvent �tre caus�es par des fautes
    d'orthographe (volontaires ou non) qui emp�chent de comptabiliser certains
    mots.
  \item Des erreurs d'�tiquetage peuvent arriver quand une personne marque par
    erreur comme courrier ind�sirable un email qui ne l'�tait pas, ou,
    inversement, laisse dans sa bo�te mail ou supprime sans �tiqueter comme tel
    un email ind�sirable.
  \item Enfin, notre repr�sentation est limit�e, en particulier parce qu'elle
    ne prend pas en compte l'ordre des mots. Nous ne disposons pas de
    suffisamment d'information pour classifier les emails aussi efficacement
    qu'un humain.
  \end{itemize}
  % \paragraph{Nombre de v�los � une borne velib.} On peut poser la pr�diction du
  % nombre de v�los retir�s d'une borne velib � un instant donn� comme un
  % probl�me de r�gression. Les individus sont des triplets (borne v�lib, date,
  % tranche horaire). Ils peuvent �tre repr�sent�es par un nombre $p$ de
  % variables, incluant la position g�ographique de la borne, si la date est en
  % semaine ou non, si la tranche horaire est de jour ou de nuit, etc. �tant
  % donn� un jeu de $n$ bornes ainsi �tiquet�es, un algorithme
  % d'apprentissage retourne une fonction $f: \RR^p \rightarrow \RR$ qui,
  % � toute borne velib, associe le nombre de v�los y �tant retir�s.
\end{exemple}

% Dans le cadre d'un probl�me de classification binaire, on appelle {\it fonction
%   de d�cision}, ou {\it fonction discriminante}, une fonction
% $g:\XX \mapsto \RR$ telle que $f(\xx) = 0$ si et seulement si $g(\xx) \leq 0$
% et $f(\xx) = 1$ si et seulement si
% $g(\xx) > 0$. \\

% % Cette d�finition se g�n�ralise dans le cas de la classification {\it
% %   multi-classe} : on a alors $C$ fonctions de d�cision $g_c:\XX \mapsto \RR$
% % telles que $f(\xx) = \argmax_{c = 1, \dots, C} g_c(\xx).$

% Le concept de fonction de d�cision permet de partitionner l'espace en {\it
%   r�gions de d�cision} : \\
% Dans le cas d'un probl�me de classification binaire, la fonction discriminante
% partitionne l'espace des observations $\XX$ en deux {\it r�gions de d�cision},
% $\Rcal_0$ et $\Rcal_1$, telles que
% \begin{equation*}
%   \Rcal_0 = \{\xx \in \XX | g(\xx) \leq 0\} \text{ et }
%   \Rcal_1 = \{\xx \in \XX | g(\xx) > 0\}.
% \end{equation*}
% % Dans le cas multi-classe, on a alors $C$ r�gions de d�cision
% % \begin{equation*}
% %   \Rcal_c = \{\xx \in \XX | g_c(\xx) = \max_k g_k(\xx) \}.
% % \end{equation*}

% Les r�gions de d�cision sont s�par�es par des {\it fronti�res de d�cision} : \\
% Dans le cadre d'un probl�me de classification, on appelle {\it fronti�re de
%   d�cision}, ou {\it discriminant}, l'ensemble des points de $\XX$ o� une
% fonction de d�cision s'annule.  Dans le cas d'un probl�me binaire, il y a
% une seule fronti�re de d�cision ; dans le cas d'un probl�me multi-classe �
% $C$ classes, il y en a $C$.




\paragraph{Remarque.} Les notions d�velopp�es jusqu'� la fin de la
section~\ref{sec:losses} peuvent l'�tre en rempla�ant $\RR^p$ par un espace
quelconque $\XX$.


\section{Espace des hypoth�ses}
Pour poser un probl�me d'apprentissage supervis�, il nous faut d�cider du
type de mod�les que nous allons consid�rer. 

On appelle \textbf{espace des hypoth�ses} l'espace de fonctions $\FF$, qui est
un sous-espace de toutes les fonctions de $\RR^p \rightarrow \YY$ d�crivant les
mod�les que nous allons consid�rer. Cet espace est choisi en
fonction de nos {\it convictions} par rapport au probl�me, ainsi que de
consid�rations pratiques sur notre capacit� � trouver facilement un � bon �
mod�le dans $\FF$.

Le choix de l'espace des hypoth�ses est fondamental.  En effet, si cet espace
ne contient pas le \og bon \fg~mod�le, % , par exemple si l'on choisit comme
% espace des hypoth�ses pour les donn�es de la figure~\ref{fig:simple_classif_pb}
% l'ensemble des droites,
il sera impossible de trouver une bonne fonction de
d�cision.  Cependant, si l'espace est trop g�n�rique, il sera plus difficile et
intensif en temps de calcul d'y trouver une bonne fonction de mod�lisation.
  

\begin{exemple}
  Dans l'exemple de la figure~\ref{fig:simple_classif_pb}, on pourra d�cider de
  se restreindre � des discriminants qui soient des ellipses � axes parall�les
  aux axes de coordonn�es.  Ainsi, l'espace des hypoth�ses sera
  \begin{equation}
    \FF = \{ \xx \mapsto \alpha (x_1-a)^2 + \beta (x_2-b)^2 - 1 \; ; (\alpha, \beta, a, b) \in \RR^4\}.
    \label{eq:hypothesis_space_ellipsis}
  \end{equation}

  Dans cet espace, il semble possible de trouver un mod�le $f$ qui s�pare les
  positifs des n�gatifs. Si nous avions choisi comme espace des hypoth�ses
  l'ensemble des fonctions lin�aires de $\RR^2$ dans $\RR$, ce ne serait pas
  possible.
\end{exemple}
\begin{figure}[h]
  \centering
  \includegraphics[width=0.45\textwidth]{figures/erm/simple_classif}
  \caption{Les exemples positifs (+) et n�gatifs (x) semblent �tre s�parables
    par une ellipse.}
  \label{fig:simple_classif_pb}
\end{figure}

La t�che d'apprentissage supervis� consiste � d�terminer une hypoth�se
$f \in \FF$ qui approche au mieux la fonction cible $\phi$ (voir
�quation~\eqref{eq:probabilistic_ml}). Pour r�aliser une telle t�che, nous
allons d�velopper dans les sections suivantes deux outils suppl�mentaires :
\begin{enumerate}
\item Une fa�on de \textbf{quantifier la qualit� d'une hypoth�se}, afin de
  pouvoir d�terminer si une hypoth�se satisfaisante (voire optimale) a �t�
  trouv�e.  Pour cela, nous allons d�finir la notion de \textbf{fonction de
    co�t}.
\item Une fa�on de \textbf{chercher une hypoth�se optimale} dans $\FF$.  Les
  algorithmes d'apprentissage supervis� que nous allons �tudier ont pour but de
  trouver dans $\FF$ l'hypoth�se optimale au sens de la fonction de
  co�t. Diff�rents algorithmes correspondent � diff�rents $\FF$, et selon les
  cas cette recherche sera exacte ou approch�e.
\end{enumerate}

\section{Minimisation du risque empirique}
\label{sec:mre}
R�soudre un probl�me d'apprentissage supervis� revient � trouver une fonction
$f \in \FF$ dont les pr�dictions soient les plus proches possibles des
v�ritables �tiquettes, sur tout l'espace $\RR^p$. On utilise pour formaliser cela
la notion de \textbf{fonction de co�t} :

Une \textbf{fonction de co�t} $L: \YY \times \YY \rightarrow \RR$, 
aussi appel�e \textbf{fonction de perte} ou \textbf{fonction d'erreur}
(en anglais : {\it cost function} ou {\it loss function})
est une fonction utilis�e pour quantifier la qualit� d'une pr�diction : 
$L(y, f(\xx))$ est d'autant plus grande que l'�tiquette $f(\xx)$ est �loign�e de
la vraie valeur $y$.

�tant donn�e une fonction de co�t $L$, nous cherchons donc $f$ qui minimise ce
co�t sur l'ensemble des valeurs possibles de $\xx \in \RR^p$, ce qui est
formalis� par la notion de \textbf{risque.} Nous supposons que les couples
$(\xx^i, y^i)$ sont les r�alisations de $n$ vecteurs al�atoires de m�me loi
qu'un couple de variables al�atoire $(X, Y).$

Dans le cadre d'un probl�me d'apprentissage supervis�, on appelle
\textbf{risque} d'un mod�le $h$ l'esp�rance de son co�t :
\begin{equation}
  \label{eq:risque}
  \Rcal(h) = \EE(L(h(X), Y)).
\end{equation}

Nous cherchons donc un mod�le $f$ tel que 
\begin{equation}
  \label{eq:risk_minimization}
  f \in \argmin_{h \in \FF} \EE(L(h(X), Y)).
\end{equation}
Ce probl�me est g�n�ralement insoluble sans plus d'hypoth�ses : nous ne
connaissons que $n$ r�alisations du couple $(X, Y)$.  On approchera donc le
risque par son estimation sur ces r�alisations.

On appelle \textbf{risque empirique} de $h$ l'estim�e du risque de $h$ d�fini par
\begin{equation}
  \label{eq:empirical_risk}
  R_n(h) = \frac{1}{n} \sum_{i=1}^n L(h(\xx^i), y^i).
\end{equation}

On appelle donc mod�le obtenu par \textbf{minimisation du risque empirique} une
fonction
\begin{equation}
  \label{eq:erm}
  f \in \argmin_{h \in \FF} \frac{1}{n} \sum_{i=1}^n L(h(\xx^i), y^i).
\end{equation}

Selon le choix de $\FF$ et $L$, l'�quation~\ref{eq:erm} peut avoir une solution
analytique explicite. Cela ne sera pas souvent le cas ; cependant on choisira
souvent une fonction de co�t convexe afin de r�soudre plus facilement ce
probl�me d'optimisation.

La minimisation du risque empirique est g�n�ralement un probl�me {\it mal pos�}
au sens de Hadamard, c'est-�-dire qu'il n'admet pas une solution unique
d�pendant de fa�on continue des conditions initiales. Il se peut par exemple
qu'un nombre infini de solutions minimise le risque empirique � z�ro (voir
figure~\ref{fig:multiple_solutions}).

\begin{figure}[h]
  \centering
  \includegraphics[width=0.4\textwidth]{figures/erm/multiple_solutions}
  \caption{Une infinit� de droites s�parent parfaitement les points positifs
    (+) des points n�gatifs (x). Chacune d'entre elles a un risque empirique
    nul.}
  \label{fig:multiple_solutions}
\end{figure}


\paragraph{Convergence} La loi des grands nombres nous garantit que le risque
empiruqe d'un mod�le $h \in \FF$ converge vers le risque quand la taille de
l'�chantillon tend vers l'infini :
\begin{equation}
  \label{eq:risk_cvg}
  R_n(h) \xrightarrow[n \rightarrow \infty]{} \Rcal(h).
\end{equation}
Cela ne suffit cependant pas � garantir que le minimum du risque empirique
$\min_{h \in \FF} R_n(h)$ converge vers le minimum du risque
$\min_{h \in \FF} \Rcal(h)$. En effet, si $\FF$ est l'espace des fonctions
mesurables, le minimiseur de $R_n(h)$ vaut g�n�ralement $0$, ce qui n'est pas
le cas de $\Rcal(h).$ \textbf{Il n'y a donc aucune garantie qu'un mod�le qui
  minimise le risque empirique minimise le risque.} C'est une remarque tr�s
importante car elle signifie que le fait qu'un mod�le minimise l'erreur sur nos
$n$ observations ne donne aucune garantie quant � sa performance sur d'autres
individus. Nous reviendrons sur ce sujet lors du prochain chapitre, en abordant
les notions de g�n�ralisation et de surapprentissage.

La convergence de la minimisation du risque empirique d�pend de $\FF$. L'�tude
de cette convergence est l'un des principaux �l�ments de la th�orie de
l'apprentissage de Vapnik-Chervonenkis, qui d�passe largement le cadre de ce
cours.


\section{Fonctions de co�t}
\label{sec:losses}
Il existe de nombreuses fonctions de co�t. Le choix d'une fonction de co�t
d�pend d'une part du probl�me en lui-m�me, autrement dit de ce que l'on trouve
pertinent pour le cas pratique consid�r�, et d'autre part de consid�rations
pratiques : peut-on ensuite r�soudre le probl�me d'optimisation qui r�sulte de
ce choix de fa�on suffisamment exacte et rapide~?%  Cette section pr�sente les
% fonctions de co�ts les plus couramment utilis�es et on pourra s'y r�f�rer tout
% au long de la lecture de cet ouvrage.
Cette section pr�sente quelques-unes des fonctions de co�t les plus utilis�es.


% \subsection{Fonctions de co�t pour la classification binaire}
% Pour d�finir des fonctions de co�t pour la classification binaire, on
% consid�rera souvent $\YY = \{-1, 1\}$. En effet, dans le cas d'une
% classification parfaite, le produit $y f(\xx)$ est alors �gal � $1$.

\subsection{Co�t 0/1 pour la classification binaire}
Dans le cas d'une fonction $f$ � valeurs binaires, on appelle \textbf{fonction de
  co�t 0/1}, ou {\it 0/1 loss}, la fonction suivante :
\begin{align*}
  L_{0/1} : \YY \times \YY & \rightarrow \RR \\
  y, f(\xx) & \mapsto
              \begin{cases}
                1 & \mbox{ si } f(\xx) \neq y \\
                0 & \mbox{ sinon.}
              \end{cases}
\end{align*}

Le risque empirique d'un mod�le $h$ sur un jeu de donn�es est alors le nombre
d'erreurs de pr�diction sur ce jeu de donn�es.

% En utilisant $\YY = \{-1, 1\}$, on peut la r��crire de la mani�re suivante :
% \begin{equation*}
%   L_{0/1}(y, f(\xx)) = \frac{1 - y f(\xx)}{2}.
% \end{equation*}

% Si l'on consid�re pour $f$ une fonction de d�cision (� valeurs r�elles) plut�t
% qu'une fonction de pr�diction � valeurs binaires, on peut d�finir la fonction
% de co�t 0/1 comme suit :
% \subsubsection{Co�t 0/1 pour la r�gression}
% Quand on consid�re une fonction de d�cision � valeurs r�elles, on appelle
% {\it fonction de co�t 0/1}, ou {\it 0/1 loss}, la fonction suivante :
% \begin{align*}
%   L_{0/1} : \YY \times \RR & \rightarrow \RR \\
%   y, f(\xx) & \mapsto
%               \begin{cases}
%                 1 & \mbox{ si } y f(\xx) \leq 0 \\
%                 0 & \mbox{ sinon.}
%               \end{cases}
% \end{align*}

% L'inconv�nient de cette fonction de co�t est qu'elle n'est pas d�rivable, ce
% qui compliquera les probl�mes d'optimisation l'utilisant. De plus, elle n'est
% pas tr�s fine : l'erreur est la m�me que $f(\xx)$ soit tr�s proche ou tr�s loin
% du seuil de d�cision.  Rappelons que pour une classification parfaite, quand
% $\YY = \{-1, 1\}$, $y f(\xx) = 1$. On peut ainsi d�finir une fonction de co�t
% qui soit d'autant plus grande que $y f(\xx)$ s'�loigne de $1$ � gauche ; on
% consid�re qu'il n'y a pas d'erreur si $y f(\xx) > 1$. Cela conduit � la
% d�finition d'erreur {\it hinge}, ainsi appel�e car elle forme un coude, ou une
% charni�re (cf. figure~\ref{fig:classif_losses}).

% \subsubsection{Erreur hinge}
% \label{sec:hinge-loss}
% On appelle {\it fonction d'erreur hinge}, ou {\it hinge loss}, la fonction
% suivante :
% \begin{align*}
%   L_{\text{hinge}} : \{-1, 1\} \times \RR & \rightarrow \RR \\
%   y, f(\xx) & \mapsto
%               \begin{cases}
%                 0 & \mbox{ si } y f(\xx) \geq 1 \\
%                 1 - y f(\xx) & \mbox{ sinon.}
%               \end{cases}
% \end{align*}
% De mani�re plus compacte, l'erreur hinge peut aussi s'�crire 
% \begin{equation*}
%   \label{eq:hinge-loss}
%   L_{\text{hinge}}(f(\xx), y) = \max\left(0, 1 - y f(\xx) \right) = \left[ 1 - y f(\xx)\right]_+. 
% \end{equation*}

% On peut aussi consid�rer que $f(\xx)$ doit �tre la plus proche possible de $1$
% pour les observations positives (et $-1$ pour les observations
% n�gatives). Ainsi, on p�nalisera aussi les cas o� $y f(\xx)$ s'�loigne de $1$
% par la droite, ce que l'on peut faire avec le {\it co�t quadratique}.

% \subsubsection{Co�t quadratique pour la classification binaire}
% Dans le cadre d'un probl�me de classification binaire, on appelle {\it co�t
%   quadratique}, ou {\it square loss}, la fonction suivante :
% \begin{align*}
%   L_{\text{square}} : \{-1, 1\} \times \RR & \rightarrow \RR \\
%   y, f(\xx) & \mapsto  \left( 1 -y f(\xx)\right)^2.
% \end{align*}

% Enfin, on peut chercher � d�finir une fonction de d�cision dont la valeur
% absolue quantifie notre confiance en sa pr�diction. On cherche alors � ce que
% $y f(\xx)$ soit la plus grande possible, et on utilise le {\it co�t
%   logistique}.

\subsection{Co�t logistique et entropie crois�e}
Consid�rons maintenant que $f$ est une fonction de d�cision � valeurs r�elles.
\label{sec:logistic_loss}
On appelle \textbf{fonction de co�t logistique}, ou {\it logistic loss}, la
fonction suivante :
\begin{align} 
  \nonumber
  L_{\log} : \{-1, 1\} \times \RR & \rightarrow \RR \\ 
  y, f(\xx) & \mapsto \ln \left( 1 + \exp(-y f(\xx))\right). 
  \label{eq:logistic_loss}
\end{align}

Si $f$ est � valeurs dans $]0, 1[$, en particulier si $f(\xx)$ est la
probabilit� que $\xx$ appartienne � la classe positive, cette fonction de co�t
est �quivalente � l'\textbf{entropie crois�e}, d�finie pour $\YY = \{0, 1\}$.

\label{sec:cross_entropy}
Dans le cas binaire, on appelle \textbf{entropie crois�e}, ou {\it cross-entropy},
la fonction suivante :
\begin{align} \nonumber
  L_H : \{0, 1\} \times ]0, 1[ & \rightarrow \RR \\ 
  y, f(\xx) & \mapsto - y \ln f(\xx) - (1-y) \ln(1-f(\xx)).
              \label{eq:cross_entropy}
\end{align}
La figure~\ref{fig:logistic_loss} illustre la valeur de la fonction de co�t
logistique en fonction de l'�tiquette $y$ de l'individu $\xx$ et de la valeur
de la fonction de d�cision $f(\xx).$

\begin{figure}[h]
  \centering
  \begin{subfigure}[t]{0.43\textwidth}
    \centering
    \includegraphics[width=\textwidth]{figures/erm/logistic_loss_neg}
    \caption{Fonction de co�t logistique pour un individu d'�tiquette n�gative, en
      fonction de la valeur de la fonction de d�cision. Cette perte est
      d'autant plus grande que la fonction de d�cision est proche de $1$.}
    \label{fig:logistic_loss_neg}
  \end{subfigure} \hfill
  \begin{subfigure}[t]{0.43\textwidth}
    \includegraphics[width=\textwidth]{figures/erm/logistic_loss_pos}  
    \caption{Fonction de co�t logistique pour un individu d'�tiquette positive, en
      fonction de la valeur de la fonction de d�cision. Cette perte est
      d'autant plus grande que la fonction de d�cision est proche de $0$.}
    \label{fig:logistic_loss_pos}
  \end{subfigure}
  \caption{Perte logistique / entropie crois�e.}
  \label{fig:logistic_loss}

  \caption{Valeur de l'entropie crois�e en fonction de la valeur de la fonction de d�cision.}
  \label{fig:logistic_loss}
\end{figure}


\paragraph{$\bigstar$ Pourquoi � entropie crois�e � ?} 
L'entropie crois�e est issue de la th�orie de l'information, d'o� son nom. En
consid�rant que la v�ritable classe de $\xx$ est mod�lis�e par une distribution
$Q$, et sa classe pr�dite par une distribution $P$, nous allons chercher �
mod�liser $P$ de sorte qu'elle soit la plus proche possible de $Q$. On utilise
pour cela la divergence de Kullback-Leibler, d�finie par :
\begin{align*}
  \text{KL}(Q||P) & = \sum_{c=0, 1} Q(y=c|\xx) \ln \frac{Q(y=c|\xx)}{P(y=c|\xx)} \\
           & = - \sum_{c=0, 1} Q(y=c|\xx) \ln P(y=c|\xx) + 
             \sum_{c=0, 1} Q(y=c|\xx) \ln Q(y=c|\xx).
\end{align*}
Comme $Q(y=c|\xx)$ vaut soit $0$ ($c$ n'est pas la classe de $\xx$) soit
$1$ (dans le cas contraire), le deuxi�me terme de cette expression est nul
et on retrouve ainsi la d�finition ci-dessus de l'entropie crois�e.


% Les fonctions de perte pour la classification binaire sont illustr�es sur la
% figure~\ref{fig:classif_losses}.
% \begin{figure}[h]
%   \centering
%   \includegraphics[width=0.7\textwidth]{figures/erm/classif_losses}
%   \caption{Fonctions de perte pour la classification binaire.}
%   \label{fig:classif_losses}
% \end{figure}

% \subsection{Co�ts pour la r�gression}
% Dans le cas d'un probl�me de r�gression, nous consid�rons maintenant
% $\YY=\RR.$ Le but de notre fonction de co�t est de p�naliser les fonctions de
% pr�diction $f$ dont la valeur est �loign�e de la valeur cible $\xx$.
\subsection{Co�t quadratique pour la r�gression}
\label{sec:quadratic_loss}
On appelle \textbf{fonction de co�t quadratique}, ou {\it quadratic loss}, ou
encore {\it squared error}, la fonction suivante :
\begin{align} \nonumber
  L_{\text{SE}} : \RR \times \RR & \rightarrow \RR \\ 
  y, f(\xx) & \mapsto \frac{1}{2} \left(y - f(\xx)\right)^2. \label{eq:quadratic_loss}  
\end{align}
Le coefficient $\frac{1}{2}$ permet d'�viter d'avoir des coefficients
multiplicateurs quand on d�rive le risque empirique pour le minimiser.

% \subsubsection{Co�t $\epsilon$-insensible}
% \label{sec:epsilon_insensitive}
% Le co�t quadratique a tendance � �tre domin� par les valeurs aberrantes : d�s
% que quelques observations dans le jeu de donn�es ont une pr�diction tr�s
% �loign�e de leur �tiquette r�elle, la qualit� de la pr�diction sur les autres
% observations importe peu. On peut ainsi lui pr�f�rer le {\it co�t absolu} : \\

% On appelle {\it fonction de co�t absolu}, ou {\it absolute error}, la fonction
% suivante :
% \begin{align*}
%   L_{\text{AE}} : \RR \times \RR & \rightarrow \RR \\
%   y, f(\xx) & \mapsto  |y - f(\xx)|.
% \end{align*}

% Avec cette fonction de co�t, m�me les pr�dictions tr�s proches de la v�ritable
% �tiquette sont p�nalis�es (m�me si elles le sont faiblement). Cependant, il est
% num�riquement quasiment impossible d'avoir une pr�diction exacte. Le co�t {\it
%   $\epsilon$-insensible} permet de rem�dier � cette limitation.

% �tant donn� $\epsilon > 0$, on appelle {\it fonction de co�t
%   $\epsilon$-insensible}, ou {\it $\epsilon$-insensitive loss}, la fonction
% suivante :
% \begin{align*}
%   L_\epsilon : \RR \times \RR & \rightarrow \RR \\
%   y, f(\xx) & \mapsto \max \left(0, |y - f(\xx)| - \epsilon\right).
% \end{align*}
    
% \subsubsection{Co�t de Huber}
% Le co�t $\epsilon$-insensible n'est d�rivable ni en $-\epsilon$ ni en
% $+\epsilon$, ce qui complique l'optimisation du risque empirique.  La {\it
%   fonction de co�t de Huber} permet d'�tablir un bon compromis entre le co�t
% quadratique (d�rivable en $0$) et le co�t absolu (qui n'explose pas dans les
% valeurs extr�mes).

% On appelle {\it fonction de co�t de Huber}, ou {\it Huber loss}, la fonction
% suivante :
% \begin{align*}
%   L_{\text{Huber}} : \RR \times \RR & \rightarrow \RR \\
%   y, f(\xx) & \mapsto
%               \begin{cases}
%                 \frac{1}{2} \left(y - f(\xx)\right)^2 & \text{ si } |y - f(\xx)| < \epsilon \\
%                 \epsilon |y - f(\xx)| - \frac{1}{2} \epsilon^2 & \text{ sinon.} 
%               \end{cases}
% \end{align*}
% Le terme $- \frac{1}{2} \epsilon^2$ permet d'assurer la continuit� de la fonction.\\

% Les fonctions de co�t pour la r�gression sont illustr�es sur la
% figure~\ref{fig:regression_losses}.

% \begin{figure}[h]
%   \centering
%   \includegraphics[width=0.7\textwidth]{figures/erm/regression_losses}
%   \caption{Fonctions de co�t pour un probl�me de r�gression.}
%   \label{fig:regression_losses}
% \end{figure}



\section{Apprentissage supervis� d'un mod�le param�trique}
\label{sec:parametric}
\subsection{Mod�les param�triques}
On parle de \textbf{mod�le param�trique} quand l'espace des hypoth�ses $\FF$
est un ensemble de fonctions d�finies par une expression analytique
param�tris�e par un nombre fini de param�tres. 

C'est le cas de l'espace des hypoth�ses d�fini plus haut par
l'�quation~\eqref{eq:hypothesis_space_ellipsis} : les param�tres sont au nombre
de $4$ et il s'agit de $\alpha$, $\beta$, $a$, et $b$. Le but de
l'apprentissage sera de d�terminer les valeurs de ces param�tres.

% on utilise un algorithme
% d'apprentissage dont le but est de trouver les valeurs optimales des param�tres
% d'un mod�le dont on a d�fini la forme analytique en fonction des descripteurs.

% La complexit� d'un mod�le param�trique grandit avec le nombre de param�tres �
% apprendre, autrement dit avec le nombre de variables. � l'inverse, la
% complexit� d'un mod�le non param�trique aura tendance � grandir avec le nombre
% d'observations.

% Par exemple, un algorithme d'apprentissage qui permet d'apprendre les
% coefficient $\alpha$, $\beta$, $\gamma$ dans la fonction de d�cision suivante :
% $f: \xx \mapsto \alpha x_1 + \beta x_2x_4^2 + \gamma e^{x_3-x_5}$ apprend un
% mod�le param�trique. Quel que soit le nombre d'observations, ce mod�le ne
% change pas.

� l'inverse, la m�thode du plus proche voisin, qui associe � $\xx$ l'�tiquette
du point du jeu d'entra�nement dont il est le plus proche en distance
euclidienne, apprend un mod�le non param�trique : on ne sait pas �crire la
fonction de d�cision comme une fonction des variables pr�dictives. % Plus il y a
% d'observations, plus le mod�le pourra apprendre une fronti�re de d�cision
% complexe.

Nous verrons au chapitre~\ref{chap:nonlin} des exemples de mod�les non
param�triques, et nous concentrons maintenant sur les mod�les de r�gression
param�triques. 

Nous consid�rons pour la suite de ce chapitre disposer d'un jeu
$\DD = \{\xx^i, y^i\}_{i=1, \dots, n}$ de $n$ observations en $p$ dimensions et
leurs �tiquettes r�elles. Nous consid�rons comme espace des hypoth�ses un
ensemble de mod�les param�tris�s par un vecteur $\bbeta \in \mathbb{R}^{m}$.


\subsection{Minimisation du risque empirique}
Si nous utilisons comme fonction de co�t le co�t quadratique d�fini par
l'�quation~\eqref{eq:quadratic_loss}, la minimisation du risque empirique comme
d�finie par l'�quation~\eqref{eq:erm} consiste � trouver
\begin{equation}
  \label{eq:erm_parametric}
  \bbeta^* \in \argmin_{\bbeta \in \RR^m} \frac{1}{n} \sum_{i=1}^n (f_{\bbeta}(\xx^i) - y^i)^2.
\end{equation}

C'est ce que l'on appelle la \textbf{minimisation des moindres carr�s}, une
m�thode bien connue depuis Gauss et Legendre. 


\subsection{Formulation probabiliste des r�gressions param�triques $\bigstar$}
Nous supposons comme pr�c�demment que les couples $(\xx^i, y^i)$ sont les
r�alisations de $n$ vecteurs al�atoires de m�me loi qu'un couple de variables
al�atoire $(X, Y).$ 

Cela revient � supposer que la relation entre $X$ et $Y$ peut s'�crire comme 
\begin{equation}
  \label{eq:param_error}
  Y = f_{\bbeta}(X) + \epsilon. 
\end{equation}

Faisons maintenant \textbf{l'hypoth�se d'un bruit gaussien centr� en $0$ : } le
terme d'erreur $\epsilon$ est normalement distribu�, centr� en $0$.
\begin{equation}
  \label{eq:param_error_gaussian}
  \epsilon \sim \Ncal(0, \; \sigma^2).
\end{equation}

L'�quation~\eqref{eq:param_error} revient alors � 
\begin{equation}
  \label{eq:param_bayes}
  Y|X=\xx \sim  \Ncal\left(f_{\bbeta}(\xx), \; \sigma^2\right).
\end{equation}


\begin{exemple}
  L'�quation~\eqref{eq:param_bayes} est illustr�e sur la
  figure~\ref{fig:linreg} dans le cas o� $p=1$ et l'espace des hypoth�ses est
  l'ensemble des fonctions lin�aires d'une variable :
  $\FF = \{ x \mapsto f_{\alpha, \beta}(x) = \alpha x + \beta \; ; (\alpha,
  \beta) \in \RR^2 \}$.
  La distribution des valeurs de l'�tiquette d'un individu $x^*$ selon le
  mod�le $f_{\alpha, \beta}$ est une gaussienne centr�e en
  $f_{\alpha, \beta}(x^*)$. Sa densit� est not�e $g_{Y|X=x^*}$. 
\end{exemple}

\begin{figure}[h]
  \centering
  \includegraphics[width=0.6\textwidth]{figures/erm/linreg}
  \caption{Pour une observation $x^*$ donn�e (ici en une dimension) , la
    distribution des valeurs possibles de l'�tiquette correspondante est une
    gaussienne centr�e en $f(x^*)$. La vraie valeur de l'�tiquette est $y^*$.}
  \label{fig:linreg}
\end{figure}


\subsection{Estimation par maximum de vraisemblance $\bigstar$}
\label{sec:least_squares}
Sous l'hypoth�se~\eqref{eq:param_bayes}, nous pouvons donc estimer $\bbeta$ en
maximisant la log-vraisemblance de l'�chantillon
$\left((\xx^1, y^1), (\xx^2, y^2), \dots, (\xx^n, y^n) \right), $ consid�r�
comme la r�alisation de l'�chantillon al�atoire\\
$\left((X_1, Y_1), (X_2, Y_2), \dots, (X_n, Y_n) \right), $ lui m�me constitu�
de $n$ copies i.i.d. de $(X, Y)$.

En notant $g_{X,Y}$ la densit� jointe de $(X, Y)$; $g_{Y|X=x}$ la densit� de
$Y|X=x$; et $g_X$ la densit� de $X$, cette log-vraisemblance s'�crit 


\begin{align*}
  \ell\left((\xx^1, y^1), (\xx^2, y^2), \dots, (\xx^n, y^n); \bbeta  \right)
  & = \ln \prod_{i=1}^n g_{X, Y}(\xx^i, y^i) 
                       = \ln \prod_{i=1}^n g_{Y|X=\xx^i}(y^i) + \ln \prod_{i=1}^n
                         g_X(\xx^i) \\
                       & = - \frac1{2\sigma^2} \sum_{i=1}^n \left(y^i -
                         f_{\bbeta}(\xx^i) \right)^2 + \Ccal.
\end{align*}
Dans cette derni�re �quation, $\Ccal$ est une constante par rapport � $\bbeta$,
et provient d'une part du coefficient $\frac1{\sqrt{2\pi}}$ de la distribution
normale et d'autre part des $g_X(\xx^i)$.

Ainsi, maximiser la vraisemblance dans ce contexte de bruit gaussien centr�
revient � minimiser 
\[\sum_{i=1}^n \left(y^i - f_{\bbeta}(\xx^i) \right)^2.\]
On retrouve ici la m�thode des moindres carr�s de l'�quation~\eqref{eq:erm_parametric}.

\section{R�gression lin�aire}
\label{sec:linreg}
Nous allons maintenant appliquer la minimisation des moindres carr�s au cas o�
$\FF$ est l'ensemble des fonctions lin�aires de $p$ variables.

\subsection{Formulation}
Nous choisissons une fonction de d�cision $f$ de la forme
\begin{equation}
  \label{eq:linear_decision}
  f: \xx \mapsto \beta_0 + \sum_{j=1}^p \beta_j x_j.
\end{equation}
Ici, $\bbeta \in \RR^{p+1}$ et donc $m=p+1$.

\subsection{Solution}
On appelle \textbf{r�gression lin�aire} le mod�le de la forme
$f: \xx \mapsto \beta_0 + \sum_{j=1}^p \beta_j x_j$ dont les coefficients sont
obtenus par minimisation de la somme des moindres carr�s, � savoir :
\begin{equation}
  \label{eq:linreg}
  \argmin_{\bbeta \in \RR^{p+1}}  \sum_{i=1}^n \left(y^i - \left(\beta_0 + 
      \sum_{j=1}^p \beta_j x_j \right)\right)^2.
\end{equation}
  
Nous pouvons r��crire le probl�me~\ref{eq:linreg} sous forme matricielle, en
ajoutant � gauche � la matrice d'observations $X \in \RR^p$ une colonne de 1 :
\begin{equation}
  \label{eq:added_ones}
  X \leftarrow   \begin{pmatrix}
    1 & x_1^1 & \cdots & x_p^1 \\
    \vdots & \vdots & \cdots & \vdots \\
    1 & x_1^n& \cdots & x_p^n \\
  \end{pmatrix}.
\end{equation}

La somme des moindres carr�s s'�crit alors
\begin{equation}
  \label{eq:rss_linreg}
  \text{RSS} = \left(\yy - X \bbeta\right)^\top \left(\yy -  X \bbeta\right).
\end{equation}

Il s'agit d'une forme quadratique convexe en $\bbeta$, que l'on peut donc
minimiser en annulant son gradient
$\nabla_{\bbeta} \text{RSS} = -2 X^\top \left(\yy - X \bbeta \right)$. La somme
des moindres carr�s est minimale si $\bbeta$ v�rifie 
\begin{equation}
  \label{eq:linreg_sol}
  X^\top X \bbeta = X^\top \yy.
\end{equation}
  
\paragraph{Solution explicite}
Si le rang de la matrice $X$ est �gal � son nombre de colonnes, alors
$X^\top X$ est inversible et la somme des moindres carr�s de
l'�quation~\eqref{eq:rss_linreg} est minimis�e pour
\begin{equation*}
  \bbeta^* = \left(X^\top X \right)^{-1} X^\top \yy.
\end{equation*}

Si $X^\top X$ n'est pas inversible, on pourra n�anmoins trouver une solution
(non unique) pour $\bbeta$ en utilisant � la place de
$\left(X^\top X \right)^{-1}$ un pseudo-inverse (par exemple, celui de
Moore-Penrose) de $X^\top X$, c'est-�-dire une matrice $M$ telle que
$X^\top X M X^\top X = X^\top X.$

\paragraph{M�thode de descente}
On peut aussi (et ce sera pr�f�rable quand $p$ est grand et que l'inversion de
la matrice $X^\top X \in \RR^{p \times p}$ est donc co�teuse) obtenir une
estimation de $\bbeta$ par un algorithme � directions de descente.

\paragraph{Interpr�tation} La r�gression lin�aire produit un mod�le interpr�table, au sens o� les
$\beta_j$ permettent de comprendre l'importance relative des variables sur la
pr�diction. En effet, plus $\lvert \beta_j \rvert$ est grande, plus la $j$-�me
variable a un effet important sur la pr�diction, et le signe de $\beta_j$ nous
indique la direction de cet effet.

Attention ! Cette interpr�tation n'est valide que si les variables ne sont pas
corr�l�es, et que $x_j$ peut �tre modifi�e sans perturber les autres
variables. De plus, si les variables sont corr�l�es, $X$ n'est pas de rang
colonne plein et $X^\top X$ n'est donc pas inversible. Ainsi la r�gression
lin�aire admet plusieurs solutions. Intuitivement, on peut passer de l'une �
l'autre de ces solutions car une perturbation d'un des poids $\beta_j$ peut
�tre compens�e en modifiant les poids des variables corr�l�es � $x_j$.

\paragraph{Remarque} Nous avons trait� ici de probl�mes de \textit{r�gression}
uniquement. Nous traiterons de classification param�trique dans la PC 6.



  
% \section{Points cl�s}
% \begin{itemize}
% \item Les trois ingr�dients d'un algorithme d'apprentissage supervis� sont :
%   \begin{itemize}
%   \item l'espace des hypoth�ses,
%   \item la fonction de co�t,
%   \item l'algorithme d'optimisation qui permet de trouver l'hypoth�se
%     optimale au sens de la fonction de co�t sur les donn�es (minimisation du
%     risque empirique).
%   \end{itemize}
% \item On peut apprendre les coefficients d'un mod�le de r�gression param�trique
%   par maximisation de vraisemblance, ce qui �quivaut � minimiser le risque
%   empirique en utilisant le co�t quadratique comme fonction de perte, et
%   revient � la m�thode des moindres carr�s.
% \item La r�gression lin�aire admet une unique solution $\bbeta^* = \left(X^\top
%     X \right)^{-1} X^\top \yy$ si et seulement si $X^\top X$ est
%   inversible. Dans le cas contraire, il existe une infinit� de solutions.

% \end{itemize}


% \begin{plusloin}
% \item La notion de complexit� d'un mod�le a �t� formalis�e par Vladimir Vapnik et
%   Alexey Chervonenkis dans les ann�es 1970, et est d�taill�e par exemple
%   dans l'ouvrage de \citet{vapnik1995}.
% \item Pour en savoir plus sur la th�orie de l'apprentissage, on pourra se
%   r�f�rer au livre de \citet{kearns1994}.
% \item On trouvera une discussion d�taill�e du compromis biais-variance dans
%   \citet{friedman1997}.
% \end{plusloin}

% % \section*{Bibliographie}
% % \vspace{-25pt}
% % \begin{thebibliography}{99}
% % \bibitem[\protect\astroncite{Crammer and Singer}{2001}]{crammer2001}
% % Crammer, K. and Singer, Y. (2001).
% % \newblock On the algorithmic implementation of multiclass kernel-based vector
% %   machines.
% % \newblock {\em Journal of Machine Learning Research}, 2:265--292.

% % \bibitem[\protect\astroncite{Friedman}{1997}]{friedman1997} Friedman,
% %   J.~H. (1997).  \newblock On bias, variance, 0/1-loss and the curse of
% %   dimensionality.  \newblock {\em Data Mining and Knowledge Discovery},
% %   1:55--77.

% % \bibitem[\protect\astroncite{Kearns et Vazirani}{1994}]{kearns1994} Kearns,
% %   M.~J. et Vazirani, U.~V. (1994).  \newblock {\em An Introduction to
% %     Computational Learning Theory}.  \newblock MIT Press, Cambridge, MA.

% % \bibitem[\protect\astroncite{Vapnik}{1995}]{vapnik1995} Vapnik, V.~N. (1995).
% %   \newblock {\em The Nature of Statistical Learning Theory}.  \newblock
% %   Springer, New York.

% % \bibitem[\protect\astroncite{Weston and Watkins}{1999}]{weston1999}
% % Weston, J. and Watkins, C. (1999).
% % \newblock Support vector machines for multi-class pattern recognition.
% % \newblock In {\em European Symposium on Artificial Neural Networks}.
% % \end{thebibliography}



%%% Local Variables:
%%% mode: latex
%%% TeX-master: "sdd_2020_poly"
%%% End:

\chapter{G�n�ralisation}
%-*- coding: iso-latin-1 -*-    
\label{chap:generalisation}

\paragraph{Notions :} g�n�ralisation ; surapprentissage ; s�lection de mod�le
; validation crois�e ; r�gularisation des mod�les param�triques lin�aires
\paragraph{Objectifs p�dagogiques :} 
\begin{itemize}      
  \setlength{\itemsep}{3pt}
\item D�tecter un risque de surapprentissage ;
\item Mettre en place un cadre permettant de s�lectionner un mod�le parmi
  plusieurs et d'estimer sa performance en g�n�ralisation ;
\item Utiliser la r�gularisation pour �viter le surapprentissage ;
\item Manipuler les r�gularisations $\ell_1$ et $\ell_2$ sur des mod�les lin�aires.
\end{itemize}

\section{G�n�ralisation et surapprentissage}
\subsection{G�n�ralisation}
Imaginons un algorithme qui, pour pr�dire l'�tiquette d'une observation $\xx$,
retourne son �tiquette si $\xx$ appartient aux donn�es dont l'�tiquette est
connue, et une valeur al�atoire sinon. Ce mod�le, qui en quelque sorte �
apprend par c\oe{}ur �, aura un risque empirique nul (et donc minimal) quelle
que soit la fonction de co�t choisie. Cependant, il fera de tr�s mauvaises
pr�dictions pour toute nouvelle observation.

\begin{encadre}
{C'est pourquoi �valuer un algorithme de machine learning sur les donn�es sur
lesquelles il a appris ne nous permet absolument pas de savoir comment il se
comportera sur de nouvelles donn�es, en d'autres mots, sa capacit� � \textbf{
  g�n�raliser}. C'est un des aspects les plus importants de l'apprentissage automatique.}
\end{encadre}

\subsection{Surapprentissage}
L'exemple, certes extr�me, que nous avons pris plus haut, illustre que l'on
peut facilement mettre au point une proc�dure d'apprentissage qui produise un
mod�le qui fait de bonnes pr�dictions sur les donn�es utilis�es pour le
construire, mais g�n�ralise mal. Au lieu de mod�liser la vraie nature des
objets qui nous int�ressent ($\Phi(X)$ dans
l'�quation~\eqref{eq:probabilistic_ml}), un tel mod�le capture aussi (voire
surtout) un bruit ($\epsilon$ dans l'�quation~\eqref{eq:probabilistic_ml}) qui
n'est pas pertinent pour l'application consid�r�e.

On dit d'un mod�le qui, plut�t que de capturer la nature des objets �
�tiqueter, mod�lise aussi le bruit et ne sera pas en mesure de g�n�raliser
qu'il \textbf{surapprend} (\textit{overfits} en anglais). 
Un mod�le qui surapprend est g�n�ralement un mod�le \textbf{trop complexe}, qui
\og colle \fg~trop aux donn�es et capture donc aussi leur bruit.
  
� l'inverse, il est aussi possible de construire un mod�le \textbf{trop simple},
dont les performances ne soient bonnes ni sur les donn�es utilis�es pour le
construire, ni en g�n�ralisation. 
On dit d'un mod�le qui est trop simple pour avoir de bonnes performances m�me
sur les donn�es utilis�es pour le construire qu'il \textbf{sous-apprend} (\textit{underfits} en
anglais).

Ces concepts sont illustr�s sur la figure~\ref{fig:overfit_class} pour un
probl�me de classification binaire et la figure~\ref{fig:overfit_regr} pour un
probl�me de r�gression.

\begin{figure}[h]
  \begin{subfigure}[t]{0.48\textwidth}
    \centering
    \includegraphics[width=\textwidth]{figures/generalisation/overfit_class}
    \caption{Pour s�parer les observations n�gatives (x) des observations
      positives (+), la droite pointill�e sous-apprend. La fronti�re de
      s�paration en trait plein ne fait aucune erreur sur les donn�es mais est
      susceptible de surapprendre. La fronti�re de s�paration en trait
      discontinu est un bon compromis.}
    \label{fig:overfit_class}
  \end{subfigure}
  \hfill
  \begin{subfigure}[t]{0.48\textwidth}
    \centering
    \includegraphics[width=\textwidth]{figures/generalisation/overfit_regr}
    \caption{Les �tiquettes $y$ des observations (repr�sent�es par des points)
      ont �t� g�n�r�es � partir d'un polyn�me de degr� $d=3$. Le mod�le de
      degr� $d=2$ approxime tr�s mal les donn�es et sous-apprend, tandis que
      celui de degr� $d=13$, dont le risque empirique est plus faible,
      surapprend.}
    \label{fig:overfit_regr}
  \end{subfigure}
  \caption{Sous-apprentissage et surapprentissage}
\end{figure}

\subsection{Compromis biais-variance $\bigstar$}
\label{sec:bias_variance}
Supposons disposer d'un jeu de donn�es $\DD = \{\xx^i, y^i\}_{i=1, \dots, n}$
de $n$ observations en $p$ dimensions et leurs �tiquettes r�elles. Nous
supposons comme au chapitre pr�c�dent que les couples $(\xx^i, y^i)$ sont les
r�alisations de $n$ vecteurs al�atoires de m�me loi qu'un couple de variables
al�atoire $(X, Y)$, $X$ �tant un vecteur al�atoire $p$-dimensionnel et $Y$ une
variable al�atoire r�elle � valeurs dans $\YY$, et qu'il existe une fonction
$\Phi : \RR^p \rightarrow \YY$ et une variable al�atoire r�elle $\epsilon$
telle que
  \begin{equation}
  Y = \Phi(X) + \epsilon.
\end{equation}
Nous supposons de plus que $\epsilon$ est d'esp�rance nulle et de variance
$\sigma^2$.

Fixons maintenant un couple $(\xx, y) \in \RR^p \times \YY$. Nous pouvons
consid�rer que la pr�diction $f(\xx)$ d'un mod�le $f$ appris sur $\DD$ est une
estimation de $\Phi(x)$, autrement dit la r�alisation d'une variable al�atoire
$F_n$ qui est une fonction d'un �chantillon al�atoire de $n$ copies iid de
$(X, Y)$. Nous pouvons alors calculer l'erreur quadratique moyenne de la
pr�diction $F_n$ :
\begin{align*}
  & \EE\left((y-F_n)^2\right)  = \EE\left((\Phi(\xx) + \epsilon - F_n)^2   \right)
   = \EE\left((\Phi(\xx) - \EE(F_n) + \EE(F_n) - F_n  + \epsilon )^2   \right) \\
  & = \EE\left((\Phi(\xx) - \EE(F_n))^2 + (\EE(F_n) - F_n  + \epsilon )^2 + 2
    (\Phi(\xx) - \EE(F_n))(\EE(F_n) - F_n  + \epsilon) \right) \\
  & = (\Phi(\xx) - \EE(F_n))^2 + \EE\left((F_n - \EE(F_n) - \epsilon )^2  \right) + 2
    (\Phi(\xx) - \EE(F_n))  \EE\left( \EE(F_n) - F_n  + \epsilon \right) \\
  & =  \text{B}(F_n)^2  + 
    \EE \left((F_n - \EE(F_n))^2 + \epsilon^2 - 2 \epsilon (F_n - \EE(F_n)) \right) + 2
    (\Phi(\xx) - \EE(F_n))  \left( \EE(F_n) - \EE(F_n)  + \EE(\epsilon) \right) \\
  & =  \text{B}(F_n)^2  + 
    \EE \left((F_n - \EE(F_n))^2 \right) + \EE(\epsilon^2) - 2 \EE\left(\epsilon (F_n - \EE(F_n)) \right) \\
    & = \text{B}(F_n)^2 + \VV(F_n) + \sigma^2.
\end{align*}
Le passage de la deuxi�me � la troisi�me ligne se fait par lin�arit� de
l'esp�rance et en observant que $\Phi(\xx)$ et $\EE(F_n)$ sont d�terministes
(ce sont des nombres, pas des variables al�atoires).

Le troisi�me terme de la somme de la quatri�me ligne disparait � la cinqui�me
car $\EE(\epsilon)=0$.

Enfin, le passage � la derni�re ligne se fait en supposant que $\epsilon$ et
$F_n$ sont ind�pendants ; on a alors
$\EE\left(\epsilon (F_n - \EE(F_n)) \right) = \EE(\epsilon) \EE(F_n -
\EE(F_n)).$ De plus, $\EE(\epsilon^2) = \VV(\epsilon)$ car $\EE(\epsilon)=0$.

Ainsi, l'erreur quadratique moyenne est la somme
\begin{itemize}
\item du carr� du biais de l'estimateur, qui quantifie � quel point les �tiquettes pr�dites diff�rent des vraies �tiquettes ;
\item de la variance de l'estimateur, qui quantifie � quel point les �tiquettes
  pr�dites pour le m�me individu $\xx$ diff�rent selon les donn�es d'entr�e
  (i.e. les r�alisations des n copies iid de $(X, Y)$) ;
\item de la variance du bruit, aussi appel�e \textbf{erreur irr�ductible} : ce terme sera l� m�me si l'estimation de $\Phi$ est exacte.
\end{itemize}

On retrouve ici le compromis biais-variance (cf
section~\ref{sec:precision_estimateur}) : un mod�le biais� peut, s'il a une variance plus faible, faire de meilleures pr�dictions qu'un mod�le non-biais�.



%Pour mieux comprendre le risque d'un mod�le $f: \XX \rightarrow \YY$, nous
% pouvons le comparer � l'erreur minimale $\Rcal^*$ qui peut �tre atteinte par
% n'importe quelle fonction mesurable de $\XX$ dans $\YY$ : c'est ce qu'on
% appelle l'\textit{exc�s d'erreur}, et que l'on peut d�composer de la fa�on
% suivante :
% \begin{equation}
%   \label{eq:estimation_approximation}
%   \Rcal(f) - \Rcal^* = \left[ \Rcal(f) - \min_{h \in \FF} 
%     \Rcal(h) \right]
%   + \left[ \min_{h \in \FF} \Rcal(h) - \Rcal^* \right]
% \end{equation}

% Le premier terme, $\Rcal(f) - \min_{h \in \FF} \Rcal(h)$, quantifie la distance
% entre le mod�le $f$ et le mod�le optimal sur $\FF$. C'est ce que l'on appelle
% {\it l'erreur d'estimation.}

% Le second terme, $\min_{h \in \FF} \Rcal(h) - \Rcal^*$, quantifie la qualit� du
% mod�le optimal sur $\FF$, autrement dit, la qualit� du choix de l'espace des
% hypoth�ses. C'est ce que l'on appelle {\it l'erreur d'approximation.} Si $\FF$
% est l'ensemble des fonctions mesurables, alors l'erreur d'approximation est
% nulle.

% Ainsi, l'�criture~\ref{eq:estimation_approximation} permet de d�composer
% l'erreur entre un terme qui d�coule de la qualit� de l'espace des hypoth�ses et
% un autre qui d�coule de la qualit� de la proc�dure d'optimisation utilis�e. En
% pratique, sauf dans des cas tr�s particuliers o� cela est rendu possible par
% construction, il n'est pas possible de calculer ces termes d'erreur.
% Cependant, cette �criture nous permet de comprendre le probl�me suivant :
% choisir un espace des hypoth�ses plus large permet g�n�ralement de r�duire
% l'erreur d'approximation, car un mod�le plus proche de la r�alit� a plus de
% chances de se trouver dans cet espace. Cependant, puisque cet espace est plus
% vaste, la solution optimale y est aussi g�n�ralement plus difficile � trouver :
% l'erreur d'estimation, elle, augmente. C'est dans ce cas qu'il y a
% surapprentissage. \\

% Un espace des hypoth�ses plus large permet g�n�ralement de construire des
% mod�les plus complexes : par exemple, l'ensemble des droites vs. l'ensemble des
% polyn�mes de degr� 13 (cf. figure~\ref{fig:overfit_regr}). C'est une variante du
% principe du \textbf{rasoir d'Ockham}, selon lequel les
% hypoth�ses les plus simples sont les plus vraisemblables.

% Il y a donc un compromis entre erreur d'approximation et erreur d'estimation :
% il est difficile de r�duire l'une sans augmenter l'autre. Ce compromis est
% g�n�ralement appel� \textbf{compromis biais-variance} : l'erreur d'approximation
% correspond au \textbf{biais} de la proc�dure d'apprentissage, tandis que l'erreur
% d'estimation correspond � sa \textbf{variance}. % On retrouvera ce compromis dans
% % l'estimation bay�sienne de param�tres � la
% % section~\ref{sec:bias_variance_bayes}.

% Consid�rons par exemple pour un probl�me de r�gression un espace des hypoth�ses
% na�f qui ne contient que des fonctions constantes. Supposons que les �tiquettes
% soient g�n�r�es par une distribution normale centr�e en $a$. Quelles que soient
% les donn�es observ�es, la proc�dure d'apprentissage va construire un mod�le qui
% retourne $a$ quelle que soit l'observation concern�e : la \textbf{variance} de la
% proc�dure par rapport au jeu de donn�es est tr�s faible. � l'inverse, comme la
% fonction de pr�diction apprise est tr�s peu sensible au jeu de donn�es, il y a
% un \textbf{biais} tr�s important qui conduit � construire des pr�dicteurs qui
% retournent $a$ pour toutes les observations.

\section{S�lection de mod�le}
Le th�or�me du \textit{no free lunch} % de~\citet{wolpert1997}
indique qu'aucun
algorithme de machine learning ne peut bien fonctionner pour \textbf{tous} les
probl�mes d'apprentissage : un algorithme qui fonctionne bien sur un type
particulier de probl�mes le compensera en fonctionnant moins bien sur d'autres
types de probl�mes. En d'autres termes, il n'y a pas de \og baguette magique
\fg~qui puisse r�soudre tous nos probl�mes de machine learning, et il est donc
essentiel, pour un probl�me donn�, de tester plusieurs possibilit�s afin de
s�lectionner le mod�le optimal. Notons au passage que plusieurs crit�res
peuvent intervenir dans ce choix : non seulement celui de la qualit� des
pr�dictions, qui nous int�resse dans ce chapitre, mais aussi celui des
ressources de calcul n�cessaires, qui peuvent �tre un facteur limitant en
pratique.

L'erreur empirique mesur�e sur les observations qui ont permis de construire le
mod�le est un mauvais estimateur de l'erreur du mod�le sur l'ensemble des
donn�es possibles, ou \textbf{erreur de g�n�ralisation} : si le mod�le
surapprend, cette erreur empirique peut �tre proche de z�ro voire nulle,
tandis que l'erreur de g�n�ralisation peut �tre arbitrairement grande.

\subsection{Jeu de test}
Il est donc indispensable d'utiliser pour �valuer un mod�le des donn�es
�tiquet�es qui n'ont pas servi � le construire. La mani�re la plus simple d'y
parvenir est de mettre de c�t� une partie des observations, r�serv�es �
l'�valuation du mod�le, et d'utiliser uniquement le reste des donn�es pour le
construire.

�tant donn� un jeu de donn�es $\DD = \{(\xx^i, y^i)\}_{i=1, \dots, n}$,
partitionn� en deux jeux $\DD_{\text{tr}}$ et $\DD_{\text{te}}$, on appelle
\textbf{jeu d'entra�nement} (\textit{training set} en anglais) l'ensemble
$\DD_{\text{tr}}$ utilis� pour entra�ner un mod�le pr�dictif, et \textbf{jeu de
  test} (\textit{test set} en anglais) l'ensemble $\DD_{\text{te}}$ utilis� pour
son �valuation.

Comme nous n'avons pas utilis� le jeu de test pour entra�ner notre mod�le, il
peut �tre consid�r� comme un jeu de donn�es \og nouvelles \fg. La perte
calcul�e sur ce jeu de test est un estimateur de l'erreur de g�n�ralisation.

Cela correspond � ce que nous avons fait dans la PC3 et au d�but du mini-projet.

\subsection{Jeu de validation}
Consid�rons maintenant la situation dans laquelle nous voulons choisir entre
$K$ mod�les. Nous pouvons alors entra�ner chacun des mod�les sur le jeu de
donn�es d'entra�nement, obtenant ainsi $K$ fonctions de d�cision
$f_1, f_2, \dots, f_K$, puis calculer l'erreur de chacun de ces mod�les sur le
jeu de test. Nous pouvons ensuite choisir comme mod�le celui qui a la plus
petite erreur sur le jeu de test:
\begin{equation}
  \hat f = \argmin_{k=1, \dots, K} \frac{1}{|\DD_{\text{te}}|} 
  \sum_{\xx, y \in \DD_{\text{te}}} L(y, f_k(\xx)).
\end{equation}
Mais quelle est son erreur de g�n�ralisation ? Comme nous avons utilis�
$\DD_{\text{te}}$ pour s�lectionner le mod�le, il ne repr�sente plus un
jeu ind�pendant compos� de donn�es nouvelles, inutilis�es pour d�terminer le
mod�le.

La solution est alors de d�couper notre jeu de donn�es en \textbf{trois} parties~:
\begin{itemize}
\item Un \textbf{jeu d'entra�nement}
  $\DD_{\text{tr}}$ sur lequel nous pourrons entra�ner nos $K$ algorithmes
  d'apprentissage ;
\item Un \textbf{jeu de validation} ({\it validation set} en anglais)
  $\DD_{\text{val}}$ sur lequel nous �valuerons les $K$ mod�les ainsi
  obtenus, afin de \textbf{s�lectionner} un mod�le d�finitif ;
\item Un \textbf{jeu de test} $\DD_{\text{te}}$ sur lequel nous �valuerons enfin
  l'erreur de g�n�ralisation du mod�le choisi.
\end{itemize}

On voit ici qu'il est important de distinguer la {\it s�lection} d'un mod�le de
son \textbf{�valuation} : les faire sur les m�mes donn�es peut nous conduire �
sous-estimer l'erreur de g�n�ralisation et le surapprentissage du mod�le
choisi.

Une fois un mod�le s�lectionn�, on peut le r�-entra�ner sur l'union du jeu
d'entra�nement et du jeu de validation afin de construire un mod�le final.

\subsection{Validation crois�e}
\label{sec:crossval}
La s�paration d'un jeu de donn�es en un jeu d'entra�nement et un jeu de test
est n�cessairement arbitraire. Nous risquons ainsi d'avoir, par hasard, cr��
des jeux de donn�es qui ne sont pas repr�sentatifs. Pour �viter cet �cueil, il
est souhaitable de reproduire plusieurs fois la proc�dure, puis de moyenner les
r�sultats obtenus afin de moyenner ces effets al�atoires. Le cadre le plus
classique pour ce faire est celui de la \textbf{validation crois�e}, illustr� sur
la figure~\ref{fig:crossval}

\begin{figure}[h]
  \centering
  \includegraphics[width=0.5\textwidth]{figures/generalisation/crossval}
  \caption{Une validation crois�e en 5 {\it folds} : Chaque observation
    appartient � un des 5 jeux de validation (en blanc) et aux 4 autres jeux
    d'entra�nement (en noir).}
  \label{fig:crossval}
\end{figure}

�tant donn� un jeu $\DD$ de $n$ observations, et un nombre $K$, on appelle {\it
  validation crois�e} la proc�dure qui consiste �
\begin{enumerate}
\item partitionner $\DD$ en $K$ parties de tailles sensiblement similaires,
  $\DD_1, \DD_2, \dots, \DD_K$
\item pour chaque valeur de $k=1, \dots, K$,
  \begin{itemize}
  \item entra�ner un mod�le sur $\bigcup_{l \neq k} \DD_l$
  \item �valuer ce mod�le sur $\DD_k$.
  \end{itemize}
\end{enumerate}
Chaque partition de $\DD$ en deux ensembles $\DD_k$ et $\bigcup_{l \neq
  k} \DD_l$ est appel�e un \textbf{fold} de la validation crois�e.

Chaque observation �tiquet�e du jeu $\DD$ appartient � un unique jeu de test,
et � $(K-1)$ jeux d'entra�nement. Ainsi, cette proc�dure g�n�re une pr�diction
par observation de $\DD$. Pour conclure sur la performance du mod�le, on peut :
\begin{itemize}
\item soit �valuer la qualit� des pr�dictions sur $\DD$ ;  
\item soit �valuer la qualit� de chacun des $K$ pr�dicteurs sur le jeu de
  test $\DD_k$ correspondant, et moyenner leurs performances. Cette deuxi�me
  approche permet aussi de rapporter l'�cart-type de ces performances, ce qui
  permet de se faire une meilleure id�e de la variabilit� de la qualit� des
  pr�dictions en fonction des donn�es d'entra�nement.
\end{itemize}

\section{Crit�res de performance}
% Il existe de nombreuses fa�ons d'�valuer la performance pr�dictive d'un mod�le
% d'apprentissage supervis�. Cette section pr�sente les principaux crit�res
% utilis�s.

\subsection{Matrice de confusion et crit�res d�riv�s}
\label{sec:confusion_matrix}
Comme nous l'avons vu, le nombre d'erreurs de classification permet d'�valuer
la qualit� d'un mod�le pr�dictif. Notons que l'on pr�f�rera g�n�ralement
d�crire le nombre d'erreurs comme une fraction du nombre d'exemples : un taux
d'erreur de $1\%$ est plus parlant qu'un nombre absolu d'erreurs.
 
Mais toutes les erreurs ne se valent pas n�cessairement. Prenons l'exemple d'un
mod�le qui pr�dise si oui ou non une radiographie pr�sente une tumeur
inqui�tante : une fausse alerte, qui sera ensuite infirm�e par des examens
compl�mentaires, est moins probl�matique que de ne pas d�celer la tumeur et de
ne pas traiter la personne concern�e.  Les performances d'un mod�le de
classification binaire peuvent �tre r�sum�es dans une
\textbf{matrice de confusion} : une matrice $M$ de deux lignes et deux
colonnes, et dont l'entr�e $M_{ck}$ est le nombre d'exemples de la classe $c$
pour laquelle l'�tiquette $k$ a �t� pr�dite.
\begin{center}
  \begin{tabular}[h]{|c|c|c|c|} \hline \multicolumn{2}{|c|}{} &
      \multicolumn{2}{|c|}{Classe r�elle} \\ \hline \multicolumn{2}{|c|}{} & 0 & 1 \\ \hline 
    Classe & 0 & vrais n�gatifs (TN) & faux n�gatifs (FN)
    \\ \cline{2-4} pr�dite & 1 & faux positifs (FP) & vrais positifs (TP)
    \\ \hline
  \end{tabular}
\end{center}

On retrouve ici les erreurs de premi�re et de deuxi�me esp�ce de la
section~\ref{sec:test_errors}, qu'on appellera plut�t respectivement faux
positifs et faux n�gatifs dans ce contexte.

Il est possible de d�river de nombreux crit�res d'�valuation construits �
partir de la matrice de confusion, comme la sp�cificit�, la sensibilit�, le
rappel ou la F-mesure. Vous pouvez vous r�f�rer � la
section~\ref{sec:confusion_matrix_derived} � la fin de ce chapitre, ou � la
\href{https://scikit-learn.org/stable/modules/model_evaluation.html#classification-metrics}{documentation
  de \texttt{sklearn.metrics}}.



\subsection{Courbe ROC $\bigstar$}
De nombreux algorithmes de classification ne retournent pas directement une
�tiquette de classe, mais utilisent une fonction de d�cision qui doit ensuite
�tre seuill�e pour devenir une �tiquette. Cette fonction de d�cision peut �tre
un score arbitraire, ou la probabilit� d'appartenir � la classe positive.
  
On appelle \textbf{courbe ROC}, de l'anglais \textbf{Receiver-Operator
  Characteristic} la courbe d�crivant l'�volution de la sensibilit� en fonction
du compl�mentaire � 1 de la sp�cificit�, parfois appel� \textbf{antisp�cificit�},
lorsque le seuil de d�cision change.

Le terme vient des t�l�communications, o� ces courbes servent � �tudier si un
syst�me arrive � s�parer le signal du bruit de fond.

On peut synth�tiser une courbe ROC par l'aire sous cette courbe, souvent
abr�g�e \textbf{AUROC} pour \textbf{Area Under the ROC}.

Un exemple de courbe ROC est pr�sent� sur la figure~\ref{fig:roc_curve}. Le
point $(0, 0)$ appara�t quand on utilise comme seuil un nombre sup�rieur � la
plus grande valeur retourn�e par la fonction de d�cision : ainsi, tous les
exemples sont �tiquet�s n�gatifs. � l'inverse, le point $(1, 1)$ appara�t quand
on utilise pour seuil une valeur inf�rieure au plus petit score retourn� par la
fonction de d�cision : tous les exemples sont alors �tiquet�s positifs.

\begin{figure}[h]
  \centering
  \includegraphics[width=0.5\textwidth]{figures/generalisation/roc_curve}
  \caption{Les courbes ROC de deux mod�les.}
  \label{fig:roc_curve}
\end{figure}

Pour construire la courbe ROC, on prend pour seuil les valeurs successives de
la fonction de d�cision sur notre jeu de donn�es. Ainsi, � chaque nouvelle
valeur de seuil, une observation que l'on pr�disait pr�c�demment n�gative
change d'�tiquette. Si cette observation est effectivement positive, la
sensibilit� augmente de $1/n_p$ (o� $n_p$ est le nombre d'exemples positifs) ;
sinon, c'est l'antisp�cificit� qui augmente de $1/n_n$, o� $n_n$ est le nombre
d'exemples n�gatifs. La courbe ROC est donc une courbe en escaliers.

Un classifieur id�al, qui ne commet aucune erreur, associe syst�matique des
scores plus faibles aux exemples n�gatifs qu'aux exemples positifs. Sa courbe
ROC suit donc le coin sup�rieur gauche du carr� $[0, 1]^2$ ; il a une aire sous
la courbe de 1.

La courbe ROC d'un classifieur al�atoire, qui fera sensiblement la m�me
proportion d'erreurs que de classifications correctes quel que soit le seuil
utilis�, suit la diagonale de ce carr�. L'aire sous la courbe ROC d'un
classifieur al�atoire vaut donc 0.5.

On peut enfin utiliser la courbe ROC pour choisir un seuil de d�cision, �
partir de la sensibilit� (ou de la sp�cificit�) que l'on souhaite garantir.

\subsection{Erreurs de r�gression}
% Dans le cas d'un probl�me de r�gression, le nombre d'erreurs n'est pas un
% crit�re appropri� pour �valuer la performance. D'une part, � cause des
% impr�cisions num�riques, il est d�licat de dire d'une pr�diction � valeur
% r�elle si elle est correcte ou non. D'autre part, un mod�le dont $50\%$ des
% pr�dictions sont correctes � $0.1\%$ pr�s et les $50$ autres pourcent sont tr�s
% �loign�es des vraies valeurs vaut-il mieux qu'un mod�le qui n'est correct qu'�
% $1\%$ pr�s, mais pour $100\%$ des exemples ?

% Ainsi, on pr�f�rera quantifier la performance d'un mod�le de r�gression en
% fonction de l'�cart entre les pr�dictions et les valeurs r�elles.

Un premier crit�re d'�valuation d'un mod�le de r�gression est, nous l'avons vu
� plusieurs reprises, l'\textbf{erreur quadratique moyenne}, ou \textbf{MSE} de
l'anglais \textbf{mean squared error}, � savoir
\begin{equation*}
  \text{MSE} = \frac1n \sum_{i=1}^n \left( f(\xx^i) - y^i \right)^2.
\end{equation*}
Des variantes sont d�crites dans la section~\ref{sec:regression_errors} ainsi que dans la \href{https://scikit-learn.org/stable/modules/model_evaluation.html#regression-metrics}{Documentation de \texttt{sklearn.metrics}}.

\section{R�gularisation}
\label{sec:generalization_regularization}
Le compromis biais-variance nous indique qu'un mod�le biais� peut �tre plus
pr�cis qu'un mod�le non-biais�. La \textbf{r�gularisation} consiste ainsi �
ajouter au risque empirique que l'on cherche � minimiser un terme, appel�
\textbf{r�gulariseur}, qui va biaiser le mod�le de sorte � ce que son risque
empirique soit plus �lev�e, mais son erreur de g�n�ralisation plus faible.

Plus un mod�le est simple, et moins il a de chances de surapprendre. Pour
limiter le risque de surapprentissage, il est donc souhaitable de limiter la
complexit� d'un mod�le. Ainsi, le r�gulariseur peut �tre vu comme un terme qui
mesure la complexit� du mod�le. La d�finition de la complexit� d'un mod�le est
une notion importante en th�orie de l'apprentissage, mais d�passe largement le
cadre de ce cours.

% Lorsque les variables sont fortement corr�l�es, ou que leur nombre d�passe
% celui des observations, la matrice $X \in \RR^{p+1}$ repr�sentant nos donn�es
% ne peut pas �tre de rang colonne plein. Ainsi, la matrice $X^\top X$ n'est pas
% inversible et il n'existe pas de solution unique � une r�gression lin�aire par
% minimisation des moindres carr�s. Il y a donc un risque de surapprentissage :
% le mod�le n'�tant pas unique, comment peut-on garantir que c'est celui que l'on
% a s�lectionn� qui g�n�ralise le mieux ?

% Pour limiter ce risque de surapprentissage, nous allons chercher � contr�ler
% simultan�ment l'erreur du mod�le sur le jeu d'entra�nement et les valeurs des
% coefficients de r�gression affect�s � chacune des variables. Contr�ler ces
% coefficients est une fa�on de contr�ler la complexit� du mod�le : comme nous le
% verrons par la suite, ce contr�le consiste � contraindre les coefficients �
% appartenir � un sous-ensemble strict de $\RR^{p+1}$ plut�t que de pouvoir
% prendre n'importe quelle valeur dans cet espace, ce qui restreint l'espace des
% solutions possibles.

En appelant $\Omega$ le r�gulariseur, la r�gularisation consiste donc �
remplacer la minimisation du risque empirique (eq.~\eqref{eq:erm}) par :
\begin{equation}
  \label{eq:regularisation}
  f \in \argmin_{h \in \FF} \frac{1}{n} \sum_{i=1}^n L(h(\xx^i), y^i) + 
  \lambda \Omega(h),
\end{equation}
o� le coefficient de r�gularisation $\lambda \in \RR_+$ contr�le
l'importance relative de chacun des termes.


% Dans le cas d'un mod�le de r�gression lin�aire, nous allons utiliser comme
% fonction de perte la somme des moindres carr�s. Les r�gulariseurs que nous
% allons voir sont fonction du vecteur de coefficients de r�gression $\bbeta$ :
% \begin{equation*}
%   \argmin_{\bbeta \in \RR^{p+1}} \left(\yy - X \bbeta \right)^\top 
%   \left(\yy - X \bbeta \right) + \lambda \Omega(\bbeta)
% \end{equation*}
% ou, de mani�re �quivalente, 
% \begin{equation}
%   \label{eq:regularisation_linreg}
%   \argmin_{\bbeta \in \RR^{p+1}} \ltwonorm{\yy - X \bbeta}^2 + \lambda \Omega(\bbeta).
% \end{equation}
% Nous utilisons ici la transformation~\ref{eq:added_ones} de $\xx$ qui
% consiste � ajouter � la matrice de design $X$ une colonne de 1 pour
% simplifier les notations.

% Le coefficient de r�gularisation $\lambda$ est un hyperparam�tre de la
% r�gression lin�aire r�gularis�e.

Quand $\lambda$ tend vers $+\infty$, le terme de r�gularisation prend de plus
en plus d'importance, jusqu'� ce qu'il domine le terme d'erreur et que seule
compte la minimisation du r�gulariseur : il n'y a plus d'apprentissage. % Dans la plupart des cas, le
% r�gulariseur est minimis� quand $\bbeta = \vec{0}$, et il n'y a plus
% d'apprentissage.
  
� l'inverse, quand $\lambda$ tend vers $0$, le terme de r�gularisation devient
n�gligeable devant le terme d'erreur, et la solution de
l'�quation~\eqref{eq:regularisation} est un minimiseur du risque
empirique.% $\bbeta$ prendra comme valeur une
% solution de la r�gression lin�aire non r�gularis�e.

Comme tout hyperparam�tre, $\lambda$ peut �tre choisi par validation
crois�e. On utilisera g�n�ralement une grille de valeurs logarithmique.

La suite de cette section pr�sente des exemples concrets de r�gularisation
appliqu�e � la r�gression lin�aire. Nous reprenons les notations de la
section~\ref{sec:linreg}.


\section{R�gularisation $\ell_2$  : r�gression ridge}
\label{sec:ridge_regression}
Une des formes les plus courantes de r�gularisation % , utilis�e dans de
% nombreux domaines faisant intervenir des probl�mes inverses mal pos�s,
consiste � utiliser comme r�gulariseur la norme $\ell_2$ du vecteur 
$\bbeta$ :  
\begin{equation}
  \label{eq:l2norm_reg}
  \Omega_{\text{ridge}}(\bbeta) = \ltwonorm{\bbeta}^2 = \sum_{j=0}^p \beta_j^2.
\end{equation}


On appelle \textbf{r�gression ridge} le mod�le $f: x \mapsto \bbeta^\top \xx$ dont
les coefficients sont obtenus par
\begin{equation}
  \label{eq:ridgereg}
  \argmin_{\bbeta \in \RR^{p+1}} \ltwonorm{\yy - X \bbeta}^2 + 
  \lambda \; \ltwonorm{\bbeta}^2.
\end{equation}    

\paragraph{Solution}
Le probl�me~\eqref{eq:ridgereg} est un probl�me d'optimisation convexe : il s'agit de minimiser une forme
quadratique. Il se r�sout en annulant le gradient en $\bbeta$ de la fonction
objective :
\begin{equation}
  \nabla_{\bbeta} \left( \ltwonorm{\yy - X \bbeta}^2 + 
    \lambda \ltwonorm{\bbeta}^2 \right) = 0
\end{equation}

En notant $I_p \in \RR^{p \times p}$ la matrice identit� en dimension $p,$ on
obtient :
\begin{equation}
  \left( \lambda I_p + X^\top X  \right) \bbeta^* = X^\top \yy.
\end{equation}
Comme $\lambda > 0$, la matrice $\lambda I_p + X^\top X$ est toujours
inversible. Notre probl�me admet donc toujours une unique solution
explicite. La r�gularisation par la norme $\ell_2$ a permis de transformer un
probl�me potentiellement mal pos� en un probl�me bien pos�, dont la solution
est :
\begin{equation}
  \label{eq:ridgereg_sol}
  \bbeta^* =  \left( \lambda I_p + X^\top X  \right)^{-1} X^\top \yy.
\end{equation}
  
% Si l'on multiplie la variable $x_j$ par une constante $\alpha$, le coefficient
% correspondant dans la r�gression lin�aire non r�gularis�e est divis� par
% $\alpha.$ En effet, si on appelle $X^*$ la matrice obtenue en rempla�ant $x_j$
% par $\alpha x_j$ dans $X$, la solution $\bbeta^*$ de la r�gression lin�aire
% correspondante v�rifie $X^* \left( \yy - X^* \bbeta^* \right) = 0$, tandis que
% la solution $\bbeta$ de la r�gression lin�aire sur $X$ v�rifie
% $X \left( \yy - X \bbeta \right) = 0.$ Ainsi, changer l'�chelle d'une variable
% a comme seul impact sur la r�gression lin�aire non r�gularis�e d'ajuster le
% coefficient correspondant de mani�re inversement proportionnelle.

% � l'inverse, dans le cas de la r�gression ridge, remplacer $x_j$ par
% $\alpha x_j$ affecte aussi le terme de r�gularisation, et a un effet plus
% complexe. L'�chelle relative des diff�rentes variables peut donc fortement
% affecter la r�gression ridge. Il est ainsi recommand� de \textbf{standardiser} les
% variables avant l'apprentissage, c'est-�-dire de toutes les ramener � avoir un
% �cart-type de 1 en les divisant par leur �cart-type :
% \begin{equation}
%   \label{eq:standardisation}
%   x_j^i \leftarrow \frac{x_j^i}{\sqrt{\frac1n \sum_{i=1}^n \left( x_j^i - 
%         \frac1n \sum_{i=1}^n x_j^i \right)^2}}
% \end{equation}    
% Attention : pour �viter le surapprentissage, il est important que cet
% �cart-type soit calcul� sur le jeu d'entra�nement uniquement, puis appliqu�
% ensuite aux jeux de test ou validation.

% La r�gression ridge a un effet de \og regroupement \fg~sur les variables
% corr�l�es, au sens o� des variables corr�l�es auront des coefficients
% similaires.

\paragraph{Chemin de r�gularisation}
On appelle \textbf{chemin de r�gularisation} l'�volution de la valeur du
coefficient de r�gression d'une variable en fonction du coefficient de
r�gularisation $\lambda$.

Le chemin de r�gularisation permet de comprendre l'effet de la r�gularisation
sur les valeurs de $\bbeta$. En voici un exemple sur la
figure~\ref{fig:ridge_path}.

\begin{figure}[h]
  \centering
  \includegraphics[width=0.5\textwidth]{figures/generalisation/ridge_path}
  \caption{Chemin de r�gularisation de la r�gression ridge pour un jeu de
    donn�es avec 12 variables. Chaque ligne repr�sente l'�volution du
    coefficient de r�gression d'une de ces variables quand $\lambda$ augmente :
    le coefficient �volue de sa valeur dans la r�gression non r�gularis�e vers
    0.}
  \label{fig:ridge_path}
\end{figure}

\paragraph{Interpr�tation g�om�trique}
�tant donn�s $\lambda \in \RR_+$, $X \in \RR^{n \times p}$ et $\yy \in \RR^n$,
il existe un unique $t \in \RR_+$ tel que le probl�me~\eqref{eq:ridgereg} soit
�quivalent �
\begin{equation}
  \label{eq:ridgereg_dual}
  \argmin_{\bbeta \in \RR^{p+1}} \ltwonorm{\yy - X \bbeta}^2 \text{ tel que }
  \ltwonorm{\bbeta}^2 \leq t.
\end{equation}
Preuve : L'�quivalence s'obtient par dualit� et en �crivant les conditions de
Karun-Kush-Tucker.
  
La r�gression ridge peut donc �tre formul�e comme un probl�me d'optimisation
quadratique (minimiser $\ltwonorm{\yy - X \bbeta}^2$) sous contraintes
($\ltwonorm{\bbeta}^2 \leq t$) : la solution doit �tre contenue dans la boule
$\ell_2$ de rayon $\sqrt{t}$. Sauf dans le cas o� l'optimisation sans
contrainte v�rifie d�j� la condition, cette solution sera sur la fronti�re de
cette boule, comme illustr� sur la figure~\ref{fig:l2reg}.

\begin{figure}[h]
  \centering
  \includegraphics[width=0.55\textwidth]{figures/generalisation/l2reg}
  \caption{La solution du probl�me d'optimisation sous
    contraintes~\eqref{eq:ridgereg_dual} (ici en deux dimensions) se situe sur
    une ligne de niveau de la somme des moindres carr�s tangente � la boule
    $\ell_2$ de rayon $\sqrt{t}$.}
  \label{fig:l2reg}
\end{figure}
  
\section{R�gularisation $\ell_1$ : lasso}
\label{sec:lasso}

\paragraph{Parcimonie}
Dans certaines applications, il peut �tre raisonnable de supposer que
l'�tiquette que l'on cherche � pr�dire n'est expliqu�e que par un nombre
restreint de variables. Il est dans ce cas souhaitable d'avoir un mod�le {\it
  parcimonieux}, ou \textbf{sparse}, c'est-�-dire dans lequel un certain nombre de
coefficients sont nuls : les variables correspondantes peuvent �tre retir�es du
mod�le.

Pour ce faire, on peut utiliser comme
r�gulariseur la norme $\ell_1$ du coefficient $\bbeta$ :
\begin{equation}
  \label{eq:l1norm_reg}
  \Omega_{\text{lasso}}(\bbeta) = \lonenorm{\bbeta} = \sum_{j=0}^p \lvert \beta_j \rvert.
\end{equation}

On appelle \textbf{lasso} le mod�le $f: x \mapsto \bbeta^\top \xx$ dont les
coefficients sont obtenus par
\begin{equation}
  \label{eq:lasso}
  \argmin_{\bbeta \in \RR^{p+1}} \ltwonorm{\yy - X \bbeta}^2 + 
  \lambda \lonenorm{\bbeta}.
\end{equation}    
Le nom de lasso est en fait un acronyme, pour \textbf{Least Absolute Shrinkage and
  Selection Operator} : il s'agit d'une m�thode qui utilise les valeurs {\it
  absolues} des coefficients (la norme $\ell_1$) pour r�duire (\textbf{shrink})
ces coefficients, ce qui permet de \textbf{s�lectionner} les variables qui
n'auront pas un coefficient nul. En traitement du signal, le lasso est aussi
connu sous le nom de \textbf{poursuite de base} (\textbf{basis pursuit} en anglais).

En cr�ant un mod�le parcimonieux et en permettant d'�liminer les variables
ayant un coefficient nul, le lasso est une m�thode de s�lection de variables
supervis�e. Il s'agit donc aussi d'une m�thode de r�duction de dimension.

\paragraph{Solution}
Le lasso~\ref{eq:lasso} n'admet pas de solution explicite. On pourra utiliser
un algorithme � directions de descente  pour le r�soudre. De plus, il ne s'agit pas
toujours d'un probl�me strictement convexe (en particulier, quand $p > n$) et
il n'admet donc pas n�cessairement une unique solution. En pratique, cela pose
surtout probl�me quand les variables ne peuvent pas �tre consid�r�es comme les
r�alisations de lois de probabilit� continues.
N�anmoins, % s'il est possible que plusieurs $\bbeta$ minimisent la
  % fonction objective du lasso, leur produit � gauche par $X$ vaut toujours la
  % m�me valeur (par convexit� stricte de la fonction de co�t) ; cela permet de
  % montrer que 
il est possible de montrer que les coefficients non nuls dans deux solutions
ont n�cessairement le m�me signe. Ainsi, l'effet d'une variable a la m�me
direction dans toutes les solutions qui la consid�rent, ce qui facilite
l'interpr�tation d'un mod�le appris par le lasso.
  
\paragraph{Interpr�tation g�om�trique}
Comme pr�c�demment, le probl�me~\ref{eq:lasso} peut �tre reformul� comme un
probl�me d'optimisation quadratique sous contraintes :

�tant donn�s $\lambda \in \RR_+$, $X \in \RR^{n \times p}$ et $\yy \in \RR^n$,
il existe un unique $t \in \RR_+$ tel que le probl�me~\ref{eq:lasso} soit
�quivalent �
\begin{equation}
  \label{eq:lasso_dual}
  \argmin_{\bbeta \in \RR^{p+1}} \ltwonorm{\yy - X \bbeta}^2 \text{ tel que }
  \lonenorm{\bbeta} \leq t.
\end{equation}

La solution doit maintenant �tre contenue dans la boule $\ell_1$ de rayon
$t$. Comme cette boule a des \og coins \fg, les lignes de niveau de la forme
quadratique sont plus susceptibles d'y �tre tangentes en un point o� une ou
plusieurs coordonn�es sont nulles (voir figure~\ref{fig:l1reg}).

\begin{figure}[h]
  \centering \includegraphics[width=0.55\textwidth]{figures/generalisation/l1reg}
  \caption{La solution du probl�me d'optimisation sous
    contraintes~\ref{eq:lasso_dual} (ici en deux dimensions) se situe sur
    une ligne de niveau de la somme des moindres carr�s tangente � la boule
    $\ell_1$ de rayon $t$.}
  \label{fig:l1reg}
\end{figure}


\paragraph{Chemin de r�gularisation}
Sur le chemin de r�gularisation du lasso (par exemple
figure~\ref{fig:lasso_path}, sur les m�mes donn�es que pour la
figure~\ref{fig:ridge_path}), on observe que les variables sortent du mod�le
les unes apr�s les autres, jusqu'� ce que tous les coefficients soient nuls. On
remarquera aussi que le chemin de r�gularisation pour n'importe quelle variable
est lin�aire par morceaux ; c'est une propri�t� du lasso.
\begin{figure}[h]
  \centering
  \includegraphics[width=0.5\textwidth]{figures/generalisation/lasso_path}
  \caption{Chemin de r�gularisation du lasso pour un jeu de donn�es avec 12
    variables. Chaque ligne repr�sente l'�volution du coefficient de r�gression
    d'une de ces variables quand $\lambda$ augmente : les variables sont
    �limin�es les unes apr�s les autres.}
  \label{fig:lasso_path}
\end{figure}

Si plusieurs variables corr�l�es contribuent � la pr�diction de l'�tiquette, le
lasso va avoir tendance � choisir une seule d'entre elles (affectant un poids
de 0 aux autres), plut�t que de r�partir les poids �quitablement comme la
r�gression ridge. C'est ainsi qu'on arrive � avoir des mod�les tr�s
parcimonieux. Cependant, le choix de cette variable est al�atoire, et peut
changer si l'on r�p�te la proc�dure d'optimisation. Le lasso a donc tendance �
�tre instable.

% \section{Exercice : SVM lin�aire pour la r�gression}
% \todo{}

\section{Compl�ments}
\subsection{Crit�res d'�valuation d'un mod�le de classification binaire d�riv�s de la matrice de confusion}
\label{sec:confusion_matrix_derived}

On appelle \textbf{vrais positifs} (en anglais \textit{true positives}) les
exemples positifs correctement classifi�s ; \textbf{faux positifs} (en anglais
\textit{false positives}) les exemples n�gatifs �tiquet�s positifs par le
mod�le ; et r�ciproquement pour les \textbf{vrais n�gatifs} (\textit{true
  negatives}) et les \textbf{faux n�gatifs} (\textit{false negatives}). On note
g�n�ralement par \textbf{TP} le nombre de vrais positifs, \textbf{FP} le nombre
de faux positifs, \textbf{TN} le nombre de vrais n�gatifs et \textbf{FN} le
nombre de faux n�gatifs.

Il est possible de d�river de nombreux crit�res d'�valuation � partir de la
matrice de confusion. En voici quelques exemples :

On appelle \textbf{rappel} (\textbf{recall} en anglais), ou \textbf{sensibilit�} ({\it
  sensitivity} en anglais), le taux de vrais positifs, c'est-�-dire la
proportion d'exemples positifs correctement identifi�s comme tels :
\begin{equation*}
  \text{Rappel} = \frac{\text{TP}}{\text{TP} + \text{FN}}.
\end{equation*}

Il est cependant tr�s facile d'avoir un bon rappel en pr�disant que \textbf{tous}
les exemples sont positifs. Ainsi, ce crit�re ne peut pas �tre utilis� seul. On
lui adjoint ainsi souvent la \textbf{pr�cision} :

On appelle \textbf{pr�cision}, ou \textbf{valeur positive pr�dictive} (\textbf{positive
  predictive value, PPV}) la proportion de pr�dictions correctes parmi les
pr�dictions positives :
\begin{equation*}
  \text{Pr�cision} = \frac{\text{TP}}{\text{TP} + \text{FP}}.
\end{equation*}

De m�me que l'on peut facilement avoir un tr�s bon rappel au d�triment de la
pr�cision, il est ais� d'obtenir une bonne pr�cision (au d�triment du rappel)
en faisant tr�s peu de pr�dictions positives (ce qui r�duit le risque qu'elles
soient erron�es)

L'anglais distingue \textbf{precision} (la pr�cision ci-dessus) et \textbf{accuracy},
qui est la proportion d'exemples correctement �tiquet�s, soit le compl�mentaire
� 1 du taux d'erreur, aussi traduit par \textbf{pr�cision} en fran�ais. On
utilisera donc ces termes avec pr�caution.

Pour r�sumer rappel et pr�cision en un seul nombre, on calculera la
\textbf{F-mesure} (\textit{F-score} ou \textit{F1-score} en anglais), qui est
la moyenne harmonique de la pr�cision et du rappel :
\begin{equation*}
  F = 2 \frac{\text{Pr�cision . Rappel}}{\text{Pr�cision} + \text{Rappel}} = 
  \frac{2 \text{TP}}{2 \text{TP} + \text{FP} + \text{FN}}.
\end{equation*}

Enfin, on appelle \textbf{sp�cificit�} le taux de vrais n�gatifs, autrement dit la
proportion d'exemples n�gatifs correctement identifi�s comme tels.
\begin{equation*}
  \text{Sp�cificit�} = \frac{\text{TN}}{\text{FP} + \text{TN}}.
\end{equation*}

\subsection{Erreurs de r�gression}
\label{sec:regression_errors}
Pour mesurer l'erreur dans la m�me unit� que celle des �tiquettes, on pr�f�re
souvent � l'erreur quadratique moyenne sa racine, g�n�ralement appel�e
\textbf{RMSE} de l'anglais \textbf{root mean squared error} :

\begin{equation*}
  \text{RMSE} = \sqrt{\frac1n \sum_{i=1}^n \left( f(\xx^i) - y^i \right)^2}.
\end{equation*}

L'interpr�tation de la RMSE requiert n�anmoins de conna�tre la distribution
des valeurs cibles : une RMSE de 1 cm n'aura pas la m�me signification selon
qu'on essaie de pr�dire la taille d'humains ou celle de drosophiles.  Pour
r�pondre � cela, il est possible de normaliser la somme des carr�s des r�sidus
non pas en en faisant la moyenne, mais en la comparant � la somme des distances
des valeurs cibles � leur moyenne.  On appelle \textbf{erreur carr�e relative},
ou \textbf{RSE} de l'anglais \textbf{relative squared error} la valeur
\begin{equation*}
  \text{RSE} = \frac{ \sum_{i=1}^n \left( f(\xx^i) - y^i \right)^2}{
    \sum_{i=1}^n \left( y^i - \frac1n \sum_{l=1}^n y^l \right)^2}.
\end{equation*}

Le compl�mentaire � 1 de la RSE est le \textbf{coefficient de d�termination}, not�
$R^2$.

On note le coefficient de d�termination $R^2$ car il s'agit du carr� du
coefficient de corr�lation de Pearson entre les pr�dictions et les valeurs r�elles. 




\begin{plusloin}
\item Au-del� des normes $\ell_1$ et $\ell_2$, il est possible d'utiliser des
  r�gulariseurs de la forme $\Omega_{\ell_q}(\bbeta) = ||\bbeta||_q^q.$
\item Une famille de r�gulariseurs appel�s \og structur�s \fg~permettent de
  s�lectionner des variables qui respectent une structure (graphe, groupes, ou
  arbre) donn�e a priori. Ces approches sont utilis�es en particulier dans des
  applications bio-informatiques, par exemple quand on cherche � construire des
  mod�les parcimonieux bas�s sur l'expression de g�nes sous l'hypoth�se que
  seul un petit nombre de voies m�taboliques (groupes de g�nes) est
  pertinent. Pour plus de d�tails, on se r�f�rera par exemple �
  \textit{Learning with structured sparsity}, J. Huang, T. Zhang \& D. Metaxas,
  Journal of Machine Learning Research 12:3371--3412 (2011).
\item Un ouvrage enti�rement consacr� au lasso et ses g�n�ralisations :
  \href{http://web.stanford.edu/~hastie/StatLearnSparsity/}{\textit{Statistical
      learning with sparsity: the {Lasso} and generalizations}}, T. Hastie,
  R. Tibshirani \& M. Wainwright (2015).
\item La r�gularisation est une technique importante des \textit{probl�mes
    inverses} que vous pourrez d�couvrir dans l'ES du m�me nom l'an prochain. 
\end{plusloin}

% \section{Points cl�s}
% \begin{itemize}
% \item Le compromis biais-variance traduit le compromis entre l'erreur
%   d'approximation, correspondant au biais de l'algorithme d'apprentissage, et
%   l'erreur d'estimation, correspondant � sa variance.
% \item La g�n�ralisation et le surapprentissage sont des pr�occupations
%   majeures en machine learning : comment s'assurer que des mod�les entra�n�s
%   pour minimiser leur erreur de pr�diction sur les donn�es observ�es seront
%   g�n�ralisables aux donn�es pour lesquelles il nous int�resse de faire des
%   pr�dictions ?
% \item Ajouter un terme de r�gularisation, fonction du vecteur des coefficients
%   $\bbeta$, au risque empirique de la r�gression lin�aire permet d'�viter le
%   surapprentissage
% \item La r�gression ridge utilise la norme $\ell_2$ de $\bbeta$ comme
%   r�gulariseur ; elle admet toujours une unique solution analytique, et a un
%   effet de regroupement sur les variables corr�l�es.
% \item Le lasso utilise la norme $\ell_1$ de $\bbeta$ comme r�gulariseur ; il
%   cr�e un mod�le parcimonieux, et permet donc d'effectuer une r�duction de
%   dimension supervis�e.
% \end{itemize}


%%% Local Variables:
%%% mode: latex
%%% TeX-master: "sdd_2020_poly"
%%% End:

\chapter{Mod�les non-lin�aires}
%%-*- coding: iso-latin-1 -*-
\label{chap:nonlin}


\paragraph{Notions :} r�seaux de neurones artificiels, apprentissage profond,
arbres de d�cision et for�ts al�atoires, m�thodes � noyaux.
\paragraph{Objectifs p�dagogiques :} 
\begin{itemize}      
  \setlength{\itemsep}{3pt}
\item D�crire les similarit�s et diff�rences entre r�seaux de neurones artificiels et mod�les lin�aires ; 
\item Reconna�tre les cas d'application de l'apprentissage profond ; 
\item Mettre en \oe{}uvre un algorithme d'apprentissage ensembliste ; 
\item Utiliser l'astuce du noyau pour apprendre des mod�les non-lin�aires �
  partir des algorithmes lin�aires vus pr�c�demment.
\end{itemize}


%%% Local Variables:
%%% mode: latex
%%% TeX-master: "sdd_2020_poly"
%%% End:




\end{document}
