%-*- coding: iso-latin-1 -*-
\documentclass[french,11pt]{article}
\usepackage{babel}
\DecimalMathComma
% Emacs: to save in encoding iso-latin-1:
% C-x C-m f
% iso-latin-1

% aspell --lang=fr --encoding='iso-8859-1' -t check selection-modele.tex

\usepackage{graphicx}
\usepackage[hidelinks]{hyperref}


% Fonts
\usepackage[latin1]{inputenc}
\usepackage[T1]{fontenc}
\usepackage{gentium}

% SI units
\usepackage{siunitx}

% Table becomes Tableau
\usepackage{caption}
\captionsetup{labelfont=sc}
\def\frenchtablename{Tableau}

% % List management
\usepackage{enumitem}

\usepackage[dvipsnames]{xcolor}
\usepackage{listings}
\lstset{%
  frame=single,                    % adds a frame around the code
  tabsize=2,                       % sets default tabsize to 2 spaces
  columns=flexible,                % doesn't add spaces to make the line fit the whole column
  basicstyle=\ttfamily,             % use monospace
  keywordstyle=\color{MidnightBlue},
  commentstyle=\color{Gray},
  stringstyle=\color{BurntOrange},
  showstringspaces=false,
}


%%%% GEOMETRY AND SPACING %%%%%%%%%%%%%%%%%%%%%%%%%%%%%%%%%%%%%%%%%%%%%%%

% % List management
% \usepackage{enumitem}
% \setlist{label=\textemdash,
%   itemsep=0pt, topsep=3pt, partopsep=0pt} 
% \setenumerate{itemsep=3pt,topsep=3pt,partopsep=0pt}

\usepackage{etex}
\usepackage[tmargin=2cm,bmargin=2cm,lmargin=2cm,footnotesep=1cm]{geometry}

\parskip=1ex\relax % space between paragraphs (incl. blank lines)

% % Headers and footers
% \pagestyle{myheadings}
% \markright{ECUE2.1 Science des donn�es \hfill PC 1 (6 mai 2020) \hfill} 
%%%%%%%%%%%%%%%%%%%%%%%%%%%%%%%%%%%%%%%%%%%%%%%%%%%%%%%%%%%%%%%%%%%%%%

%%% SYMBOLS %%%%%%%%%%%%%%%%%%%%%%%%%%%%%%%%%%%%%%%%%%%%%%%%%%%%%%%%%%
% % Math symbols
\usepackage{amsmath}
\usepackage{amssymb}
\usepackage{bm}

% Math symbols
\newcommand{\Acal}{\mathcal{A}}
\newcommand{\Bcal}{\mathcal{B}}
\newcommand{\Ccal}{\mathcal{C}}
\newcommand{\Ecal}{\mathcal{E}}
\newcommand{\Gcal}{\mathcal{G}}
\newcommand{\Ical}{\mathcal{I}}
\newcommand{\Kcal}{\mathcal{K}}
\newcommand{\Lcal}{\mathcal{L}}
\newcommand{\Ncal}{\mathcal{N}}
\newcommand{\Pcal}{\mathcal{P}}
\newcommand{\Qcal}{\mathcal{Q}}
\newcommand{\Rcal}{\mathcal{R}}
\newcommand{\Scal}{\mathcal{S}}
\newcommand{\Tcal}{\mathcal{T}}

\newcommand{\DD}{\mathcal{D}}
\newcommand{\FF}{\mathcal{F}}
\newcommand{\HH}{\mathcal{H}}
\newcommand{\LL}{\mathcal{L}}
\newcommand{\MM}{\mathcal{M}}
\newcommand{\OO}{\mathcal{O}}
\newcommand{\TT}{\mathcal{T}}
\newcommand{\UU}{\mathcal{U}}
\newcommand{\XX}{\mathcal{X}}
\newcommand{\YY}{\mathcal{Y}}
\newcommand{\ZZ}{\mathcal{Z}}

\newcommand{\CC}{\mathbb{C}}
\newcommand{\EE}{\mathbb{E}}
\newcommand{\NN}{\mathbb{N}}
\newcommand{\PP}{\mathbb{P}}
\newcommand{\RR}{\mathbb{R}}
\newcommand{\VV}{\mathbb{V}}


% Vectors
\makeatletter
\newcommand{\avec}{\vec{a}\@ifnextchar{^}{\,}{}}
\newcommand{\bb}{\vec{b}\@ifnextchar{^}{\,}{}}
\newcommand{\cvec}{\vec{c}\@ifnextchar{^}{\,}{}}
\newcommand{\gvec}{\vec{g}\@ifnextchar{^}{\,}{}}
\newcommand{\hh}{\vec{h}\@ifnextchar{^}{\,}{}}
\newcommand{\mm}{\vec{m}\@ifnextchar{^}{\,}{}}
\newcommand{\oo}{\vec{o}\@ifnextchar{^}{\,}{}}
\newcommand{\pp}{\vec{p}\@ifnextchar{^}{\,}{}}
\newcommand{\rr}{\vec{r}\@ifnextchar{^}{\,}{}}
\newcommand{\tvec}{\vec{t}\@ifnextchar{^}{\,}{}}
\newcommand{\uu}{\vec{u}\@ifnextchar{^}{\,}{}}
\newcommand{\uprime}{\vec{u'}\@ifnextchar{^}{\,}{}}
\newcommand{\vv}{\vec{v}\@ifnextchar{^}{\,}{}}
\newcommand{\vprime}{\vec{v'}\@ifnextchar{^}{\,}{}}
\newcommand{\ww}{\vec{w}\@ifnextchar{^}{\,}{}}
\newcommand{\xx}{\vec{x}\@ifnextchar{^}{\,}{}}
\newcommand{\yy}{\vec{y}\@ifnextchar{^}{\,}{}}
\newcommand{\zz}{\vec{z}\@ifnextchar{^}{\,}{}}
\newcommand{\aalpha}{\vec{\alpha}\@ifnextchar{^}{\,}{}}
\newcommand{\bbeta}{\vec{\beta}\@ifnextchar{^}{\,}{}}
\newcommand{\ttheta}{\vec{\theta}\@ifnextchar{^}{\,}{}}
\newcommand{\mmu}{\vec{\mu}\@ifnextchar{^}{\,}{}}
\newcommand{\xxi}{\vec{\xi}\@ifnextchar{^}{\,}{}}
\makeatother

% Hats
\newcommand{\thetahat}{{\widehat \theta}}
\newcommand{\hatn}{{\widehat n}}
\newcommand{\hatp}{{\widehat p}}
\newcommand{\hatm}{{\widehat m}}
\newcommand{\hatmu}{{\widehat \mu}}
\newcommand{\hatsigma}{{\widehat \sigma}}

\newcommand{\hatmle}[1]{\widehat{#1}_{\text{MLE}}}
\newcommand{\hatbys}[1]{\widehat{#1}_{\text{Bayes}}}

\DeclareMathOperator*{\argmax}{arg\,max}
\DeclareMathOperator*{\argmin}{arg\,min}

\newcommand{\cvn}{\xrightarrow[n \rightarrow +\infty]{}}
\newcommand{\cvproba}{\stackrel{\PP}{\longrightarrow}}
\newcommand{\cvps}{\stackrel{\text{p.s.}}{\longrightarrow}}
\newcommand{\cvltwo}{\stackrel{\Lcal^2}{\longrightarrow}}
\newcommand{\cvloi}{\stackrel{\Lcal}{\longrightarrow}}

% \newcommand{\lzeronorm}[1]{\left|\left|#1\right|\right|_0}
% \newcommand{\lonenorm}[1]{\left|\left|#1\right|\right|_1}
% \newcommand{\ltwonorm}[1]{\left|\left|#1\right|\right|_2}
% \newcommand{\lzeronorm}[1]{\|#1\|_0}
% \newcommand{\lonenorm}[1]{\|#1\|_1}
% \newcommand{\ltwonorm}[1]{\|#1\|_2}
\usepackage{mathtools}
\DeclarePairedDelimiter{\abs}{\lvert}{\rvert}
\DeclarePairedDelimiter{\norm}{\lVert}{\rVert}
\DeclarePairedDelimiter{\bignorm}{\bigg\lVert}{\bigg\rVert}
\DeclarePairedDelimiter{\innerproduct}{\langle}{\rangle}

\DeclarePairedDelimiter{\lzeronorm}{\lVert}{\rVert_0}
\DeclarePairedDelimiter{\lonenorm}{\lVert}{\rVert_1}
\DeclarePairedDelimiter{\ltwonorm}{\lVert}{\rVert_2}
%%%%%%%%%%%%%%%%%%%%%%%%%%%%%%%%%%%%%%%%%%%%%%%%%%%%%%%%%%%%%%%%%%%%%%


\begin{document}

\begin{center}
\bf\large ECUE21.1: Science des donn�es \hfill
QCM 3 -- Construction d'estimateurs
\end{center}

\noindent
\hfill 29 avril 2020

\noindent
\rule{\textwidth}{.4pt}

\medskip


\paragraph{Question 1.} Soit $(x_1, x_2, \dots, x_n)$ un �chantillon d'une
variable al�atoire $X$. On suppose que $X$ suit une loi param�tris�e
par $\gamma$. La vraisemblance de $(x_1, x_2, \dots, x_n)$ est donn�e par
\begin{itemize}
\item[$\square$] $\PP(x_1, x_2, \dots, x_n, \gamma)$
\item[$\square$] $\PP(x_1, x_2, \dots, x_n | \gamma)$
\item[$\square$] $\PP(\gamma | x_1, x_2, \dots, x_n)$
\item[$\square$] $\prod_{i=1}^n \PP(x_i|\gamma)$
\item[$\square$] $\prod_{i=1}^n \PP(\gamma|x_i)$
\end{itemize}

\paragraph{Question 2.} Soit $X$ une loi exponentielle de param�tre
$\lambda$. L'estimateur par maximum de vraisemblance de $\lambda$ est donn� par
\begin{itemize}
\item[$\square$] $L_n = n \ln(\lambda) - \lambda \sum_{i=1}^n X_i,$ o� $(X_1, X_2, \dots, X_n)$ est un �chantillon al�atoire de $X$
\item[$\square$] $\widehat{\lambda} = n \ln(\lambda) - \lambda \sum_{i=1}^n x_i,$ o� $(x_1, x_2, \dots, x_n)$ est un �chantillon al�atoire de $X$
\item[$\square$] $L_n = \frac{n}{\sum_{i=1}^n X_i},$ o� $(X_1, X_2, \dots, X_n)$ est un �chantillon al�atoire de $X$
\item[$\square$] $\widehat{\lambda} = \frac{n}{\sum_{i=1}^n x_i},$ o� $(x_1, x_2, \dots, x_n)$ est un �chantillon al�atoire de $X$.
\end{itemize}

\paragraph{Question 3. $\bigstar$} L'estimateur de Bayes est plus proche de l'esp�rance a
priori que de l'estimateur par maximum de vraisemblance quand la taille de
l'�chantillon est
\begin{itemize}
\item[$\square$] grande
\item[$\square$] petite
\item[$\square$] �a d�pend.
\end{itemize}


\newpage

\section*{Solution}

\paragraph{Question 1.} Par d�finition (cf. �quation~(3.7) du poly),
\[
L(x_1, x_2, \dots, x_n; \gamma) = \PP(x_1, x_2, \dots, x_n | \gamma) = \prod_{i=1}^n \PP(x_i|\gamma).
\]

\paragraph{Question 2.} 
Par d�finition la vraisemblance d'un �chantillon $(x_1, x_2, \dots, x_n)$ est donn�e par
\[
  L(x_1, x_2, \dots, x_n; \lambda) = \prod_{i=1}^n \lambda e^{- \lambda x_i}
  = \lambda^n \prod_{i=1}^n e^{- \lambda x_i},
\] 
et donc sa \textit{log-vraisemblance} vaut 
\[
  \ell(x_1, x_2, \dots, x_n; \lambda) = \ln \left(\lambda^n \prod_{i=1}^n e^{- \lambda x_i}\right)
  = n \ln(\lambda) - \lambda \sum_{i=1}^n x_i.
\]
La fonction $\lambda \mapsto n \ln(\lambda) - \lambda \sum_{i=1}^n x_i$ est
concave sur $]0, +\infty[ \rightarrow \RR$ et on peut donc la maximiser en
annulant sa d�riv�e.

On obtient \textit{l'estimation par maximum de vraisemblance} de $\lambda$ suivante :
\[
  \hatmle{\lambda} = \frac{n}{\sum_{i=1}^n x_i}
\]
et, si on appelle $(X_1, X_2, \dots, X_n)$ un �chantillon al�atoire de $X$, on
obtient \textit{l'estimateur par maximum de vraisemblance} de $\lambda$ :
\[
  L_n = \frac{n}{\sum_{i=1}^n X_n}.
\]


\paragraph{Question 3.} La tendance que nous avons observ�e sur l'exemple de la
section 3.6 (cf. � Remarque importante �) se v�rifie en g�n�ral : plus on
observe d'�chantillons, plus on s'�loigne de l'a priori pour se rapprocher d'un
estimateur issu uniquement des donn�es.
\end{document}
